
% LaTeX Article Template - using defaults
\documentclass{article}
%\setlength{\topmargin}{-0.75in}         % move margin up 0.75in
%\setlength{\textwidth}{6.5in}      % text width 6.5in
%\setlength{\textheight}{9.25in}         % textlength 9.25in
%\setlength{\oddsidemargin}{0in}    % have left margin 1in over
%\setlength{\baselineskip}{0.1in}

\usepackage{graphicx} %images
\usepackage{listings} %code
\usepackage{amsmath} %pmatrix
\usepackage{comment} %multiline comments
\usepackage{algpseudocode} % creating proceedures
\usepackage{verbatim}
\usepackage{url}
\usepackage{fullpage}
\usepackage{epigraph}

\newcommand{\includegraphicsfit}[1]
{\includegraphics[width=\textwidth,height=\textheight,keepaspectratio]{#1}}

\newcommand{\includegraphicssmall}[1]
{\includegraphics[width=.25\textwidth,height=.25\textheight,keepaspectratio]{#1}}

\def\signed#1{{\leavevmode\unskip\nobreak\hfil\penalty50\hskip2em
  \hbox{}\nobreak\hfil\raise-3pt\hbox{(#1)}%
  \parfillskip=0pt \finalhyphendemerits=0 \endgraf}}

\begin{document}
\begin{center}
\textbf{Real World Application of the Parameter-less Population Pyramid}
\end{center}

Real world applications of evolutionary optimization are often
hindered by the need to determine problem specific parameter
settings.  
Parameters can have a significant impact on search efficiency~\cite{goldberg:1991:gasize}
and finding good settings can be very time consuming~\cite{grefenstette:1986:optimalga}.
While some previous methods have reduced~\cite{Back:1992:selfadapt}
or removed~\cite{pelikan:2004:parameterlesshboa} the
need for parameter tuning, many do so by trading efficiency for
general applicability.  The Parameter-less Population Pyramid (P3)
is an evolutionary technique that requires \textbf{no parameters}
and is still broadly effective. P3 strikes a balance between
continuous integration of diversity and exploitative elitist
operators, allowing it to solve easy problems quickly and hard
problems eventually.  Work currently under review for publication
compared P3 with three optimally tuned state-of-the-art algorithms~\cite{thierens:2013:ltgahiff,doerr:2013:lambdalambda}.
Across seven problem classes P3 always found the
optimum at least a constant factor \emph{\textbf{faster}} than all comparisons.
More importantly, on the three most challenging problem classes,
P3 had a lower order of computational complexity as measured by
evaluations.  Based on over one trillion evaluations, these results
suggest P3 has wide applicability to a broad class of
problems.  \textbf{Therefore, the primary focus of this project is to integrate
P3 into X-TOOLSS in order to (1) allow non-experts easy access to P3,
(2) improve P3's ability to optimize real world problems and (3) expand
P3 to new evolutionary domains utilizing Genetic Programming.}

P3' driving principle is to make
minimal assumptions about the search landscape while maximally
exploiting gathered information.  As such, P3 does not use a fixed
population size.  Unlike most evolutionary optimization systems, P3 uses multiple
populations, with each population akin to a generation. 
The optimization process begins with a single randomly generated solution.
Having no other knowledge of the search landscape, P3 uses local search
to bring that solution to an optima.  This solution is then stored in $F_0$
(the first population).  Without making further assumptions
about the search landscape, P3 chooses another random solution to optimize using
local search, which is again stored in $F_0$.  Given two solutions in $F_0$, P3 can then
perform an entropy based optimal mixing crossover~\cite{bosman:2011:lsbbo}.  The
result of this mixing is then put into $F_1$.  As this new solution has nothing
to cross with, the process begins again at a random solution.  In this way new
solutions are added and crossed with existing levels of the population pyramid,
with new levels being created as needed.  This allows P3 to simultaneously exploit
all of its evaluations to produce a single high quality solution while avoiding convergence through
constant addition of new genetic material.

A challenge for all theory based optimization techniques is to make the transition
to solving meaningful problems.  While the field has progressed significantly beyond
its origins, many practitioners use outdated evolutionary systems~\cite{bechmann:2013:hummies}
or rely on methods tailored to their specific problem~\cite{izzo:2013:hummies}.
The fact that these suboptimal optimizers can already outperform human designers
suggests that bridging the gap between academia and industry may yield spectacular results.

There are two primary steps to making P3 an effective optimizer in practice: accessibility and adoption.
To achieve these goals we plan to integrate P3 into X-TOOLSS.\footnote{\url{http://nxt.ncat.edu/}}
X-TOOLSS is an open source optimization framework developed by Dr. Dozier's lab at North Carolina A\&T.
This framework consists of a number of canonical evolutionary optimization techniques, and is under
active use by engineers at NASA.  X-TOOLSS provides an easy to use interface which allows non-computer
scientists to apply evolutionary optimization to their domain specific problem.  Bundled with the
software are data collection and visualization tools which facilitate deeper understanding of
the optimization process and comparisons between optimization techniques.

To gauge the interest of X-TOOLSS users, with the help of Dr. Dozier, we asked Adam Burt, a NASA
Aerospace Engineer, to try out P3.  Here is his unedited response comparing P3 to his past work
using X-TOOLSS's basic evolutionary optimizers:

\begin{quote}
So I wanted to let you know that P3 appears to be doing incredibly well on the topology optimization problem.  Typically for 200ish elements it takes around 8-9000 function evaluations to find a good solution. P3 does it in roughly 600!!  I have already shared my enthusiasm with a co-worker of mine who has also done GA based topology optimization and he seemed fairly impressed as well.  I think you might really have something here.  I am going to do a few paper searches to see if I can find other authors who have done similar problems to see how many function evaluations they have on average, but I feel like 600 ish is very small for a GA.  My co-worker was also pretty excited about the prospect of a tuneless GA, I believe that will have a lot of benefit in the engineering application world. \signed{Adam Burt, NASA Aerospace Engineer}
\end{quote}

Adam has graciously worked with us to perform these preliminary experiments, but more general
release is currently hindered by the expert centric design of P3's software.  Even now that Adam
knows how to use our software, it requires him to have two separate tool sets for optimization.
Therefore we feel it is imperative to integrate P3 into X-TOOLSS to make it more user friendly,
allowing more engineers to benefit from P3's quality.

Beyond accessibility, X-TOOLSS provides an invaluable potential to P3 for academic data collection.
The framework's optimization analysis tools have been refined through continuous multi-user use, and
will likely lead to further insights into P3's optimization.  Furthermore, as X-TOOLSS has a number
of built in evolutionary optimizers, it provides a perfect device for gathering comparative results
as P3 is applied to new and challenging problems.  Once integrated, non-computer scientists can utilize
X-TOOLSS to compare P3 directly with existing evolutionary techniques.  This can provide the foundation
for collaborative research between industry and academia, with industry gaining high quality solutions
to challenging problems and academia gathering data on how to improve optimization techniques.

%\includegraphicsfit{WithRast}

P3 is currently limited to evolving discrete value encoded linear genomes. 
Therefore we propose to leverage Grammatical Evolution~\cite{oneill:2001:grammatical}
to give P3 the ability to evolve entire programs.  Grammatical Evolution (GE) is a subset
of Genetic Programming which utilizes a non-trivial genotype to phenotype mapping, converting
binary strings into full fledged programs.  To apply GE, a domain expert first creates a
grammar which defines the structure of all possible useful programs.  When evaluating a
solution, bits are read from the individual to determine which grammar expansion rules to perform,
deterministically creating a program.  This is an ideal transition for P3 as from an
evolutionary standpoint GE applications will be no different than binary string applications as
only the evaluation step will have to change.

\subsubsection*{Intellectual Merit}
%Proposed work is highly innovative and has potential for having a ground-breaking impact in its field
%Project aligns with BEACON mission and shows promise of opening new areas for multidisciplinary investigation
%Project appears very likely to generate successful (funded) proposals
%Degree of Multidisciplinarity of Project - No chance

All previous parameter-less methods in evolutionary computing perform at least a constant factor
worse than when using methods with optimal problem specific parameter tuning~\cite{pelikan:2004:parameterlesshboa}.
What makes P3 so interesting as that in all of our preliminary data it has performed at
least a constant factor \textbf{better} than optimally tuned state-of-the-art algorithms.
However, these results are limited to artificial problems.  In order to perform a complete analysis,
P3 needs to be applied to real world optimization problems.  Through X-TOOLSS and 
with the help of engineers at NASA, this
analysis can be performed with relative ease.

Once collected, P3's ability to tackle real problems will provide either further evidence for its usefulness
to people outside of the evolutionary optimization community or as a means for further improving P3.
In either case it will help drive the direction of future research in parameter-less evolutionary optimization.
Expanding P3 to evolve entire programs will also provide interesting insight into the challenging domain
of Genetic Programming.


\subsubsection*{Broader Impacts}
%Education - Not likely
%Project is organized around solution of a problem identified by a particular company/group of companies as important to them
%Some diversity reflected among project participants (current or named for future appointment under project)
%Project participants are from two or more BEACON institutions
Once P3 has been integrated into X-TOOLSS, its parameter-less nature will allow
end users (specifically NASA engineers) to try P3 on their optimization problems with ease.
Our preliminary results suggest P3 will produce better quality solutions in less time
than existing X-TOOLSS optimizers.

Beyond the direct benefits gained by NASA and X-TOOLSS users, better understanding of P3
can improve evolutionary optimization in general.  As mentioned previously, evolutionary
methods are already able to produce human competitive designs, with enough successes each
year to hold an annual competition.\footnote{\url{http://www.genetic-programming.org/combined.php}}
These results impact everything from space exploration to cancer screenings, software error correction
to protein structure prediction.  Evolution of programs can help P3 create truly innovative solutions to complex problems.


\newpage
\bibliographystyle{abbrv}
\bibliography{../main}
\end{document}

