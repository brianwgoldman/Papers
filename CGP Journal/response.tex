\documentclass{letter}
\usepackage{hyperref}
\usepackage{verbatim}
\usepackage[T1]{fontenc}
\usepackage{fullpage}
\signature{Brian W. Goldman and William F. Punch}
%\address{21 Bridge Street \\ Smallville \\ Dunwich DU3 4WE}
\begin{document}

\begin{letter}{}
\opening{Greetings Editors and Reviewers:}

First, we would like to thank both the reviewers and the editorial staff for their
time in examining our work.  The comments we received were very helpful and
have lead to significant improvements.

Before covering individual points, we would like to respond to the two
reviewers who suggested we could improve the support of our conclusions by
including problems outside the binary domain.  We completely agree, and as
noted by Reviewer 3 we feel like this is an excellent avenue for future work.  However,
the focus of this work was to delve deeply into a small set of problems to
develop hypotheses about CGP's inherent behavior.  To that
end we felt that increasing the paper's breadth would limit our depth.  As was
pointed out by two reviewers, our work required extensive experimentation
to achieve our conclusions on a small set of problems.  More directly, to
include a single new problem (boolean or otherwise) with
comparable analysis would require over four thousand additional runs.
%including proper parameter tuning and data collection.

The reason we chose binary problems was to allow for the
detailed analysis of behavior in the latter half of Section VI.  If we had done
four problems, but using diverse domains, it would have been much more challenging
to determine why we observed changes in behavior between problems.  By using
four problems with similar domains and the same operators, we were able to determine
how our techniques scale with problem difficulty.  Similarly, the analysis tools
we developed to perform the comparison allowed us to rely on problem uniformity.
Finally, since the binary domain is
the most studied and successful in CGP literature, detailed analysis and conclusions drawn there
are likely to have the most impact on the field.

We would like to reiterate that our thesis is understanding
CGP's problem solving behaviors and not necessarily to suggest any of our modifications
as universal improvements.  We agree that the inclusion of wall time comparisons on a diverse
set of hard real world problems can be very informative when defending new methods
for optimization, but we strive more for comprehension than improvement.

\begin{comment}
A major focus of our work (ordering effect on the location and number of active genes)
is grounded in theory which makes no assumptions on the type of problem being solved.
Furthermore, the operators we investigate deal with domain independent problems.
As the experimental results matched the theory, we have no reason to suspect our
conclusions on these topics to change when the operators or data type is changed.
However, we understand that the following could change depending on data type: The relative effectiveness
of the proposed modifications (IE, skip vs single, reorder vs normal, etc), the
evolved genome sizes, and the charts in Table IV and Table V.  Yet we do not feel
these are the primary concern of the work, nor do we have any reason to believe
the conclusions should only hold true for binary problems.
\end{comment}

With that said, we would now like to provide point by point responses to the reviewer
comments.

\textbf{Reviewer 1:}

\begin{quote}
You may find it useful to at least mention other representations of CGP in
which the dependency of node ordering has been reduced, or even, removed.
\end{quote}
Page 3, right column, last paragraph before Section Reorder: This paragraph was
added with appropriate citations to a few versions of CGP which do not follow Equation 1.

\begin{quote}
When comparing your results with previous publications [8], why didn't you used the same function set?
\end{quote}
In [8], the authors use AND, OR, NAND, NOR for all of the other boolean problems tested (Even Parity, Digital Adder, and Digital Comparitor),
but use AND, AND with one input inverted, XOR and OR on digital multiplier.
The two papers [8] cites that performed research on this problem used AND, XOR.  They claim this makes the problem easier as
human designed solutions use only AND, XOR.  Therefore eliminating XOR as a choice likely makes this problem even harder.
That said, our paper chose to keep the operators consistent across problems.
However, Page 8, left column we have added more
information about this comparison.  Notably, we used our parameters with their operators
and maintained Reorder as being statistically better.

\begin{quote}It is unfortunate that all analysis was limited to boolean problems\end{quote}
Covered in general response.

\begin{quote}[I]t may be helpful to make it clear that Skip, Accumulate etc. are your invention\end{quote}
For terseness we felt that as our abstract states our focus was on ``creating methods''
combined with fact that these methods both exist in previous published work, no modification
was necessary.

\begin{quote}[I]n many CGP implementations it is not common to exclude the gene's current value during mutation.\end{quote}
Page 1, right column, last paragraph: We have corrected this to point out both that it isn't
so much common as unclearly defined and that our choice was made to allow
for precise behavioral analysis.  While the distinction may result in minor differences in behavior,
we see no reason for it to effect any conclusions drawn about CGP's aggregate behavior.

\begin{quote}Once all of a *node's* dependencies...\end{quote}
Fixed.

\begin{quote}References or at least a very brief description of the purpose of for
Kruskal-Wallis (as for the Mann-Whitney U test) would aid those not familiar.\end{quote}
Page 7, Footnote 3 and 4 have been added.


\textbf{Reviewer 2:}

\begin{quote}Please provide the reasons why the interaction is important? Also, please explain (briefly) how your goals can be achieved.\end{quote}
Page 1, left column, last sentence as been added.

\begin{quote}[W]hy only ``one'' active gene is mutated?\end{quote}
Page 3, left column, last paragraph before Section Genome Reordering: Added explanation that more than
one could have been used, but one is sufficient and removes a parameter.

\begin{quote}``and defined Normal as CGP's historical ordering method'', do you mean the ordering method of the original CGP?\end{quote}
Yes.  Page 3, right column, paragraph starting with ``To rectify these'', last sentence clarifies our meaning.

\begin{quote}``While this increase ... drawback.'', this sentence is not entirely clear to me. Please elaborate more and provide reasons for this statement.\end{quote}
Page 4, right column, last paragraph before Section DAG.  Last few sentences expand on the idea.

\begin{quote}``mutating node'' or ``mutated node''?\end{quote}
Mutating is in fact intended here.  This section is discussing the process of performing
a mutation on a node, so at this point in the description it is ``mutating.''

\begin{quote}It would be nice to give some rationales why potential configurations run 5 times and later 4 times? Why are higher values used?\end{quote}
Page 6, right column, last full paragraph, starting from ``We chose to run each initially 5 times.''

\begin{quote}``we would suggest Skip be used over accumulate for any real world application.'' is quite a strong claim. Only binary problems are examined in the experiments, is it enough to support your claim?\end{quote}
Page 7, right column, last full paragraph: Added some language to clarify our reasoning.  From what we can tell
the methods are different ways of doing the exact same thing.  Since Accumulate is more complicated, we suggest Skip.

\begin{quote}If possible, please provide the average, standard deviation of ``Evaluations to success'' and the average running time of each configuration.\end{quote}
We further examined the statistical measures
of evaluations to success which we report in Table II.  As it is not safe to assume the number
of evaluations to success is normally distributed (in fact, they are most likely not normal),
we chose to provide non-parametric (median) instead of parametric (mean) measures,
as was done in [8].  Upon our review, we decided Median Absolute Deviation was not
capturing useful information, and we now show bootstrap confidence intervals, which is a much
more statistically sound and interesting value.

As to including the wall clock time for our experiments, we attempted to cover our
reasoning for not doing this starting on Page 8, right column, continuing to the end of the section.
In order to provide a fair comparison of runtime, we would need to redo tuning
with a focus on minimizing wall clock time instead of number of evaluations.
An additional
problem in reporting this value is hardware diversity.  To complete our experiments
we utilized a computing cluster with diverse hardware quality, and running all experiments
on uniform hardware is infeasible for us at this time.


\textbf{Reviewer 3:}

\begin{quote}Although I am not suggesting any more work be undertaken. It would have been nice
to see more work done on harder problems. ... The paper has focused on Boolean problems and it would be interesting to
carry out similar analysis in other domains. Of course, this remains for the future.\end{quote}
Covered in general response.

\begin{quote}Mutation is generally carried out in two ways [rate vs probability]...The two methods require markedly
different amounts of computation...\end{quote}
It is possible to achieve O(1) amortized cost when performing mutations using the probability method.
This can be done leveraging the binomial distribution, which in many languages is implemented to give
the random number of trials k of N total which are true given probability P.  k genes can then
be selected from the genome at random just as when using a mutation rate.  This operator is
advantageous in that allows for a flexible number of mutating genes, up to and including
mutating all genes in a single step.  We alluded to this Page 9, left column, first full paragraph.
Note, if we were to extend Single to use a configurable
number of genes, this would be our choice of method.  As such we would disagree that the probability method
by necessity requires more computation than rate.

That said, for clarity we have corrected all references to mutation ``rate'' and replaced them with ``probability.''

\begin{quote}Another quite minor point is that as stated in Miller's book (page 28) when a function gene is selected for mutation, no
check is made as to whether the mutated function is the same as that before mutation\end{quote}
Page 1, right column, last paragraph: We have clarified this.  See response to similar comment by Reviewer 1.

\begin{quote}``than'' should be ``then''\end{quote}
Fixed.

\begin{quote}It is worth noting in the paper that [checking active genes are changed] does incur a computational cost.\end{quote}
Page 2, left column, last full paragraph.

\begin{quote}I think the paper would benefit from some
remarks about the extra computational cost of carrying out the Accumulate strategy\end{quote}
Page 2, right column, last paragraph.

\begin{quote}Since one knows what the parent's active genes are, one could change an active gene at random.\end{quote}
Page 3, left column, last paragraph before Section Genome Ordering partially covers this.  In general
we felt the existing arguments about the importance of neutral drift covered why inactive genes are
also mutated.

\begin{quote}``...this representation likely...'' -> ``...this representation is likely...''\end{quote}
Fixed.

\begin{quote}``convenience'' -> ``convenient''\end{quote}
Page 4, left column, first paragraph: We decided the word convenient was not necessary and removed
it entirely.

\begin{quote}``effecting'' -> ``affecting''\end{quote}
Fixed.

\begin{quote}It is probably a Latex issue, but the caption TABLE 1 merges with the bottom line of the table.\end{quote}
The IEEE standard is to have captions above a table.  Our original submission had captions below
the tables, which was why there was a problem.  Caption locations for all tables have been corrected.

\begin{quote}making comparisons with modular CGP on 3-bit parity is of little value actually\end{quote}
That comparison has been removed as we agree it was not useful.

\begin{quote}``the are'' -> ``they are''\end{quote}
Fixed.

\begin{quote}If this is true, it may help to explain CGP uses such large genomes...\end{quote}
Fixed.

\begin{quote}``most effect'' -> ``Largest effect''?\end{quote}
Fixed.

\closing{Thank you again for your thoughtful consideration.}
\end{letter}

\end{document}
