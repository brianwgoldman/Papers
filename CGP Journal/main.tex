
%% bare_jrnl.tex
%% V1.4
%% 2012/12/27
%% by Michael Shell
%% see http://www.michaelshell.org/
%% for current contact information.
%%
%% This is a skeleton file demonstrating the use of IEEEtran.cls
%% (requires IEEEtran.cls version 1.8 or later) with an IEEE journal paper.
%%
%% Support sites:
%% http://www.michaelshell.org/tex/ieeetran/
%% http://www.ctan.org/tex-archive/macros/latex/contrib/IEEEtran/
%% and
%% http://www.ieee.org/



% *** Authors should verify (and, if needed, correct) their LaTeX system  ***
% *** with the testflow diagnostic prior to trusting their LaTeX platform ***
% *** with production work. IEEE's font choices can trigger bugs that do  ***
% *** not appear when using other class files.                            ***
% The testflow support page is at:
% http://www.michaelshell.org/tex/testflow/


%%*************************************************************************
%% Legal Notice:
%% This code is offered as-is without any warranty either expressed or
%% implied; without even the implied warranty of MERCHANTABILITY or
%% FITNESS FOR A PARTICULAR PURPOSE! 
%% User assumes all risk.
%% In no event shall IEEE or any contributor to this code be liable for
%% any damages or losses, including, but not limited to, incidental,
%% consequential, or any other damages, resulting from the use or misuse
%% of any information contained here.
%%
%% All comments are the opinions of their respective authors and are not
%% necessarily endorsed by the IEEE.
%%
%% This work is distributed under the LaTeX Project Public License (LPPL)
%% ( http://www.latex-project.org/ ) version 1.3, and may be freely used,
%% distributed and modified. A copy of the LPPL, version 1.3, is included
%% in the base LaTeX documentation of all distributions of LaTeX released
%% 2003/12/01 or later.
%% Retain all contribution notices and credits.
%% ** Modified files should be clearly indicated as such, including  **
%% ** renaming them and changing author support contact information. **
%%
%% File list of work: IEEEtran.cls, IEEEtran_HOWTO.pdf, bare_adv.tex,
%%                    bare_conf.tex, bare_jrnl.tex, bare_jrnl_compsoc.tex,
%%                    bare_jrnl_transmag.tex
%%*************************************************************************

% Note that the a4paper option is mainly intended so that authors in
% countries using A4 can easily print to A4 and see how their papers will
% look in print - the typesetting of the document will not typically be
% affected with changes in paper size (but the bottom and side margins will).
% Use the testflow package mentioned above to verify correct handling of
% both paper sizes by the user's LaTeX system.
%
% Also note that the "draftcls" or "draftclsnofoot", not "draft", option
% should be used if it is desired that the figures are to be displayed in
% draft mode.
%
\documentclass[journal]{IEEEtran}
%
% If IEEEtran.cls has not been installed into the LaTeX system files,
% manually specify the path to it like:
% \documentclass[journal]{../sty/IEEEtran}

%\linespread{2.5}



% Some very useful LaTeX packages include:
% (uncomment the ones you want to load)


% *** MISC UTILITY PACKAGES ***
%
%\usepackage{ifpdf}
% Heiko Oberdiek's ifpdf.sty is very useful if you need conditional
% compilation based on whether the output is pdf or dvi.
% usage:
% \ifpdf
%   % pdf code
% \else
%   % dvi code
% \fi
% The latest version of ifpdf.sty can be obtained from:
% http://www.ctan.org/tex-archive/macros/latex/contrib/oberdiek/
% Also, note that IEEEtran.cls V1.7 and later provides a builtin
% \ifCLASSINFOpdf conditional that works the same way.
% When switching from latex to pdflatex and vice-versa, the compiler may
% have to be run twice to clear warning/error messages.






% *** CITATION PACKAGES ***
%
%\usepackage{cite}
% cite.sty was written by Donald Arseneau
% V1.6 and later of IEEEtran pre-defines the format of the cite.sty package
% \cite{} output to follow that of IEEE. Loading the cite package will
% result in citation numbers being automatically sorted and properly
% "compressed/ranged". e.g., [1], [9], [2], [7], [5], [6] without using
% cite.sty will become [1], [2], [5]--[7], [9] using cite.sty. cite.sty's
% \cite will automatically add leading space, if needed. Use cite.sty's
% noadjust option (cite.sty V3.8 and later) if you want to turn this off
% such as if a citation ever needs to be enclosed in parenthesis.
% cite.sty is already installed on most LaTeX systems. Be sure and use
% version 4.0 (2003-05-27) and later if using hyperref.sty. cite.sty does
% not currently provide for hyperlinked citations.
% The latest version can be obtained at:
% http://www.ctan.org/tex-archive/macros/latex/contrib/cite/
% The documentation is contained in the cite.sty file itself.






% *** GRAPHICS RELATED PACKAGES ***
%
\ifCLASSINFOpdf
  % \usepackage[pdftex]{graphicx}
  % declare the path(s) where your graphic files are
  % \graphicspath{{../pdf/}{../jpeg/}}
  % and their extensions so you won't have to specify these with
  % every instance of \includegraphics
  % \DeclareGraphicsExtensions{.pdf,.jpeg,.png}
\else
  % or other class option (dvipsone, dvipdf, if not using dvips). graphicx
  % will default to the driver specified in the system graphics.cfg if no
  % driver is specified.
  % \usepackage[dvips]{graphicx}
  % declare the path(s) where your graphic files are
  % \graphicspath{{../eps/}}
  % and their extensions so you won't have to specify these with
  % every instance of \includegraphics
  % \DeclareGraphicsExtensions{.eps}
\fi
% graphicx was written by David Carlisle and Sebastian Rahtz. It is
% required if you want graphics, photos, etc. graphicx.sty is already
% installed on most LaTeX systems. The latest version and documentation
% can be obtained at: 
% http://www.ctan.org/tex-archive/macros/latex/required/graphics/
% Another good source of documentation is "Using Imported Graphics in
% LaTeX2e" by Keith Reckdahl which can be found at:
% http://www.ctan.org/tex-archive/info/epslatex/
%
% latex, and pdflatex in dvi mode, support graphics in encapsulated
% postscript (.eps) format. pdflatex in pdf mode supports graphics
% in .pdf, .jpeg, .png and .mps (metapost) formats. Users should ensure
% that all non-photo figures use a vector format (.eps, .pdf, .mps) and
% not a bitmapped formats (.jpeg, .png). IEEE frowns on bitmapped formats
% which can result in "jaggedy"/blurry rendering of lines and letters as
% well as large increases in file sizes.
%
% You can find documentation about the pdfTeX application at:
% http://www.tug.org/applications/pdftex





% *** MATH PACKAGES ***
%
%\usepackage[cmex10]{amsmath}
% A popular package from the American Mathematical Society that provides
% many useful and powerful commands for dealing with mathematics. If using
% it, be sure to load this package with the cmex10 option to ensure that
% only type 1 fonts will utilized at all point sizes. Without this option,
% it is possible that some math symbols, particularly those within
% footnotes, will be rendered in bitmap form which will result in a
% document that can not be IEEE Xplore compliant!
%
% Also, note that the amsmath package sets \interdisplaylinepenalty to 10000
% thus preventing page breaks from occurring within multiline equations. Use:
%\interdisplaylinepenalty=2500
% after loading amsmath to restore such page breaks as IEEEtran.cls normally
% does. amsmath.sty is already installed on most LaTeX systems. The latest
% version and documentation can be obtained at:
% http://www.ctan.org/tex-archive/macros/latex/required/amslatex/math/





% *** SPECIALIZED LIST PACKAGES ***
%
%\usepackage{algorithmic}
\usepackage{algpseudocode}
% algorithmic.sty was written by Peter Williams and Rogerio Brito.
% This package provides an algorithmic environment fo describing algorithms.
% You can use the algorithmic environment in-text or within a figure
% environment to provide for a floating algorithm. Do NOT use the algorithm
% floating environment provided by algorithm.sty (by the same authors) or
% algorithm2e.sty (by Christophe Fiorio) as IEEE does not use dedicated
% algorithm float types and packages that provide these will not provide
% correct IEEE style captions. The latest version and documentation of
% algorithmic.sty can be obtained at:
% http://www.ctan.org/tex-archive/macros/latex/contrib/algorithms/
% There is also a support site at:
% http://algorithms.berlios.de/index.html
% Also of interest may be the (relatively newer and more customizable)
% algorithmicx.sty package by Szasz Janos:
% http://www.ctan.org/tex-archive/macros/latex/contrib/algorithmicx/




% *** ALIGNMENT PACKAGES ***
%
%\usepackage{array}
% Frank Mittelbach's and David Carlisle's array.sty patches and improves
% the standard LaTeX2e array and tabular environments to provide better
% appearance and additional user controls. As the default LaTeX2e table
% generation code is lacking to the point of almost being broken with
% respect to the quality of the end results, all users are strongly
% advised to use an enhanced (at the very least that provided by array.sty)
% set of table tools. array.sty is already installed on most systems. The
% latest version and documentation can be obtained at:
% http://www.ctan.org/tex-archive/macros/latex/required/tools/


% IEEEtran contains the IEEEeqnarray family of commands that can be used to
% generate multiline equations as well as matrices, tables, etc., of high
% quality.




% *** SUBFIGURE PACKAGES ***
%\ifCLASSOPTIONcompsoc
%  \usepackage[caption=false,font=normalsize,labelfont=sf,textfont=sf]{subfig}
%\else
%  \usepackage[caption=false,font=footnotesize]{subfig}
%\fi
% subfig.sty, written by Steven Douglas Cochran, is the modern replacement
% for subfigure.sty, the latter of which is no longer maintained and is
% incompatible with some LaTeX packages including fixltx2e. However,
% subfig.sty requires and automatically loads Axel Sommerfeldt's caption.sty
% which will override IEEEtran.cls' handling of captions and this will result
% in non-IEEE style figure/table captions. To prevent this problem, be sure
% and invoke subfig.sty's "caption=false" package option (available since
% subfig.sty version 1.3, 2005/06/28) as this is will preserve IEEEtran.cls
% handling of captions.
% Note that the Computer Society format requires a larger sans serif font
% than the serif footnote size font used in traditional IEEE formatting
% and thus the need to invoke different subfig.sty package options depending
% on whether compsoc mode has been enabled.
%
% The latest version and documentation of subfig.sty can be obtained at:
% http://www.ctan.org/tex-archive/macros/latex/contrib/subfig/




% *** FLOAT PACKAGES ***
%
%\usepackage{fixltx2e}
% fixltx2e, the successor to the earlier fix2col.sty, was written by
% Frank Mittelbach and David Carlisle. This package corrects a few problems
% in the LaTeX2e kernel, the most notable of which is that in current
% LaTeX2e releases, the ordering of single and double column floats is not
% guaranteed to be preserved. Thus, an unpatched LaTeX2e can allow a
% single column figure to be placed prior to an earlier double column
% figure. The latest version and documentation can be found at:
% http://www.ctan.org/tex-archive/macros/latex/base/


%\usepackage{stfloats}
% stfloats.sty was written by Sigitas Tolusis. This package gives LaTeX2e
% the ability to do double column floats at the bottom of the page as well
% as the top. (e.g., "\begin{figure*}[!b]" is not normally possible in
% LaTeX2e). It also provides a command:
%\fnbelowfloat
% to enable the placement of footnotes below bottom floats (the standard
% LaTeX2e kernel puts them above bottom floats). This is an invasive package
% which rewrites many portions of the LaTeX2e float routines. It may not work
% with other packages that modify the LaTeX2e float routines. The latest
% version and documentation can be obtained at:
% http://www.ctan.org/tex-archive/macros/latex/contrib/sttools/
% Do not use the stfloats baselinefloat ability as IEEE does not allow
% \baselineskip to stretch. Authors submitting work to the IEEE should note
% that IEEE rarely uses double column equations and that authors should try
% to avoid such use. Do not be tempted to use the cuted.sty or midfloat.sty
% packages (also by Sigitas Tolusis) as IEEE does not format its papers in
% such ways.
% Do not attempt to use stfloats with fixltx2e as they are incompatible.
% Instead, use Morten Hogholm'a dblfloatfix which combines the features
% of both fixltx2e and stfloats:
%
% \usepackage{dblfloatfix}
% The latest version can be found at:
% http://www.ctan.org/tex-archive/macros/latex/contrib/dblfloatfix/




%\ifCLASSOPTIONcaptionsoff
%  \usepackage[nomarkers]{endfloat}
% \let\MYoriglatexcaption\caption
% \renewcommand{\caption}[2][\relax]{\MYoriglatexcaption[#2]{#2}}
%\fi
% endfloat.sty was written by James Darrell McCauley, Jeff Goldberg and 
% Axel Sommerfeldt. This package may be useful when used in conjunction with 
% IEEEtran.cls'  captionsoff option. Some IEEE journals/societies require that
% submissions have lists of figures/tables at the end of the paper and that
% figures/tables without any captions are placed on a page by themselves at
% the end of the document. If needed, the draftcls IEEEtran class option or
% \CLASSINPUTbaselinestretch interface can be used to increase the line
% spacing as well. Be sure and use the nomarkers option of endfloat to
% prevent endfloat from "marking" where the figures would have been placed
% in the text. The two hack lines of code above are a slight modification of
% that suggested by in the endfloat docs (section 8.4.1) to ensure that
% the full captions always appear in the list of figures/tables - even if
% the user used the short optional argument of \caption[]{}.
% IEEE papers do not typically make use of \caption[]'s optional argument,
% so this should not be an issue. A similar trick can be used to disable
% captions of packages such as subfig.sty that lack options to turn off
% the subcaptions:
% For subfig.sty:
% \let\MYorigsubfloat\subfloat
% \renewcommand{\subfloat}[2][\relax]{\MYorigsubfloat[]{#2}}
% However, the above trick will not work if both optional arguments of
% the \subfloat command are used. Furthermore, there needs to be a
% description of each subfigure *somewhere* and endfloat does not add
% subfigure captions to its list of figures. Thus, the best approach is to
% avoid the use of subfigure captions (many IEEE journals avoid them anyway)
% and instead reference/explain all the subfigures within the main caption.
% The latest version of endfloat.sty and its documentation can obtained at:
% http://www.ctan.org/tex-archive/macros/latex/contrib/endfloat/
%
% The IEEEtran \ifCLASSOPTIONcaptionsoff conditional can also be used
% later in the document, say, to conditionally put the References on a 
% page by themselves.




% *** PDF, URL AND HYPERLINK PACKAGES ***
%
%\usepackage{url}
% url.sty was written by Donald Arseneau. It provides better support for
% handling and breaking URLs. url.sty is already installed on most LaTeX
% systems. The latest version and documentation can be obtained at:
% http://www.ctan.org/tex-archive/macros/latex/contrib/url/
% Basically, \url{my_url_here}.




% *** Do not adjust lengths that control margins, column widths, etc. ***
% *** Do not use packages that alter fonts (such as pslatex).         ***
% There should be no need to do such things with IEEEtran.cls V1.6 and later.
% (Unless specifically asked to do so by the journal or conference you plan
% to submit to, of course. )


% correct bad hyphenation here
%\hyphenation{op-tical net-works semi-conduc-tor}

\usepackage{verbatim}
\usepackage{color, colortbl}
\definecolor{Gray}{gray}{0.9}
\usepackage{graphicx}
\usepackage{rotating}
\usepackage{multirow}

\newcommand{\graphicthird}[1]
{\includegraphics[width=.3\textwidth,height=.3\textheight,keepaspectratio]{#1}}

\newcommand{\thirdlabel}[1]
{\multicolumn{1}{|c|}{\raisebox{.15\textwidth}{\rotatebox[origin=c]{90}{\textbf{\em #1}}}}}

\begin{document}
%
% paper title
% can use linebreaks \\ within to get better formatting as desired
% Do not put math or special symbols in the title.
\title{Analysis of Cartesian Genetic Programming's Evolutionary Mechanisms}
%
%
% author names and IEEE memberships
% note positions of commas and nonbreaking spaces ( ~ ) LaTeX will not break
% a structure at a ~ so this keeps an author's name from being broken across
% two lines.
% use \thanks{} to gain access to the first footnote area
% a separate \thanks must be used for each paragraph as LaTeX2e's \thanks
% was not built to handle multiple paragraphs
%

%\author{Michael~Shell,~\IEEEmembership{Member,~IEEE,}
%        John~Doe,~\IEEEmembership{Fellow,~OSA,}
%        and~Jane~Doe,~\IEEEmembership{Life~Fellow,~IEEE}% <-this % stops a space
\author{Brian~W.~Goldman, William~F.~Punch}
%\thanks{M. Shell is with the Department
%of Electrical and Computer Engineering, Georgia Institute of Technology, Atlanta,
%GA, 30332 USA e-mail: (see http://www.michaelshell.org/contact.html).}% <-this % stops a space
%\thanks{J. Doe and J. Doe are with Anonymous University.}% <-this % stops a space
%\thanks{Manuscript received April 19, 2005; revised December 27, 2012.}}

% note the % following the last \IEEEmembership and also \thanks - 
% these prevent an unwanted space from occurring between the last author name
% and the end of the author line. i.e., if you had this:
% 
% \author{....lastname \thanks{...} \thanks{...} }
%                     ^------------^------------^----Do not want these spaces!
%
% a space would be appended to the last name and could cause every name on that
% line to be shifted left slightly. This is one of those "LaTeX things". For
% instance, "\textbf{A} \textbf{B}" will typeset as "A B" not "AB". To get
% "AB" then you have to do: "\textbf{A}\textbf{B}"
% \thanks is no different in this regard, so shield the last } of each \thanks
% that ends a line with a % and do not let a space in before the next \thanks.
% Spaces after \IEEEmembership other than the last one are OK (and needed) as
% you are supposed to have spaces between the names. For what it is worth,
% this is a minor point as most people would not even notice if the said evil
% space somehow managed to creep in.



% The paper headers
\markboth{IEEE Transactions on Evolutionary Computation,~Vol.~\#\#, No.~\#, Month~Year}%
{IEEE Transactions on Evolutionary Computation,~Vol.~\#\#, No.~\#, Month~Year}
% The only time the second header will appear is for the odd numbered pages
% after the title page when using the twoside option.
% 
% *** Note that you probably will NOT want to include the author's ***
% *** name in the headers of peer review papers.                   ***
% You can use \ifCLASSOPTIONpeerreview for conditional compilation here if
% you desire.




% If you want to put a publisher's ID mark on the page you can do it like
% this:
%\IEEEpubid{0000--0000/00\$00.00~\copyright~2012 IEEE}
% Remember, if you use this you must call \IEEEpubidadjcol in the second
% column for its text to clear the IEEEpubid mark.



% use for special paper notices
%\IEEEspecialpapernotice{(Invited Paper)}




% make the title area
\maketitle

% As a general rule, do not put math, special symbols or citations
% in the abstract or keywords.
\begin{abstract}
TODO The abstract goes here.
\end{abstract}

% Note that keywords are not normally used for peerreview papers.
\begin{IEEEkeywords}
Cartesian genetic programming (CGP), Analysis
\end{IEEEkeywords}






% For peer review papers, you can put extra information on the cover
% page as needed:
% \ifCLASSOPTIONpeerreview
% \begin{center} \bfseries EDICS Category: 3-BBND \end{center}
% \fi
%
% For peerreview papers, this IEEEtran command inserts a page break and
% creates the second title. It will be ignored for other modes.
\IEEEpeerreviewmaketitle



\section{Introduction}
% The very first letter is a 2 line initial drop letter followed
% by the rest of the first word in caps.
% 
% form to use if the first word consists of a single letter:
% \IEEEPARstart{A}{demo} file is ....
% 
% form to use if you need the single drop letter followed by
% normal text (unknown if ever used by IEEE):
% \IEEEPARstart{A}{}demo file is ....
% 
% Some journals put the first two words in caps:
% \IEEEPARstart{T}{his demo} file is ....
% 
% Here we have the typical use of a "T" for an initial drop letter
% and "HIS" in caps to complete the first word.
\IEEEPARstart{W}{hile effective} evolutionary methods are often simple to design
and implement, it can sometimes be difficult to discern the reason a technique
or behavior in that system leads to effective search.  This can be exceptionally
true for Genetic Programming (GP) systems, as they often include nontrivial
genotype to phenotype maps and operators that act on genotypes without regard to
phenotypic impact.  Yet understanding the root causes of evolutionary success
and failure is critical to improving existing techniques, designing new techniques,
and understanding how and where to apply each optimization system.

In this work we set out to develop a deeper understanding of what makes Cartesian
Genetic Programming (CGP) an effective evolutionary optimizer.  Our primary focus is
on how mutation and genome ordering interact.  In Section~\ref{sec:cgp} we provide
an introduction to CGP, with sufficient background to put later sections into
context.  Section~\ref{sec:duplicate} explains previous work about mutation's potential
to create identifiably wasted evaluations, and proposed methods to prevent that
waste.  Section~\ref{sec:ordering} discusses the impact of CGP's genome ordering,
with methods to both examine and bypass the limitations imposed by that ordering.
Each technique is then rigorously tuned and qualitatively compared on a collection
of benchmarks in Section~\ref{sec:quality}.  Section~\ref{sec:analysis} then
provides a deep look into the details of CGP's search, and how each method impacts
search, with final remarks given in Section~\ref{sec:conclusion}.

There have been a number of previous studies into various aspects of GP evolution.
For instance~\cite{daida3:2003:treebias} looked into how trees of specific shapes,
which helped inspire the use of new grammar based operations~\cite{xuan:2006:grammar}.
Analysis of the root cause of bloat has helped to inform methods for controlling
bloat~\cite{luke:2006:bloat}.  A few studies of evolutionary mechanisms have also
been done in CGP directly, related to bloat~\cite{miller:2001:bloat},
neutrality~\cite{vassilev:2000:neutrality}, and structural bias~\cite{payne:2009:bias}.


\section{Cartesian Genetic Programming}
\label{sec:cgp}
Cartesian Genetic Programming (CGP) was originally proposed as a method for general
genetic programming in~\cite{miller:2000:CGPorigin}.  While its beginnings and
its name come from evolving circuits on a two dimensional grid, modern CGP can
represent any directed acyclic graph (DAG), with applications in evolving binary
circuits~\cite{walker:2008:cgpmodules},
robot controllers~\cite{harding:2005:robots},
neural networks~\cite{khan:2010:cgpann},
image classifiers~\cite{harding:2012:mtcgp},
and regression~\cite{harding:2009:smcgp}.

CGP represents DAGs using a linear genome of integer values.  Each node in the
DAG is encoded using a collection of genes, with one gene specifying the function
the node applies to its inputs, and the remaining genes expressing where the
node takes input from.  Nodes can take input from any problem input and from
any node preceding them in the genome.  To complete the representation, a set
of extra genes are included at the end of the genome to specify which nodes or
input locations to use as function outputs.

Unlike tree based GP encodings, CGP is able to perform value reuse.  This means
that the output of a node can be reused by any number of other nodes as input values.
CGP's requirement that nodes connect forward in the genome prevents this ability
from causing cycles.

As output locations and information flow in the DAG are evolvable, there are often
large portions of the genome which do not participate in creating any of the
output values.  These nodes are referred to as inactive, with the nodes currently
being used to create output values referred to as active.  Inactive nodes
allow CGP individuals to drift genetically, as they can be mutated without effecting
the fitness of the individual.  This genetic drift can then be incorporated, as
mutation can change the DAG structure causing previously inactive nodes to become
active.  Previous work suggests CGP is most efficient when up to 95\% of the genome
is inactive~\cite{miller:2006:redundancy}, yet other work suggests this may
be a result of hidden parsimony pressure in CGP~\cite{goldman:2013:ordering}.

CGP uses very simple evolutionary mechanisms.  By far the most common strategy
is $1+4$, in which each generation a single parent produces four offspring using
mutation.  The best offspring then competes with the parent, with the offspring
replacing the parent if it is no less fit.  This replacement strategy encourages
drift, as completely neutral mutations are always preferred to stagnation.  Mutation
is commonly performed by mutating each gene at a set probability, where mutation
involves changing a gene randomly to some different valid value.  For instance,
if a function gene is chosen for mutation, its new value is randomly chosen from
all possible functions, with the exception of the genes current value.  Combined,
this form of CGP can be viewed as a stochastic hill climber with neutrality.
For a more in depth description of CGP, see~\cite{miller:2011:chapter2}.

\section{Duplicate Evaluation Avoidance}
\label{sec:duplicate}
CGP's mutation operator is traditionally applied to all genes equally, with no
consideration if genes are part of an active or inactive node.  This has the
potential to lead offspring who's only mutated genes are part of inactive nodes.
These offspring are actively identical (contain identical active genes) to their
parents, and therefore by definition have identical fitness to their parents.
As such, evaluating these individuals is wasteful, as their fitness can be
assigned without the computational complexity of performing an evaluation.
Furthermore, calculating which nodes are active before evaluating an individual
improves evaluation speed~\cite{vasicek:2012:efficient}, meaning detecting when duplicated individuals are
created can be done without significant computational overhead.
Previous work~\cite{goldman:2013:cgpwaste} has taken an initial look at the
effect this waste can have, and proposed \emph{Skip}, \emph{Accumulate}, and
\emph{Single} as methods for avoiding or preventing these duplicate evaluations.

The amount of evaluations wasted on these actively identical individuals is
directly related to the mutation rate and the number of active genes.  With
sufficiently high mutation rates and enough active genes, the probability of
creating an offspring which is actively identical to its parent is very low.
Yet such configuration make make it difficult for CGP to optimize, as mutation
is likely to change many active genes in each application.  As was discussed
in~\cite{goldman:2013:cgpwaste}, this can make normal CGP very sensitive to the
mutation rate in use.

\subsection{Skip}
The \emph{Skip} method for avoiding duplicate evaluations involves the least
amount of modification from CGP's normal behavior.  After an offspring is produced
and its set of active nodes is determined but before it is evaluated, it is compared
with its parent.  Each gene in each active node is compared for equivalence with
the corresponding gene in the offspring's parent.  If all genes are found to be equal
than the individual is not evaluated.  Instead the offspring is given the same fitness
as its parent, as there is no way for it to be phenotypically different from its parent.
Note that since inactive genes are not compared for equivalence it is still possible
for offspring to genetically differ from their parents.

Because \emph{Skip} does not modify any evolutionary mechanisms from normal CGP,
in all possible cases it can do no worse.  If the mutation rate is high enough relative
to the number of active genes, \emph{Skip} will act identically to normal CGP as
the probability of an offspring being actively identical to its parent is effectively zero.
When the mutation rate is low enough relative to the number of active genes, some number of
offspring will be actively identical to their parents, resulting in a reduction in evaluations
but no change in the evolutionary trajectory.

From a tuning perspective we would expect both intuitively and from initial
experimentation~\cite{goldman:2013:cgpwaste} that \emph{Skip} will be less sensitive
to the mutation rate and will perform best using a lower mutation rate than normal CGP.
The former comes from the fact that in normal CGP any mutation rate that has a significant
probability of creating offspring actively identical to their parents will be penalized
by wasting evaluations.  In \emph{Skip} there are only two penalties for reducing the
mutation rate.  First, if search becomes trapped in a local optima, low mutation
rates may have increased difficulty escaping to find the global optimum.  Second,
exceptionally low mutation rates may result in large computational overhead as
more and more time is spent attempting to produce an evaluable useful offspring.
Note these penalties also exist in normal CGP.

\subsection{Accumulate}
Similar to \emph{Skip}, \emph{Accumulate} works by adding a step between offspring
creation and offspring evaluation.  Instead of skipping evaluations with offspring
are determined to be actively identical to their parents, \emph{Accumulate} enters
into a cycle of repeated mutation until an individual worth evaluating is created.
While the working offspring is actively identical to the
parent, a new individual is produced by applying mutation to the working offspring.
Note that each mutation step can probabilistically mutate any number of genes, just
as in the mutation used by normal CGP.
In this way each iteration the working offspring accumulates mutations to inactive
genes until an individual is produced that differs by one or more active genes from
the parent.  This repeatedly mutated offspring is then evaluated.  If its fitness
is less than the original parent, the final set of mutations are reverted
and the offspring's fitness is set to the same as the original parent.

Viewed in another way, \emph{Accumulate} performs a micro evolution on each offspring
before it is fully produced.  The final result of each offspring step is to produce
an offspring which is either actively identical to its parent, but with mutations
to its inactive genes, or an offspring which is at least as fit as the parent but
not actively identical.

Previous experimentation has shown that \emph{Accumulate} acts very similarly to
\emph{Skip}~\cite{goldman:2013:cgpwaste}, with the exception that \emph{Accumulate}
favors lower mutation rates. While at first these algorithms may appear
quite different, further consideration shows how they are similar.  In \emph{Skip}
an actively identical offspring is likely to be selected as we expect mutations to
active genes to more often reduce fitness than in prove it.  In the next generation
this actively identical offspring is then mutated again, with its lineage likely
continuing until it finally does mutate an active gene.  At this point if the
mutant is better it replaces the parent, otherwise the parent, which has been accumulating
mutations to inactive genes, is kept.  In this way \emph{Skip} mirrors \emph{Accumulate},
except over multiple generations instead of compressed into a single generation.

\subsection{Single}
While \emph{Skip} and \emph{Accumulate} focus on what to do when offspring are
produced which are actively identical to their parents, \emph{Skip} changes
how mutation works to ensure only evaluable offspring are created.  Instead
of mutating each gene at a set probability, \emph{Single} chooses genes at
random to mutate until exactly one active gene is mutated.

This modification gives \emph{Single} three properties distinct from the other
forms of duplicate avoidance.  First, \emph{Single} forces offspring to have
an active gene which is different than there parent.  This limits drift as the
mutated active gene must either be to an intron or represent another way to
code a solution of equal quality.  Second, as a benefit of forced changes
\emph{Single} avoids the overhead of repeatedly generating individuals without
creating an evaluable offspring.  Third, \emph{Single} does not require a mutation
rate parameter, effectively setting the mutation rate to $\frac{1}{a}$ for active
genes and $\frac{1}{a+1}$ for inactive genes, where $a$ is the number of active genes.

In encodings without inactive genes, limiting mutation to changing exactly one gene
could prevent an algorithm from escaping some types of local maxima.  Yet because
CGP allows for inactive genes, \emph{Single} is still able to escape most local
maxima using sufficient drift of inactive genes.  When a high percentage of
the genome becomes active or when \emph{Single} is otherwise limited in its ability
to drift due to lack of introns, it does have an increased potential to become stuck.
As will be discussed in Section~\ref{sec:ordering} and was shown in~\cite{goldman:2013:ordering}
this problem of highly active genomes is very unlikely.

\section{Genome Ordering}
\label{sec:ordering}
As was discussed in Section~\ref{sec:cgp}, CGP uses node ordering in the genome
to prevent cycles, as nodes are only able to receive input from sources that
precede them in the genome.  This restriction does not limit CGP's ability
to represent DAGs, as all DAGs can be serialized to fit this requirement.
Yet this likely has an impact on CGP's ability to evolve specific
DAGs~\cite{goldman:2013:ordering}.

Primarily, enforcing node ordering adds artificial limitations CGP's ability to
connect nodes.  Beyond preventing cycles, it also prevents nodes from connecting
which would not create a cycle based solely on the evolutionary happenstance of
where those nodes were evolved.  Similarly, it is impossible for adjacent nodes
to ever be connected through an intermediate node.  As a result, useful structures
in the genome may need to be evolved repeatedly in order to find the best
location for that structure in the genome.

Ordering combined with the random reset mutation method may be the cause of CGP's
immensely inactive genomes.  As was rigorously defined in~\cite{goldman:2013:ordering},
the number of active genes in a genome is expected to scale logarithmically
with genome size, independent of problem application.  Furthermore, the less
active a genome, the lower the likelihood of structure ordering preventing structure
connection.

To rectify these issues an examine CGP without ordering bias, \cite{goldman:2013:ordering}
created \emph{Reorder} and \emph{DAG} as methods for genome ordering, and defined
\emph{Normal} as CGP's historical ordering method.

\subsection{Reorder}
\label{sec:reorder}
The concept behind \emph{Reorder} is to randomly shuffled node ordering without
impacting gene expression.  This is possible because for a given node, there can
be a large number of other nodes which neither have no required ordering,
as they neither directly or indirectly take input from the node nor provide their
output to the node.  In general, this process works by assigning nodes new
locations in the genome at random once all of the nodes they take input from have been assigned
locations earlier in the genome.

The first step in performing \emph{Reorder} on a genome is to create convenience
data structures to store direct connection relationships, such that given a node
we can determine in constant time where it gets its input and which nodes read
its output.  These structures can be build in order $O(N)$ time, where $N$ is the
number of nodes.

With this preprocessing complete, the algorithm given in Figure~\ref{fig:reorder}
is used to assign each node a new location in the modified genome.
It starts by constructing the \emph{addable} set, which contains all nodes who's direct
dependencies have already been added.  Initially this contains only the input locations.
Once a node is assigned a location, all of the nodes that depend on its output
have that dependency marked as satisfied.  Once all of a nodes dependencies
are satisfied, it can be put into \emph{addable}.  In this way, nodes are
randomly removed from \emph{addable}, assigned the next possible location
in the genome, with new nodes added to \emph{addable} as it becomes semantically
viable.  Iteration ends when all nodes have been assigned new locations.
This algorithm requires $O(AN)$, where $A$ is the arity used by the nodes
and $N$ is the number of nodes.

\begin{figure}
  \begin{algorithmic}
    \Procedure{RandomSerialization}{}
    \State $addable \leftarrow input\_locations$
    \State $index \leftarrow |input\_locations|$
    \State $new\_loc \leftarrow \emptyset$
    \While{$|addable| > 0$}
      \State $working \leftarrow randpop(addable)$
      \If{$working \notin input\_locations$}
        \State $new\_loc[working] \leftarrow index$
        \State $index \leftarrow index + 1$
      \EndIf
      \ForAll{$link \in feeds\_to(working)$}
        \State Satisfy $link$'s dependence on $working$
        \If{$|unsatisfied(link)| = 0$}
          \State $addable \leftarrow addable \cup \{link\}$
        \EndIf
      \EndFor
    \EndWhile
    \State\Return $new\_loc$
    \EndProcedure
  \end{algorithmic}
  \caption{Algorithm which converts any DAG into a random serial ordering such
           that all nodes take input from nodes that precede them in the genome.
           Used by \emph{Reorder} to determine new node locations and \emph{DAG} to
           determine node evaluation ordering.}
  \label{fig:reorder}
\end{figure}

The final step in \emph{Reorder} is to use the list of new locations to convert
the existing genome into the reordered genome.  This can be done in the straight
forward manner by converting connection genes and output location genes using
the $new\_loc$ map returned from the algorithm in Figure~\ref{fig:reorder}.
This also requires $O(AN)$ time, as all connection genes in all nodes must
be converted.  As this is the final step unique to \emph{Reorder}, and
no previous steps were of higher complexity, it is also the complexity of
the algorithm as a whole.  Furthermore it is of the same complexity class
of copying the entire genome, meaning it does not change CGP's overall complexity.

\emph{Reorder} is used once each generation to shuffle the nodes of the parent
organism.  As the shuffling does not semantically change the parent, it does
not need to be reevaluated.  Furthermore, as shuffling does not change the requirement
that nodes only depend on those preceding them in the genome, shuffling does not
require any changes to any other CGP methods, such as evaluation and mutation.
Yet after shuffling, the potential mutations that can be applied to the parent
have changed.  Consider two nodes $X$ and $Y$ which are in the genome, but
have no dependence relationship.  As $X$ does not take input from $Y$ and
$Y$ does not take input from $X$, either can be placed preceding the other
in the genome.  Without shuffling, the ordering in which these two nodes originally
evolved will be preserved indefinitely, such that if $X$ precedes $Y$ at node creation,
$X$ will always precede $Y$.  No mutation in \emph{Normal} is capable of making $X$
take input from $Y$.  Yet CGP using \emph{Reorder} is capable of shuffling $Y$ to
precede $X$, allowing subsequent mutations to make the connection.

This node movement has the potential to reduce node reinvention.  If evolution created
$X$ and $Y$, and mutating $X$ to depend on $Y$ would improve fitness, mutation's
ability to make that improvement would depend on happenstance.  If $X$ evolved
preceding $Y$, mutation would have to recreate either $X$ or $Y$ in new locations
in order to make the required connection.  \emph{Reorder} could achieve the same
effect through shuffling, and without costly node reinvention.

The price \emph{Reorder} pays for this potential improvement is increased mutational search
space and minor runtime costs.  Even though \emph{Reorder} is not capable of representing
different individuals that \emph{Normal}, their mutational adjacency is significantly different.
By mutational adjacency, we mean the set of individual's which are most likely to be created
when mutating a given individual.  For instance in \emph{Normal} if $X$ precedes $Y$, individuals
containing $Y$ dependent on $X$ could be adjacent to the original individual, but those containing
$X$ dependent on $Y$ would not.  As \emph{Reorder} can shuffle node order, both sets would be
adjacent.  In this way ordering limits how \emph{Normal} performs its search.  This limitation
will benefit \emph{Normal} any time making $X$ dependent on $Y$ is detrimental, as
\emph{Normal} won't waste evaluations on individuals having that mutation.


\subsection{DAG}
\label{sec:dag}
The principle behind \emph{DAG} is to remove the traditional CGP requirement that
nodes can only take input from nodes that precede them in the genome.  Instead
this is replaced with the relaxed requirement than the genome cannot encode for
any cycles.  To accomplish this goal, modifications must be made to how CGP
performs mutation and evaluation.

When a connection gene is chosen for mutation, its value is still chosen randomly
from all possible alternative values.  To achieve this, we incrementally determine
which nodes transitively depend on the mutating node and which are not, with
information stored in the $known$ map.  Initially, we know that the node
is dependent on itself and that input locations and the nodes current inputs
are not dependent on the mutating node.  Nodes are then
tested in a random order for dependence using the algorithm given in Figure~\ref{fig:dag}.
The first node returned that is not dependent on the mutating node is then
used as the new connection gene value.

\begin{figure}
  \begin{algorithmic}
  \Procedure{IsDependent}{$working$, $known$}
    \If{$working \in known$}
      \State\Return{$known[working]$}
    \EndIf
    \ForAll{$link \in reads\_from(working)$}
      \If{\Call{IsDependent}{$link$, $known$}}
        \State $known[working] \leftarrow True$
        \State\Return{$True$}
      \EndIf
    \EndFor
    \State $known[working] \leftarrow False$
    \State\Return{$False$}
  \EndProcedure
  \end{algorithmic}
  \caption{Algorithm used by \emph{DAG} to determine valid connection gene values.}
  \label{fig:dag}
\end{figure}

The algorithm given in Figure~\ref{fig:dag} is a non-repeated recursive depth
first search of the individual's DAG.  Search terminates as soon as a dependent
node is found, and information from previous searches is recorded to prevent
repeated search.  Each recursion level results in a node having its dependency
set, meaning early calls will likely result in lots of recursion, but subsequent
calls will find answers faster.  In the worst case, this algorithm may need
to examine all nodes in the genome to determine if $working$ is dependent on the
mutating node, giving it a complexity of $O(N)$.  In the worst case, this algorithm
may be called on all possible nodes in the genome to check their dependence.
Yet cumulatively those calls can only take $O(N)$ time due to the non-repeating
nature of the search.  As a result, the expected worst case run time for \emph{DAG}
mutation is $O(mAN^2)$, as when $m$ is the mutation rate we expect $mAN$
connection genes to be mutated, each requiring a worst case of $N$ dependency checks.
This is in contrast to \emph{Normal} and \emph{Reorder} which require $O(mAN)$ to perform
mutation, as individual mutations can be performed in constant time.

The method for evaluating \emph{DAG} individuals is very similar to the method for
efficiently evaluating \emph{Normal} and \emph{Reorder} individuals.  Just as before,
preprocessing is done to determine which nodes are active, and only those nodes are
executed during evaluation.  Just as before the process begins from the output locations,
recursively following connection genes and marking nodes as active until the input locations
are reached.  Here the process diverges, as in \emph{DAG} we must determine not only the set of
nodes that are active, but the order in which those nodes should be executed.

The algorithm given in Figure~\ref{fig:reorder} can be reused to determine the order
in which nodes should be executed during evaluation.  Instead of using $new\_loc$
to reorder the genome, we can use this map to specify what order nodes should be
executed in.  For instance, if a node was given the new location of $X$, we know
that once all nodes at locations preceding $X$ have been evaluated, the node at $X$
can be evaluated.  Stripping the inactive nodes from $new\_loc$ and inverting the map
results in an efficient, valid ordering in which to execute the nodes.

The process of determining which nodes are active requires $O(AN)$ time, regardless
of how the node ordering is handled.  As was discussed in Section~\ref{sec:reorder},
getting the $new\_loc$ map requires $O(AN)$ time.  Finally, converting the map
into an ordering of the active nodes takes $O(N)$ time.  Each of these steps are
sequential, meaning \emph{DAG} does not increase the runtime complexity of evaluation
preprocessing.  Furthermore, once the preprocessing step is done, \emph{DAG} takes
identical time to evaluate than the other methods, regardless of the number of
input combinations the individual is evaluated on.

With these modifications, there is no longer any meaning to the position of a node
in the genome.  In \emph{Normal} a node near the input locations will always
have a highly restricted set of values.  More completely, the number of valid
values for a connection gene in \emph{Normal} is related to the number of nodes
preceding that gene, and independent of the number of nodes following that gene.
In \emph{DAG} a connection gene's valid values only depends on the current solution
represented by the individual, independent of the genes location in the genome.
As such, there are far less artificial limitations on how \emph{DAG} can modify
solutions.

While both reduce mutational limitations, \emph{DAG}'s changes are more broad that
\emph{Reorder}.  While both have the possibility to use mutation to connect any two
nodes such that no cycle is created, \emph{Reorder} still maintains a bias about
how often some types of connection can happen.  Consider again two nodes $X$ and $Y$,
such that neither is dependent on the other.  If $X$ is transitively dependent on very
few nodes, and $Y$ is transitively dependent on many nodes, in general \emph{Reorder}
will shuffle the genome such that $X$ is before $Y$.  This is true because the probability
that all of $X$'s dependencies are added at any point of building the new order is always
higher than for $Y$'s dependencies.  As a result we expect $X$ to be put into $addable$ sooner,
and as a result we expect it to be added sooner.  The greater the discrepancy in these
set sizes the less likely the order between the nodes will be reversed.
Conversely, \emph{DAG} has no such bias.  It is always possible to mutate $X$ to connect to $Y$.

The probability of connecting $X$ to $Y$ is also influenced by how many nodes transitively depend on each.
In this example, let us assume $X$ has more nodes transitively dependent on it than $Y$ has.
In \emph{Reorder}, $X$ would again appear closer to the start of the genome, this time because the number
of possible nodes preceding it is reduced.  \emph{DAG} is biased in the reverse, such that $X$ will
have a higher chance of connecting to $Y$ that $Y$ to $X$.  This is because $X$ can connect to fewer
nodes, so each mutation has a higher chance of choosing $Y$.

\section{Qualitative Comparisons}
\label{sec:quality}
While it is possible to theorize the effects of each CGP variant on search, and
how those effects may interact with each other, without empirical evidence it is
mostly speculation.  Therefore we propose to start by testing the effectiveness
of the variants and their combinations in problem solving.  Deeper analysis of
the causes of their effectiveness are given in Section~\ref{sec:analysis}.

All told we have proposed three new methods for avoiding duplicate evaluations
(\emph{Skip}, \emph{Accumulate}, and \emph{Single}) and two new methods for
how to deal with genome ordering (\emph{Reorder} and \emph{DAG}).  As these two
sets are non overlapping, and there is no intuitive groupings, we chose to try
all possible combinations.  As a control we included \emph{Normal} ordering,
as it provides insight to how each of the duplicate prevention techniques work
if applied alone.  Similar measures are not needed in testing the ordering techniques,
as \emph{Skip} has no impact on evolutionary search, and is a strict improvement
to using no duplication detection at all.

In total this creates 9 algorithmic configurations.  Of these previous work
has been done testing the duplication methods with \emph{Normal} ordering~\cite{goldman:2013:cgpwaste}
as well as analysis of all three ordering methods with \emph{Single}~\cite{goldman:2013:ordering}.
This means there are 4 novel combinations, beyond this being the first time a cross
comparison has been performed.  This is also the first time the ordering methods
have been rigorously tested, as much of the work in~\cite{goldman:2013:ordering}
was analysis oriented.

\subsection{Problem Set}
In order to provide test landscapes for evolution, we chose problems common to
CGP literature and previous work with the duplication prevention and ordering
techniques.  We have chosen 4 binary circuit problems, as the binary representation
allows for some of the precise analysis performed in Section~\ref{sec:analysis}.
For all problems, we used the function set \{and, or, nand, nor\}.
To cover a range of different binary problems, each of the four chosen have
different numbers of inputs, different number of outputs, and different levels
of difficulty.

The first problem, 3-bit parity, represents probably the most
common historic application for
CGP~\cite{yu:2001:neutrality,miller:2006:redundancy,walker:2008:cgpmodules}.
We include this problem purely for backward comparison, as we agree that in general
it is too simple of a problem to merit its own conclusions~\cite{white:2013:bgpb}.
Yet as part of the group it may help yield understanding about how each variant
performs on simple problems.

As representatives of binary problems with varying input and output sizes, we
reuse the 4-16-bit Decode and 16-4-bit Encode proposed in~\cite{goldman:2013:cgpwaste}.
In these problems, the circuit evolved must either convert a 4-bit encoded integer
to a 1 on the corresponding output line (Decode), or take a 1 on one of the 16
input lines and convert it back into a 4-bit encoded integer (Encode).  These
problems share properties with the commonly used Multiplexer problem, but include
multiple output lines, and can be evaluated quicker as each only requires 16 possible
test points.

Finally we chose the 3-bit Multiply to represent hard binary problems,
as was suggested in~\cite{white:2013:bgpb} and used
by~\cite{vassilev:2000:neutrality,miller:2006:redundancy,walker:2008:cgpmodules}.
This problem is very difficult by comparison to the other problems, and has
the largest number of test points.



%TODO Consider adding 5-bit Parity or Equality, 11-bit Multiplexer.

\subsection{Parameter Setting}
\label{sec:parameter}
In order to fairly compare each method, we should ensure proper parameter configuration.
This is to avoid the potential criticism that the parameter settings chosen benefit
a specific method more than others.  While tuning can lead to the alternative issue
of methods only being effective after extensive problem specific tuning, we set out to use
rough grained values and provide each technique with equal tuning time to alleviate that bias.

We focus on two parameter values in our configuration: mutation rate and genome size.
While there are other potential configurable parameters (population size, offspring size, etc),
we feel the CGP literature has converged on settings for those parameters (1, 4).
Furthermore, as duplication detection explicitly deals with the relationship between
mutation rate and number of active genes, and the number of active genes depends
on the genome size and ordering method, mutation rate and genome size seem to be
the most likely to impact results.

To set parameter values, we started by defining a grid of parameter values:
50, 100, 200, 500, 1000, 2000, 5000, 10000 for genome size and 0.05, 0.02, 0.01,
0.005, 0.002, 0.001, 0.0005, 0.0002 for mutation rate.  Note that here a genome
size of 50 means there are 50 nodes, each composed of multiple genes, plus the
required genes to specify output locations.  Also note that mutation rate of 0.05
is used here to mean each gene has a 5\% probability of being mutated in a single
application of mutation.  These values were chosen as they completely cover the
range of previously used parameter settings, with each value having at least
one paired value for the other parameter which is feasibly effective.
The grid is formed by trying all possible mutation rates
with all possible genome sizes, resulting in 64 potential configurations for each
of the algorithmic configurations.  Finally note that as \emph{Single} does not
use a mutation rate, only the eight configurations with different genome sizes were
used when \emph{Single} was employed.

To choose which parameter configuration to use with each algorithmic configuration
for each problem, we used an iterative process of performing runs, comparing means, removing configurations,
and repeating.  For each algorithm configuration and problem, we first run all
potential parameter configurations 5 times (320 runs if not using \emph{Single}, 40 otherwise).
The parameter configurations are then sorted based on their median.  The best half are then
each run 4 more times, with the best half of that group chosen.  In this way, 32 configurations
are run 5 times, 16 are run 9 times, 8 are run 13 times, 4 are run 17 times, 2 are run 21 times,
and 2 are run 25 times.  For configuration with single, all 8 parameter configurations are run 17 times
before any are removed.  This iterative reduction allows us to focus tuning on
those parameter configurations most likely to be effective, with more information
gathered before choosing between high quality configurations.  In total this means
3924 complete runs are required to set the parameter configuration for all
algorithm configurations for a single problem.

To ensure termination, all runs were limited to 10,000,000 evaluations, well beyond
the expected time to completion.  As the median was used to compare configurations,
and not the mean, selection should be insensitive to this limit unless a large
portion of runs are failing to optimize.

To prevent undue biasing in the final results, none of these tuning runs were
used for any other data analysis.  This helps reduce selection's effect on
result distribution, similar to how it is common to split training and testing
runs.  In this case, the runs used to set the parameter configuration are considered
the training runs for that configuration.  As a result, our final results were gather
from a completely independent set of runs to test the configurations general characteristics.
Each configuration was run 50 times to ensure statistical power.

\subsection{Results}

\begin{table*}
	\centering
	\begin{tabular}{|c|c|c|c|c|c|c|c|c|}
	  \hline
	  \multicolumn{9}{|c|}{\textbf{3-bit Parity}, Kruskal-Wallis=$5.99\cdot 10^{-14}$} \\ \hline
\textbf{Duplication} & \textbf{Ordering} & \textbf{Genome Size} & \textbf{Mutation Rate} & \textbf{MES} & \textbf{MAD} & \textbf{Active} & \textbf{Used} & \textbf{p-value} \\ \hline
\emph{Accumulate} & \emph{Normal} & 10,000 & 0.02 & 682 & 348 & 122 & 20 & 0.5602 \\ \hline
\emph{Accumulate} & \emph{Reorder}&  5,000 & 0.01 & 796 & 434 & 331 & 23 & 0.7174 \\ \hline
\rowcolor{Gray}
\emph{Skip} & \emph{Normal}       & 10,000 & 0.02 & 823 & 363 & 119 & 20 & NA \\ \hline
\emph{Skip} & \emph{Reorder}      &  2,000 & 0.01 & 870 & 362 & 198 & 20 & 0.3125 \\ \hline
\emph{Accumulate} & \emph{Dag}    &  5,000 & 0.01 & 964 & 499 & 1,174 & 22 & 0.3310 \\ \hline
\emph{Skip} & \emph{Dag}          & 10,000 & 0.02 & 1,042 & 620 & 2,083 & 24 & 0.0248 \\ \hline
\emph{Single} & \emph{Normal}     & 500 & NA & 1,214 & 497 & 39 & 17 & 0.0046 \\ \hline
\emph{Single} & \emph{Reorder}    & 200 & NA & 1,506 & 823 & 57 & 18 & 0 \\ \hline
\emph{Single} & \emph{Dag}        & 100 & NA & 2,569 & 1,313 & 43 & 15 & 0 \\ \hline\hline

	  \multicolumn{9}{|c|}{\textbf{16-4-bit Encode}, Kruskal-Wallis=$1.19\cdot 10^{-6}$} \\ \hline

\textbf{Duplication} & \textbf{Ordering} & \textbf{Genome Size} & \textbf{Mutation Rate} & \textbf{MES} & \textbf{MAD} & \textbf{Active} & \textbf{Used} & \textbf{p-value} \\ \hline
\rowcolor{Gray}
\emph{Skip} & \emph{Normal}        & 10,000 & 0.005 & 16,226 & 7,455  &   283 & 49 & NA \\ \hline
\emph{Skip} & \emph{Reorder}       & 10,000 & 0.005 & 17,883 & 8,378  & 2,070 & 62 & 0.6918 \\ \hline
\emph{Accumulate} & \emph{Reorder} & 10,000 & 0.002 & 18,132 & 8,926  & 2,027 & 64 & 0.8822 \\ \hline
\emph{Accumulate} & \emph{Normal}  & 10,000 & 0.005 & 21,428 & 6,379  &   279 & 49 & 0.2525 \\ \hline
\emph{Accumulate} & \emph{Dag}     & 10,000 & 0.002 & 24,855 & 12,512 & 4,059 & 69 & 0.0736 \\ \hline
\emph{Skip} & \emph{Dag}           & 10,000 & 0.002 & 25,678 & 12,034 & 4,178 & 62 & 0.0211 \\ \hline
\emph{Single} & \emph{Reorder}     & 100    & NA    & 26,558 & 7,858  &    52 & 32 & 0.0076 \\ \hline
\emph{Single} & \emph{Normal}      & 2,000  & NA    & 31,057 & 14,475 &   141 & 43 & 0.0085 \\ \hline
\emph{Single} & \emph{Dag}         & 100    & NA    & 34,208 & 10,069 &    59 & 34 & 0 \\ \hline\hline

	  \multicolumn{9}{|c|}{\textbf{4-16-bit Decode}, Kruskal-Wallis=$1.02\cdot 10^{-36}$} \\ \hline


\textbf{Duplication} & \textbf{Ordering} & \textbf{Genome Size} & \textbf{Mutation Rate} & \textbf{MES} & \textbf{MAD} & \textbf{Active} & \textbf{Used} & \textbf{p-value} \\ \hline
\emph{Skip} & \emph{Reorder}       &  2,000 & 0.002 &  62,853 & 20,023 &   757 & 114 & 0.5247 \\ \hline
\emph{Single} & \emph{Reorder}     &    500 &    NA &  63,731 & 13,119 &   245 &  97 & 0.3610 \\ \hline
\emph{Accumulate} & \emph{Normal}  &  1,000 & 0.002 &  67,390 & 11,991 &   211 &  87 & 0.8767 \\ \hline
\rowcolor{Gray}
\emph{Skip} & \emph{Normal}        &  1,000 & 0.002 &  68,231 & 15,115 &   202 &  85 & NA \\ \hline
\emph{Single} & \emph{Normal}      &  1,000 &    NA &  68,819 & 14,590 &   203 &  85 & 0.7695 \\ \hline
\emph{Accumulate} & \emph{Reorder} &  2,000 & 0.002 &  72,991 & 18,650 &   760 & 115 & 0.1868 \\ \hline
\emph{Skip} & \emph{Dag}           & 10,000 & 0.001 & 132,821 & 29,432 & 4,983 & 125 & 0 \\ \hline
\emph{Accumulate} & \emph{Dag}     &  5,000 & 0.002 & 133,188 & 29,326 & 2,542 & 125 & 0 \\ \hline
\emph{Single} & \emph{Dag}         &    500 &    NA & 160,696 & 39,242 &   299 &  97 & 0 \\ \hline\hline

	  \multicolumn{9}{|c|}{\textbf{3-bit Multiply}, Kruskal-Wallis=$5.14\cdot 10^{-18}$} \\ \hline

\textbf{Duplication} & \textbf{Ordering} & \textbf{Genome Size} & \textbf{Mutation Rate} & \textbf{MES} & \textbf{MAD} & \textbf{Active} & \textbf{Used} & \textbf{p-value} \\ \hline
\emph{Skip} & \emph{Reorder}      & 10,000 & 0.0005  & 196,116  & 60,326  & 3,175 & 484 & 0  \\ \hline
\emph{Accumulate} & \emph{Reorder}& 10,000 & 0.001 & 199,193 & 85,970 & 3,117 & 489 & 0  \\ \hline
\emph{Single} & \emph{Dag} & 200  & NA & 239,844 & 78,739 & 114 & 74 & 0 \\ \hline
\emph{Single} & \emph{Reorder}    & 500   & NA & 243,664 & 91,935 & 266 & 131 & 0 \\ \hline
\emph{Skip} & \emph{Dag}          & 2,000 & 0.002 & 369,931 & 174,408 & 952 & 149 & 0.0100 \\ \hline
\emph{Accumulate} & \emph{Dag}    & 2,000 & 0.002  & 379,202 & 147,289 & 973 & 158 & 0.0053  \\ \hline
\emph{Single} & \emph{Normal}     & 5,000  & NA & 399,911 & 194,384 & 273 & 112 & 0.0047 \\ \hline
\emph{Accumulate} & \emph{Normal} & 2,000 & 0.002 & 483,938 & 227,686 & 168 & 87 & 0.45 \\ \hline
\rowcolor{Gray}
\emph{Skip} & \emph{Normal} & 2,000 & 0.002 & 657,121 & 311,376 & 175 & 84 & NA \\ \hline

	\end{tabular}
	\caption{Results for algorithm configuration on each problem.  Highlighted row
	is the control configuration \emph{Skip Normal}.  MES=Median Evaluations to Success; MAD=Median
	Absolute Deviation; p-value=Result from Mann-Whitney~U comparison with control.
	}
	\label{tab:results}
\end{table*}

The results from running each tuned parameter configuration for each algorithm
combination on each problem are summarized by Table~\ref{tab:results}.
Rows in each table are ordered by that algorithm combination's median evaluations to
success (MES), with the best algorithm combination appearing at the top of each table.
The highlighted row marks \emph{Skip Normal}, the control algorithm combination
representing traditional CGP's performance.  All configurations for each problem
were then testing using Kruskal-Wallis to determine if any configuration is
statistically different from any other.  All problems received highly significant
values.  As such, each algorithm combination was then compared with the control
to determine statistical significance using pairwise Mann-Whitney~U tests. The
resulting p-value for each test is reported in the table.

As discussed in Section~\ref{sec:parameter}, the genome size and mutation rate
for each algorithm combination was extensively optimized.  In line with previous
research~\cite{miller:2006:redundancy}, this led to massive genome sizes for
almost all algorithm combinations on all problems.  With the exception of 
some combinations using \emph{Single}, all algorithm combinations used over
1,000 nodes.  Half of those configurations used 10,000 nodes, the maximum allowed
by our tuning, which may imply these combinations work best with even larger
genomes.

Of the 12 configurations including \emph{Single}, 11 used a genome size smaller
than all configurations not using \emph{Single}.  The exception was \emph{Single Normal}
on Multiply.  No other algorithm had as much impact on genome size tuning as
the inclusion or exclusion of \emph{Single}.  In previous work~\cite{goldman:2013:ordering}
it appeared \emph{Reorder} and \emph{DAG} were allowing CGP to use smaller genomes.
In light of our results, this conclusion should be amended.
It appears that \emph{Single} is most effective when the genome size is small and
that \emph{Reorder} and \emph{DAG} are best able to cope with smaller genome sizes.
Independent of duplication method, the ordering methods do not seem to create any
pressure on which genome size is chosen.

\begin{comment}
Genome size explosion seems most prevalent on easier problems, with Parity and
Encode having much higher genome sizes than Decode and Multiply.  Initially
this seems counter intuitive, as we would expect harder problems to require
larger solutions which in turn should require larger genomes.  Yet the genome
sizes in use are so far beyond what a solution to these problems require that
solution size is likely not the cause of this behavior.  Instead, consider
the usefulness of unexplored nodes.

On simple enough problems, exploring lots of randomly generated nodes may be
sufficient to generate high quality solutions.  At relatively low numbers of
evaluations, large genotypes likely mean each mutation to a connection gene
will activate nodes that have never been active, resulting in exploration.
At higher numbers of evaluations, mutations to connection genes that are not currently
active have a high chance of connecting to a node that was tested previously
and found ineffective.  As a result, having a smaller genome size on harder problems
may reduce the chance of exploratory mutations when optimizing near the global optimum.

Under this theory, we should expect Parity to have the largest genome sizes
\end{comment}

Unlike genome size, all configurations choose approximately the same mutation rate,
with different rates for each problem.  With the exception of Parity, there appears
to be a general consensus that each gene should be mutated with probability 0.002,
independent of genome size.  Parity used a higher mutation rate, but this is likely
because 3-bit Parity is by far the easiest problem tested and therefore likely benefits
from higher exploration.

On many of the problems, many of the configurations have little to no statistical
difference from the control in their evaluations required to reach the global optimum.
On Parity and Encode it is probably safe to conclude that \emph{Single} is worse than
the control, although as these are the easiest problems tested that difference may have
little real world meaning.  \emph{DAG} is almost certainly worse than the control on
Decode.  Its results are so much worse that we can also likely conclude that \emph{DAG}
is worse than all non \emph{DAG} configuration on that problem, even though no statistical
tests are presented here.  The control on multiply did by far the worst of all configurations,
with only \emph{Accumulate Normal} likely failing statistical significance.

By ordering the algorithm combinations by MES and examining that ordering across each
problem, we can look for general trends.  From the tables, it appears on easier problems
which method of duplication is in use is the strongest predictor of rank, while
the ordering methods are more important on harder problems.  This comes from the
observation that in for Parity and Encode (Upper half of Table~\ref{tab:results})
the first column is effectively sorted, while for Decode and Multiply
(Lower half of Table~\ref{tab:results})
the second column is sorted.
Each table includes coincidentally one exception to these rules.  On Parity and
Encode pairings with \emph{DAG} are ranked lower than other ordering methods, so much
so as to break the ordering once on both problems.  On Decode \emph{Reorder Duplicate}
ranks worse than expected.  Finally on Multiply \emph{Single DAG} does far better
than expected.

On almost every problem, \emph{DAG} required
more evaluations to success than \emph{Reorder}, and it only outperformed \emph{Normal}
on Multiply.  Combined with its's algorithmic overhead described in
Section~\ref{sec:dag}, \emph{DAG} is therefore not likely to be useful for anything
but comparison purposes.

On all four problems, \emph{Reorder} either finds the solution in comparable
or significantly less evaluations than \emph{Normal} ordering, independent of
the duplicate method in use.  Combined with the complexity analysis in
Section~\ref{sec:reorder} which shows it is no more asymptotically complex,
we would suggest \emph{Reorder} as an alternative to \emph{Normal} on hard
problems.

The difference between using \emph{Skip} and \emph{Accumulate} remains
negligible~\cite{goldman:2013:cgpwaste}, even in combination with different ordering
methods.  As \emph{Accumulate} is far more complex in terms of algorithm function
than \emph{Skip}, we would suggest \emph{Skip} be used over \emph{Accumulate} for
any real world application.

Our parameter tuning focused on minimizing the number of evaluations each algorithm
combination required to solve problems, as that is the primary measure of runtime
for CGP and GP in general.  As such, comparing the true runtime of the tuned parameters
is likely to give potentially misleading results.  In using profilers and examining
general trends, one of the largest indicators of runtime per evaluation was genome size.
This makes sense as each time an offspring is generated all of the parent's genes
must first be copied.  Also, larger genome sizes correlate with more active nodes,
each of which must be executed on each input tested during individual evaluation.

With these caveats, the observed runtime for each algorithm combination follows
what we would expect given the number of evaluations performed and the genome size.
When \emph{Reorder} and \emph{Normal} use the same genome sizes, they require
nearly the same amount of time per evaluation, with \emph{Reorder}'s added
steps and increased number of active nodes constituting a relatively small increase.
In general this difference in time per evaluation is trumped by differences in
actual number of evaluations, which remains the primary cause of differential total runtime.
As predicted \emph{DAG} does take longer than the other algorithms to perform each evaluation,
reinforcing the conclusion that \emph{DAG} should not be used for applications.

As our implementation of \emph{Skip} and \emph{Accumulate} required each gene
to generate a random number to determine if it should mutate, these methods took
significantly longer per evaluation than \emph{Single}.  Utilizing a binomial
distribution, identical behavior could have been achieved in effectively constant
instead of linear time.  Further skewing the comparison is that \emph{Single}'s
tuning resulted in far smaller genomes, giving it a significant edge over the other
techniques in runtime per evaluation.

If we had selected on runtime as apposed to evaluations during tuning, we would expect whichever
technique is most effective at small genome sizes to obtain the best results.
From theoretical work in~\cite{goldman:2013:ordering}, we would expect \emph{Reorder}
to work better with this limitation than \emph{Normal}.  Assuming the tuned values
we obtained are any gauge, \emph{Single Reorder} will likely require the least
amount of runtime when tuned to reduce runtime.

\subsection{Scalability}
TODO Discuss using tuned parameters on different problem sizes and analyze
results.  Include O(N) table similar to previous paper.

TODO Include scaling comparison with~\cite{walker:2008:cgpmodules}.

\section{Analysis of Evolutionary Mechanisms}
\label{sec:analysis}
Comparing how many evaluations each algorithm takes to solve a set of benchmark
problems only provides us with a limited amount of understanding about how
those algorithms actually function.  We propose here to look for further detail,
both to check on theorized capabilities of each variant and CGP in general, and
to look for previously hidden aspects of CGP.
To do so we have devised a number of novel metrics to examine what makes a
CGP run successful.

As a primary concept to our analysis, we determine the \emph{footprint} of nodes
in the genome.  What we mean by \emph{footprint} is a measure the semantic use
of a node.  For each input that is given to an individual, all of its nodes
will produce a single output.  These output values can be recorded and ordered.
As the domain in use in binary, the outputs can be viewed as a bit string, with
each bit the value that a node outputs when an individual is presented with a
specific input.  Similar recording can be done for other domains, but binary
allows for by far the simplest encoding.

The \emph{footprint} of a node is a complete description of its functionality.
Any two nodes with identical \emph{footprint} values have identical behavior,
even if their method for encoding that behavior is drastically different.  From
this perspective, we can view CGP evolution as attempting to evolve a node who's
\emph{footprint} matches the desired output \emph{footprint}.  Doing so allows
us to examine how the evolutionary operators change the behavior of nodes,
not just their gene values.

Utilizing \emph{footprint} analysis, we can extract the semantically useful
portions of an individual, a subset of the active nodes.  Consider two active
nodes with the same \emph{footprint}.  As CGP is a graph representation, only
one of these nodes is actually necessary.  Given one in the genome, any node that reads
from the other can have its connection changed without changing its behavior.

\begin{figure}
  \begin{algorithmic}
  \Procedure{Simplify}{}
    \State $reconnect \leftarrow \emptyset$
    \ForAll{$node \in reversed(active)$}
      \State $reconnect[footprint(node)] \leftarrow node$
    \EndFor
    \ForAll{$i \in input\_locations$}
      \State $reconnect[footprint(i)] \leftarrow i$
    \EndFor
    \ForAll{$g \in genes$}
      \If{$g$ is a connection gene}
        \If{$g \in reconnect$}
          \State $g \leftarrow reconnect[footprint(g)]$
        \EndIf
      \EndIf
    \EndFor
  \EndProcedure
  \end{algorithmic}
  \caption{Algorithm to ensure each active \emph{footprint} is only
           calculated once.}
  \label{fig:simplify}
\end{figure}

The \Call{Simplify}{} algorithm given in Figure~\ref{fig:simplify} is designed
to efficiently reduce the redundancy encoded in a genome.  Generally, this
algorithm will move all connections to the first node that produces the desired
\emph{footprint}.  After application, and recalculation of which nodes are still
active, each active node produces a unique \emph{footprint}.  Any node that is
transitively dependent on its own \emph{footprint} will always be removed from
the active set.  In general, if a set of nodes all produce the same \emph{footprint},
whichever has the least transitive dependencies will be kept.  Note that this
does guarantee the minimum possible genome, an algorithm capable of that requires
exponential runtime.  The \Call{Simplify}{} algorithm requires $O(AN)$ time,
as the final for loop is the most expensive step.  As before, this means
\Call{Simplify}{} is no more complex than copying a genome.

By examining which nodes remain active after simplification, we can discuss
the usefulness of each node as either being part of the minimum solution, a
duplicate of a node in the minimum solution, or irrelevant to the solution.
As such we can then examine how redundancy is being used, how frequently useful
structures exist in the inactive nodes, and how useful nodes are constructed by evolution.

\begin{comment}
\begin{figure}
  \begin{algorithmic}
  \Procedure{RepeatedFootprint}{}
    \State $reachable \leftarrow \emptyset$
    \ForAll{$i \in input\_locations$}
      \State $reachable[i] \leftarrow footprint(i)$
    \EndFor
    \State $repeated \leftarrow \emptyset$
    \ForAll{$node \in active$}
      \State $working \leftarrow footprint(node)$
      \State $direct \leftarrow \emptyset$
      \ForAll{$link \in reads\_from(node)$}
        \State $direct \leftarrow direct \cup reachable[link]$
      \EndFor
      \If{$working \in direct$}
        \State $repeated \leftarrow repeated \cup \{working\}$
      \Else
        \State $direct \leftarrow direct \cup \{working\}$
      \EndIf
      \State $reachable[node] \leftarrow direct$
    \EndFor
    \State\Return{$repeated$}
  \EndProcedure
  \end{algorithmic}
  \caption{Algorithm to determine which nodes transitively depend on their
  own footprint.}
  \label{fig:repeated}
\end{figure}

An efficient algorithm to remove some of the redundancy in the genome is given
in Figure~\ref{fig:repeated}.  This algorithm finds the set of nodes which are
transitively dependent on their own footprint.  In other words, in order for
each node in the $repeated$ set, some other node in the genome has to produce
the same output.  As an example, consider node $Y$ which is in the $repeated$
set.  There must exist some node $X$ which $Y$ transitively depends on that has
the same footprint as $Y$.  As a result, $Y$ can always be remove, with anything
previously dependent on $Y$ now dependent on $X$.  All nodes in the $repeated$
set can therefore be removed from the genome without effecting its ability to
calculate any value.  This also has the potential to cause cascading removals
if there are intermediary nodes which are no longer useful once all of the
$repeated$ nodes have been removed.  In total this algorithm runs in $O(AN^2)$
time for an arbitrary individual.  Note that this is an offline algorithm,
so it only needs to be run on individuals used in production or for analysis.

\begin{figure}
  \begin{algorithmic}
  \Procedure{Simplify}{$p\_r$, $p\_c$, $p\_i$, $best$}
    \If{$|p\_r| = 0$}
      \State\Return{$p\_i$}
    \ElsIf{$|p\_i| + |p\_r| \geq |best|$}
      \State\Return{$best$}
    \EndIf
    \State $working, p\_a \leftarrow randpop(p\_r)$
    \State $covrd \leftarrow p\_c \cup {working}$
    \State $ansc \leftarrow p\_a \cup {working}$
    \ForAll{$node \in has\_footprint(working)$}
      \State $node\_valid \leftarrow True$
      \State $incd \leftarrow p\_i \cup {node}$
      \State $reqd \leftarrow p\_r$
      \ForAll{$link \in reads\_from(node)$}
        \State $foot \leftarrow footprint(link)$
        \If{$foot \in ansc$}
          \State $node\_valid \leftarrow False$
          \State \textbf{break}
        \ElsIf{$foot \notin covrd$}
          \State $reqd[foot] \leftarrow reqd[foot] \cup ansc$
        \EndIf
      \EndFor
      \If{$node\_valid$}
        \State $solution \leftarrow \Call{Simplify}{reqd, covrd, incd, best}$
        \If{$|solution| < |best|$}
          \State $best \leftarrow solution$
        \EndIf
      \EndIf
    \EndFor
    \State\Return{$best$}
  \EndProcedure
  \end{algorithmic}
  \caption{Recursive algorithm which determines the minimum set of nodes required
    to generate a set of required outputs.}
  \label{fig:simplify}
\end{figure}

While the previous algorithm is guaranteed to only remove redundant nodes, it
does not remove all redundant nodes.  In order to remove redundant nodes without
any transitive relationships, we must use the algorithm given in
Figure~\ref{fig:simplify}.  This is a recursive algorithm that tries all possible
choices of nodes that produce each footprint to see which combination results
in the fewest active nodes.  The algorithm takes in a mapping of footprints
required by previous calls to all ancestor footprints of those requirements ($p\_r$), footprints previously covered ($p\_c$),
nodes that have been previously included in the solution ($p\_i$), and smallest
set of nodes found so far which produces all required footprints ($best$).  Initially
$p\_c$ contains the set of input footprints, with $p\_r$ mapping each of the output
footprints to empty sets unless that footprint is already in $p\_c$.  $p\_i$
is initialized to the empty set, and $best$ starts out as the set of all $active$ nodes.
Recursion ends when the the nodes in $p\_i$ have no unsatisfied requirements or
all possible solutions containing $p\_i$ that can satisfy all requirements are
no better than the current best solution.  While this algorithm is exponential
in nature, in practice it has a reasonable runtime.  Using the algorithm in
Figure~\ref{fig:repeated} to prevent $repeated$ nodes from being returned by
$has\_footprint$ significantly reduces the number of combinations that need
to be tested.  Furthermore, by preventing recursion from exploring solutions
worse than the current known best, significant time can be saved.  Lastly, as
with the algorithm in Figure~\ref{fig:repeated}, this is an offline algorithm,
and has no impact on evolutionary speed.

Using the algorithm given in Figure~\ref{fig:simplify}, we can determine the
absolute minimum set of nodes in a genome's solution that are necessary to
reproduce the original output of that genome.  This allows us to discuss
the usefulness of each node as either being part of the minimum solution, a
duplicate of a node in the minimum solution, or irrelevant to the solution.
As such we can then examine how redundancy is being used, how frequently useful
structures exist in the inactive nodes, and how useful nodes are constructed by evolution.
This algorithm can also be used to construct simplified genomes when combined
with the algorithm given in Figure~\ref{fig:reorder}.
\end{comment}
\subsection{Never Active Nodes}
%TODO Somewhere in here you should talk about how Normal has the least number of active,
%which is probably related to how many never become active
In line with the suggestions made in~\cite{miller:2006:redundancy}, the evolved
number of active nodes across all problems and all algorithm configurations
in Table~\ref{tab:results}
was significantly lower than the genome size.  With the exception of a few combinations
with \emph{DAG} and \emph{Single}, the evolved results were over 50\% inactive.
\emph{Skip Normal}, our control representing normal CGP, had only 1\%, 3\%, 20\%,
and 9\% of nodes in the genome active in the median runs of Parity, Encode, Decode,
and Multiply, respectively.  These final results fall in line with the predictions
made in~\cite{goldman:2013:ordering}, in that if the problems are ordered based on
their number of output locations, their evolved percentage of active nodes is also ordered.

With how inactive the final genomes are, we ask ``what is the rest for?''
Just because a node is not active in the final solution does not mean it was never
useful.  For a node to be truly of no use to search, we must consider nodes that
were never active.  Never active nodes are nodes such that for all parents of the
final solution (the line of descent), no parent had that node active.  Note that
for \emph{Reorder} this demarcation of node activity follows the node through
each shuffle, such that if a node was marked as never active before the genome
was shuffled it is marked as never active after the shuffle, and vice versa.
This is in contrast to marking locations as never active, which was not done in
this study.

\begin{table*}
	\centering
  \begin{tabular}{c|c|c|c|}
    \cline{2-4}
    & \textbf{\em Normal} & \textbf{\em Reorder} & \textbf{\em DAG} \\ \hline
    %\begin{comment}
    \thirdlabel{Skip} & \graphicthird{multiply_skip_normal} &
                        \graphicthird{multiply_skip_reorder} &
                        \graphicthird{multiply_skip_dag}\\ \hline
    \thirdlabel{Accumulate} & \graphicthird{multiply_accumulate_normal} &
                              \graphicthird{multiply_accumulate_reorder} &
                              \graphicthird{multiply_accumulate_dag}\\ \hline
    \thirdlabel{Single} & \graphicthird{multiply_single_normal} &
                          \graphicthird{multiply_single_reorder} &
                          \graphicthird{multiply_single_dag}\\ \hline
    %\end{comment}
	\end{tabular}
	\caption{Location of never active nodes across all 50 runs for all nine algorithm combinations on the Multiply problem.}
	\label{tab:never_active}
\end{table*}

Table~\ref{tab:never_active} provides information about never active nodes for
each algorithm combination on the 3-bit Multiply problem.  The horizontal axis
in each of the nine plots shows different genome locations, while the vertical
axis denotes the level of fitness the recorded individual obtained.  The shading
of each coordinate gives the percentage of 50 runs that had an individual of
the given fitness level such that the node at the given location had never
previously been active.  While only Multiply is shown here, all of the
other problems tested had similar behavior, with allowances for differences in
number of evaluations performed.

Examining the columns of Table~\ref{tab:never_active} makes it clear that
each of the ordering methods had significantly different never active behavior.
Looking at \emph{Normal} we see that only at very high fitness levels does evolution
begin to activate nodes in the latter half of the genome.  Even by the final solution
large percentages of the genome were never activated, with \emph{Skip}, \emph{Accumulate},
and \emph{Single} having median percentages of never active nodes of 43.35\%, 42.05\%, and
56.17\%, respectively.  This is in stark contrast to \emph{Reorder} and \emph{DAG},
which never had a median higher than 0.05\%.

Compared to the difference in columns, the difference in the rows of Table~\ref{tab:never_active}
caused by the different duplication methods were almost non-existent.  Only
\emph{Single} has a distinguishable behavior, but this is almost certainly the result
of the genome size and not the duplication method.  \emph{Single Normal} used
a genome size 2.5 times larger than \emph{Normal} did with the other techniques,
so the resulting increase in never active nodes makes sense.  When \emph{Single}
was paired with \emph{Reorder} and \emph{DAG} the genome sizes were far smaller,
resulting in noisier graphs with less never active nodes.

So now the question is why does \emph{Normal} use so little of its genome?  To
answer this we must consider what can prevent a node from becoming active.  For
a node to never be active all tested offspring with mutations that result in that node being activated
must have been less fit than their parent.  Even if those offspring are no more fit, the selection method
used by CGP would have allowed that offspring to replace its parent.  This leads
to two possible causes for a node never being active: bad genes or infrequently tried.

The bad genes cause for a node never being active suggests the behavior encoded
by a node is simply useless on the problem landscape.  If this was the root cause
of never active nodes, we would expect \emph{Reorder} and \emph{DAG} to have
similar problems, as neither technique explicitly changes the likelihood of node
behaviors.  Furthermore, we would expect the location of never active nodes to be
uniform in the genome, which is obviously not the case.  Finally, even if bad
genes were the problem, we would expect the mutational drift allowed by CGP's
selection method to change their bad genes and eventually allow them to be active.

The infrequently tried cause for a node never being active suggests that few
individuals are ever produced which have that node active, limiting the likelihood
of those individuals being selected regardless of potential fitness.  Under this
cause and combined with the theory given in~\cite{goldman:2013:ordering}, we
expect a strong positional bias in the location of the never active nodes.  Furthermore,
we would expect nodes near the end of the genome to be the most likely to be never active.
Finally we would expect \emph{Reorder} and \emph{DAG} to have less or no positional bias.
All of these expectations match the results given in Table~\ref{tab:never_active}.

Examining the progression in fitness across the different ordering methods shows
the effect of initialization on each.  All three methods share the same initialization
method, and as such all have very similar never active node behavior early in
evolution.  This initialization similarity explains why \emph{DAG} has any positional
bias at all, even though its evolutionary mechanisms have none.  Similarly \emph{Reorder}
diverges from \emph{Normal} given sufficient time, eventually losing its positional
bias.

This early similarity may be part of the reason \emph{Normal} and \emph{Reorder}
have such similar behavior on relatively easy problems.  After relatively few
evaluations both have explored similar sections of their genome.  Yet on hard
problems that require further exploration \emph{Reorder} is able to do so, while
\emph{Normal} remains positionally limited.

\subsection{Node Behaviors}


\begin{table*}
	\centering
  \begin{tabular}{c|c|c|c|}
    \cline{2-4}
    & \textbf{\em Normal} & \textbf{\em Reorder} & \textbf{\em DAG} \\ \hline
    \thirdlabel{Skip} & \graphicthird{bar_encode_skip_normal} &
                        \graphicthird{bar_encode_skip_reorder} &
                        \graphicthird{bar_encode_skip_dag}\\ \hline
    \thirdlabel{Accumulate} & \graphicthird{bar_encode_accumulate_normal} &
                              \graphicthird{bar_encode_accumulate_reorder} &
                              \graphicthird{bar_encode_accumulate_dag}\\ \hline
    \thirdlabel{Single} & \graphicthird{bar_encode_single_normal} &
                          \graphicthird{bar_encode_single_reorder} &
                          \graphicthird{bar_encode_single_dag}\\ \hline
	\end{tabular}
	\caption{Average node behavior for all nine algorithm combinations on the Encode problem.
	         Excludes never active nodes.}
	\label{tab:encode_behavior}
\end{table*}

\begin{table*}
	\centering
  \begin{tabular}{c|c|c|c|}
    \cline{2-4}
    & \textbf{\em Normal} & \textbf{\em Reorder} & \textbf{\em DAG} \\ \hline
    \thirdlabel{Skip} & \graphicthird{bar_multiply_skip_normal} &
                        \graphicthird{bar_multiply_skip_reorder} &
                        \graphicthird{bar_multiply_skip_dag}\\ \hline
    \thirdlabel{Accumulate} & \graphicthird{bar_multiply_accumulate_normal} &
                              \graphicthird{bar_multiply_accumulate_reorder} &
                              \graphicthird{bar_multiply_accumulate_dag}\\ \hline
    \thirdlabel{Single} & \graphicthird{bar_multiply_single_normal} &
                          \graphicthird{bar_multiply_single_reorder} &
                          \graphicthird{bar_multiply_single_dag}\\ \hline
	\end{tabular}
	\caption{Average node behavior for all nine algorithm combinations on the Multiply problem.
	         Excludes never active nodes.}
	\label{tab:multiply_behavior}
\end{table*}

In CGP it is common to report what fraction of the genome is active, as this represents
the easiest estimate of solution size.  By utilizing \emph{footprint} comparison
and the \Call{Simplify}{} algorithm, we can get a much better estimate of what
purpose each node in the genome serves.
Table~\ref{tab:results} provides two columns relevant to node behavior: Active
and Used.  The former gives the traditional measure of the median amount of
active nodes in the solution found.  The latter is the median number of nodes
still active after applying the \Call{Simplify}{} algorithm.

On all problems and for all algorithm combinations, the number of active nodes is
far smaller than the genome size, and the number of Used nodes
is far smaller than the number of Active nodes.  In some extreme cases less
than 1\% of active nodes in the evolved solution are actually necessary.  Only the
most extreme configurations have even 50\% of active nodes in use by the
simplified solution.

Table~\ref{tab:encode_behavior} and Table~\ref{tab:multiply_behavior} help illustrate
the behavioral breakdown of nodes for each algorithm combination on the Encode
and Multiply problems, respectively.  These two problems were chosen as they
represent different ends of the quality spectrum for \emph{Skip Normal} and
because together the are sufficiently representative of behaviors seen on the other
problems.

Each diagram describes node behavior as falling into eight distinct categories.
The first category are the constant nodes, both active and inactive.  These nodes
return the same value for all tested inputs.  The ``Used'' bar is the average
number of nodes which are part of the simplified solution.  The ``Useful'' bars
express how many nodes, either active or inactive, have identical \emph{footprint}
values to used nodes.  These nodes represent either independent methods of calculating
the same behavior or are effectively pass through nodes which mimic the behavior
of their inputs.  Active ``Intron'' nodes are those active nodes that do not
fall into either the constant or used categories, with inactive ``Intron'' nodes
being those inactive nodes that have identical \emph{footprint} values to active
intron nodes.  The final bar is ``Explore'', which marks how many nodes on
average produce a \emph{footprint} not part of any previous category.  Note that
all never active nodes are ignored by these plots, as their behavior is meaningless.

The first striking feature of these tables is how many nodes produce constant
outputs.  As the instruction set used in our genomes never require constant values
to produce any non constant behavior (TRUE NAND X $\Leftrightarrow$ X NAND X)
and most operations with constants result in pass through nodes
(FALSE OR X $\Leftrightarrow$ X), we would not expect evolution to proliferate
these constant nodes.  If these constants are in fact of no use, future work
could likely improve CGP's performance by reducing their prevalence.  Yet
it is possible they serve an evolutionary function by allowing mutation more
flexibility.  For instance, pass through nodes may be allowing mutation to create
adjustments not easily done if this behavior was forbidden.

As with previous analysis, \emph{Skip} and \emph{Accumulate} have nearly identical
plots, regardless of ordering method and problem.  \emph{Single} again has
a significantly different plot when paired with \emph{Reorder} or \emph{DAG},
with some hits of a change on Encode with \emph{Normal}.  These combinations
were also the most likely to have significantly smaller genome sizes.  From the
results tables we see that the number of used nodes on each problem is relatively
constant across the combinations even though the number of active nodes and the
genome sizes changed significantly.  As such, the proportion of nodes in the
used category will vary as these other features change.

From Table~\ref{tab:encode_behavior} and Table~\ref{tab:multiply_behavior}, having
the plurality of active nodes in the used category appears to correlate with
increase evaluations to success.  Notice that in all \emph{Single} plots the
tallest bar for active nodes is ``Used,'' and the only non \emph{Single} plots
where this is true are \emph{Normal} on Multiply.  On Encode, \emph{Single}
runs had the highest median evaluations to success, with \emph{Normal} runs
doing similarly on Multiply.  This appears to imply that having duplication
of node behavior in the active nodes, and not just inactive nodes, is beneficial
to search.  If this is true, it may help explain CGP uses such large genomes,
as \emph{Normal} is only able to achieve enough active nodes to allow duplication
when the genome size significantly exceeds the actual solution size.  Also
the success of \emph{Reorder} and \emph{DAG} on Multiply may also be partially be attributable
to this ability to increase the number of active nodes.

One significant difference in behavior between the two problems is proportion of
explore nodes.  A potential explanation is that on 16-4-bit Encode there are only
$2^{16} = 65,536$ \emph{footprints} while on 3-bit Multiply there are $2^{64} = 1.845\cdot 10^{19}$
\emph{footprints}.  Combined with the fact that not all \emph{footprints} are equally
likely to be produced, the difference may just be that on Encode nodes are
that much more likely to recreate existing behavior.  Another possibility is that
CGP increases the diversity of behavior of inactive nodes over evolutionary time.
As such, the fact that Multiply required approximately an order of magnitude more
evaluations to solve would lead to the discrepancy.

\subsection{Parent Replacement}

\begin{table*}
	\centering
	\begin{tabular}{|c|c|c|c|c|}
	  \hline
	  \multicolumn{5}{|c|}{\textbf{3-bit Parity}} \\ \hline
\textbf{Duplicate} & \textbf{Ordering} & \textbf{Active Unchanged} & \textbf{No Reactivation} & \textbf{Reactivated Changed} \\ \hline
       \emph{Skip} &     \emph{DAG} & 0.00\% & 3.48\% & 55.63\% \\ \hline
 \emph{Accumulate} &     \emph{DAG} & 0.00\% & 8.48\% & 52.82\% \\ \hline
 \emph{Accumulate} & \emph{Reorder} & 2.09\% & 33.26\% & 58.46\% \\ \hline
   \rowcolor{Gray}
       \emph{Skip} &  \emph{Normal} & 2.25\% & 39.35\% & 67.17\% \\ \hline
 \emph{Accumulate} &  \emph{Normal} & 2.69\% & 39.99\% & 67.27\% \\ \hline
       \emph{Skip} & \emph{Reorder} & 11.46\% & 53.99\% & 59.47\% \\ \hline
     \emph{Single} & \emph{Reorder} & 24.94\% & 80.49\% & 60.63\% \\ \hline
     \emph{Single} &  \emph{Normal} & 26.44\% & 83.55\% & 59.96\% \\ \hline
     \emph{Single} &     \emph{DAG} & 27.80\% & 83.73\% & 59.92\% \\ \hline\hline

  \multicolumn{5}{|c|}{\textbf{16-4-bit Encode}} \\ \hline

\textbf{Duplicate} & \textbf{Ordering} & \textbf{Active Unchanged} & \textbf{No Reactivation} & \textbf{Reactivated Changed} \\ \hline
 \emph{Accumulate} &     \emph{DAG} & 0.05\% & 4.06\% & 48.70\% \\ \hline
       \emph{Skip} &     \emph{DAG} & 0.06\% & 3.85\% & 48.33\% \\ \hline
 \emph{Accumulate} & \emph{Reorder} & 1.77\% & 19.24\% & 55.47\% \\ \hline
       \emph{Skip} & \emph{Reorder} & 2.02\% & 19.80\% & 56.15\% \\ \hline
   \rowcolor{Gray}
       \emph{Skip} &  \emph{Normal} & 14.16\% & 54.78\% & 64.14\% \\ \hline
 \emph{Accumulate} &  \emph{Normal} & 14.52\% & 54.79\% & 64.17\% \\ \hline
     \emph{Single} &  \emph{Normal} & 33.77\% & 79.34\% & 59.86\% \\ \hline
     \emph{Single} &     \emph{DAG} & 35.98\% & 80.91\% & 58.39\% \\ \hline
     \emph{Single} & \emph{Reorder} & 39.95\% & 78.91\% & 59.72\% \\ \hline\hline

  \multicolumn{5}{|c|}{\textbf{4-16-bit Decode}} \\ \hline

\textbf{Duplicate} & \textbf{Ordering} & \textbf{Active Unchanged} & \textbf{No Reactivation} & \textbf{Reactivated Changed} \\ \hline
 \emph{Accumulate} &     \emph{DAG} & 0.03\% & 10.10\% & 54.99\% \\ \hline
       \emph{Skip} &     \emph{DAG} & 0.05\% & 10.21\% & 51.33\% \\ \hline
 \emph{Accumulate} & \emph{Reorder} & 9.98\% & 49.65\% & 70.57\% \\ \hline
       \emph{Skip} & \emph{Reorder} & 12.53\% & 51.63\% & 70.24\% \\ \hline
     \emph{Single} & \emph{Reorder} & 28.38\% & 79.73\% & 66.85\% \\ \hline
     \emph{Single} &  \emph{Normal} & 29.43\% & 80.86\% & 65.85\% \\ \hline
     \emph{Single} &     \emph{DAG} & 32.60\% & 81.73\% & 57.70\% \\ \hline
 \emph{Accumulate} &  \emph{Normal} & 51.11\% & 84.54\% & 72.57\% \\ \hline
   \rowcolor{Gray}
       \emph{Skip} &  \emph{Normal} & 65.47\% & 89.17\% & 72.21\% \\ \hline\hline

  \multicolumn{5}{|c|}{\textbf{3-bit Multiply}} \\ \hline

\textbf{Duplicate} & \textbf{Ordering} & \textbf{Active Unchanged} & \textbf{No Reactivation} & \textbf{Reactivated Changed} \\ \hline
 \emph{Accumulate} & \emph{Reorder} & 1.14\% & 22.09\% & 71.75\% \\ \hline
 \emph{Accumulate} &     \emph{DAG} & 4.52\% & 45.38\% & 63.42\% \\ \hline
       \emph{Skip} &     \emph{DAG} & 5.05\% & 45.58\% & 63.03\% \\ \hline
       \emph{Skip} & \emph{Reorder} & 10.40\% & 47.67\% & 71.79\% \\ \hline
     \emph{Single} &  \emph{Normal} & 18.85\% & 78.19\% & 70.11\% \\ \hline
     \emph{Single} &     \emph{DAG} & 20.72\% & 80.20\% & 65.10\% \\ \hline
     \emph{Single} & \emph{Reorder} & 22.14\% & 80.21\% & 69.76\% \\ \hline
 \emph{Accumulate} &  \emph{Normal} & 57.69\% & 87.27\% & 78.95\% \\ \hline
   \rowcolor{Gray}
       \emph{Skip} &  \emph{Normal} & 73.46\% & 92.04\% & 79.07\% \\ \hline
	\end{tabular}
	\caption{Behavior of offspring replacement for each algorithm combination
	         on each problem.}
	\label{tab:reactivate}
\end{table*}

An important aspect of CGP is its replacement strategy that allows for neutral drift.
Each generation the best offspring replaces its parent if it is no less fit.  Previous
research has shown the impact removing this feature can have on
performance~\cite{yu:2001:neutrality}, but how it interacts with the different variants
of CGP has not been investigated previously.

Table~\ref{tab:reactivate} provides a
novel look at CGP's replacement behavior.  Each time an offspring replaced a parent
in the final runs, we recorded how the behavior of each node differed between the
parent and the offspring.

The \textbf{Active Unchanged} reports how often all active nodes in the offspring
have identical \emph{footprints} to the corresponding nodes in the offspring's parent.
This goes beyond the concept of actively identical offspring and checks of the behavior
of any active node was modified by the mutation.  The rows in each of the four
tables are sorted using this column.  Smaller values indicate an algorithm
combination that produced more changes in active node behavior.

The \textbf{No Reactivation} column uses the word reactivation to refer to nodes
that were active in some ancestor of the offspring, were inactive in the offspring's
parent, but are active in the offspring.  Reactivation is a measure of how
often CGP reuses previously useful active nodes.  Reported in the table is how
frequently an offspring replaces its parent with no reactivated nodes.  Smaller
values indicate an algorithm combination that reuses nodes, as opposed to
including randomly generated nodes (never active nodes) or ignoring inactive nodes
entirely.

The \textbf{Reactivated Changed} column states the percentage of reactivated nodes
that have a different \emph{footprint} from the last time they were active.  Conversely,
reactivated nodes without a change mean that CGP has taken a previously useful behavior,
stored it in the inactive space, and reactivated it in later mutations.  Smaller values
indicate an algorithm combination that more often preserve node behaviors while
inactive.

Combined these three measures provide somewhat surprising evidence.  Most notably
is that \emph{Single}, a technique designed to force offspring to be actively different
than their parents, produces offspring with no changes in active node behavior
20\%-40\% of the time.  The further exposes CGP's ability to have genetically different
yet phenotypically identical individuals.  Even more surprising is that \emph{Single}
appears to increase the chance of producing offspring without changing active behavior.
Somewhat unexpectedly \emph{Normal} appears to also reduce changes to active node behavior,
causing the most effect after \emph{Single}.

The likely cause of both is the number
of active nodes receiving mutations each generation.  Even though \emph{Single}
forces a gene in an active node to be mutated, it limits mutation to only one change.
If this one change does not change that node's behavior, the offspring will be marked
as active unchanged.  Conversely, \emph{Skip} and \emph{Accumulate} can mutate any
number of active genes each generation, improving their chance of a change in behavior.
Similarly, even though \emph{Normal} makes no explicit change in mutation, it is characterized
by a significantly smaller number of active nodes than \emph{Reorder} or \emph{DAG}.  More
active nodes means more chances for mutation to change the behavior of at least one.

Somewhat unintuitively, node reactivation and active node \emph{footprint} changes appear
to be highly correlated.  This is apparent in the fact that the \textbf{No Reactivation}
column in each table is almost perfectly sorted, even though rows were sorted using the
\textbf{Active Unchanged} column.  Again the number of active nodes seems the likely
cause.  Having fewer active nodes reduces the number of introns, making it more
difficult for inactive nodes to become active.  Fewer active nodes also decreases
how many inactive nodes could have ever been active.

Potentially the most interesting results come from the control combination of
\emph{Skip Normal}.  As problem difficulty increases, \emph{Skip Normal} produces
an increasingly large number of offspring with no change in active node behavior,
with little to no node reactivation, and a high chance of changing nodes before
they are reactivated.  This suggests two important features: \emph{Skip} is likely
saving a significant number of evaluations, and \emph{Normal} CGP is likely using
inactive nodes primarily to inject random behavior.  The former comes from the
argument that a large part of the active unchanged individuals are also likely
actively identical to their parents.  The latter is supported by the fact that
\emph{Skip Normal} so infrequently reactivates a node with tested behavior.  More
often than not offspring connecting in inactive nodes are testing that node's behavior
for the first time.

\subsection{Source of High Variance}
TODO Look for general indicators of high evaluation runs.

TODO Look for footprint moving.

TODO Look for active node clustering.

TODO Look into how often children replace parents.


% An example of a floating figure using the graphicx package.
% Note that \label must occur AFTER (or within) \caption.
% For figures, \caption should occur after the \includegraphics.
% Note that IEEEtran v1.7 and later has special internal code that
% is designed to preserve the operation of \label within \caption
% even when the captionsoff option is in effect. However, because
% of issues like this, it may be the safest practice to put all your
% \label just after \caption rather than within \caption{}.
%
% Reminder: the "draftcls" or "draftclsnofoot", not "draft", class
% option should be used if it is desired that the figures are to be
% displayed while in draft mode.
%
%\begin{figure}[!t]
%\centering
%\includegraphics[width=2.5in]{myfigure}
% where an .eps filename suffix will be assumed under latex, 
% and a .pdf suffix will be assumed for pdflatex; or what has been declared
% via \DeclareGraphicsExtensions.
%\caption{Simulation Results.}
%\label{fig_sim}
%\end{figure}

% Note that IEEE typically puts floats only at the top, even when this
% results in a large percentage of a column being occupied by floats.


% An example of a double column floating figure using two subfigures.
% (The subfig.sty package must be loaded for this to work.)
% The subfigure \label commands are set within each subfloat command,
% and the \label for the overall figure must come after \caption.
% \hfil is used as a separator to get equal spacing.
% Watch out that the combined width of all the subfigures on a 
% line do not exceed the text width or a line break will occur.
%
%\begin{figure*}[!t]
%\centering
%\subfloat[Case I]{\includegraphics[width=2.5in]{box}%
%\label{fig_first_case}}
%\hfil
%\subfloat[Case II]{\includegraphics[width=2.5in]{box}%
%\label{fig_second_case}}
%\caption{Simulation results.}
%\label{fig_sim}
%\end{figure*}
%
% Note that often IEEE papers with subfigures do not employ subfigure
% captions (using the optional argument to \subfloat[]), but instead will
% reference/describe all of them (a), (b), etc., within the main caption.


% An example of a floating table. Note that, for IEEE style tables, the 
% \caption command should come BEFORE the table. Table text will default to
% \footnotesize as IEEE normally uses this smaller font for tables.
% The \label must come after \caption as always.
%
%\begin{table}[!t]
%% increase table row spacing, adjust to taste
%\renewcommand{\arraystretch}{1.3}
% if using array.sty, it might be a good idea to tweak the value of
% \extrarowheight as needed to properly center the text within the cells
%\caption{An Example of a Table}
%\label{table_example}
%\centering
%% Some packages, such as MDW tools, offer better commands for making tables
%% than the plain LaTeX2e tabular which is used here.
%\begin{tabular}{|c||c|}
%\hline
%One & Two\\
%\hline
%Three & Four\\
%\hline
%\end{tabular}
%\end{table}


% Note that IEEE does not put floats in the very first column - or typically
% anywhere on the first page for that matter. Also, in-text middle ("here")
% positioning is not used. Most IEEE journals use top floats exclusively.
% Note that, LaTeX2e, unlike IEEE journals, places footnotes above bottom
% floats. This can be corrected via the \fnbelowfloat command of the
% stfloats package.



\section{Conclusion}
\label{sec:conclusion}
TODO The conclusion goes here.





% if have a single appendix:
%\appendix[Proof of the Zonklar Equations]
% or
%\appendix  % for no appendix heading
% do not use \section anymore after \appendix, only \section*
% is possibly needed

% use appendices with more than one appendix
% then use \section to start each appendix
% you must declare a \section before using any
% \subsection or using \label (\appendices by itself
% starts a section numbered zero.)
%

\begin{comment}
\appendices
\section{Example Appendix}
Appendix one text goes here.

% you can choose not to have a title for an appendix
% if you want by leaving the argument blank
\section{}
Appendix two text goes here.

\end{comment}
% use section* for acknowledgement
\section*{Acknowledgment}


The authors would like to thank...


% Can use something like this to put references on a page
% by themselves when using endfloat and the captionsoff option.
\ifCLASSOPTIONcaptionsoff
  \newpage
\fi



% trigger a \newpage just before the given reference
% number - used to balance the columns on the last page
% adjust value as needed - may need to be readjusted if
% the document is modified later
%\IEEEtriggeratref{8}
% The "triggered" command can be changed if desired:
%\IEEEtriggercmd{\enlargethispage{-5in}}

% references section

% can use a bibliography generated by BibTeX as a .bbl file
% BibTeX documentation can be easily obtained at:
% http://www.ctan.org/tex-archive/biblio/bibtex/contrib/doc/
% The IEEEtran BibTeX style support page is at:
% http://www.michaelshell.org/tex/ieeetran/bibtex/
%\bibliographystyle{IEEEtran}
% argument is your BibTeX string definitions and bibliography database(s)
%\bibliography{IEEEabrv,../bib/paper}
%
% <OR> manually copy in the resultant .bbl file
% set second argument of \begin to the number of references
% (used to reserve space for the reference number labels box)
%\begin{thebibliography}{1}

%\bibitem{IEEEhowto:kopka}
%H.~Kopka and P.~W. Daly, \emph{A Guide to \LaTeX}, 3rd~ed.\hskip 1em plus
%  0.5em minus 0.4em\relax Harlow, England: Addison-Wesley, 1999.

%\end{thebibliography}

\bibliographystyle{IEEEtran}
\bibliography{IEEEabrv,../main}

% biography section
% 
% If you have an EPS/PDF photo (graphicx package needed) extra braces are
% needed around the contents of the optional argument to biography to prevent
% the LaTeX parser from getting confused when it sees the complicated
% \includegraphics command within an optional argument. (You could create
% your own custom macro containing the \includegraphics command to make things
% simpler here.)
%\begin{IEEEbiography}[{\includegraphics[width=1in,height=1.25in,clip,keepaspectratio]{mshell}}]{Michael Shell}
% or if you just want to reserve a space for a photo:

\begin{IEEEbiography}{Brian W. Goldman}
TODO
\end{IEEEbiography}

% if you will not have a photo at all:
\begin{IEEEbiographynophoto}{William F. Punch}
TODO
\end{IEEEbiographynophoto}

% insert where needed to balance the two columns on the last page with
% biographies
%\newpage

% You can push biographies down or up by placing
% a \vfill before or after them. The appropriate
% use of \vfill depends on what kind of text is
% on the last page and whether or not the columns
% are being equalized.

%\vfill

% Can be used to pull up biographies so that the bottom of the last one
% is flush with the other column.
%\enlargethispage{-5in}



% that's all folks
\end{document}


