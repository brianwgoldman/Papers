
%%%%%%%%%%%%%%%%%%%%%%% file typeinst.tex %%%%%%%%%%%%%%%%%%%%%%%%%
%
% This is the LaTeX source for the instructions to authors using
% the LaTeX document class 'llncs.cls' for contributions to
% the Lecture Notes in Computer Sciences series.
% http://www.springer.com/lncs       Springer Heidelberg 2006/05/04
%
% It may be used as a template for your own input - copy it
% to a new file with a new name and use it as the basis
% for your article.
%
% NB: the document class 'llncs' has its own and detailed documentation, see
% ftp://ftp.springer.de/data/pubftp/pub/tex/latex/llncs/latex2e/llncsdoc.pdf
%
%%%%%%%%%%%%%%%%%%%%%%%%%%%%%%%%%%%%%%%%%%%%%%%%%%%%%%%%%%%%%%%%%%%


\documentclass[runningheads,a4paper]{llncs}
%\linespread{2.5}
\usepackage{amssymb}
\setcounter{tocdepth}{3}
\usepackage{graphicx}
\usepackage{verbatim}
\usepackage{url}
\usepackage{hhline}
\newcommand{\includegraphicsfit}[1]
{\includegraphics[width=.9\textwidth,height=\textheight,keepaspectratio]{#1}}
%\urldef{\mailsa}\path|{alfred.hofmann, ursula.barth, ingrid.haas, frank.holzwarth,|
%\urldef{\mailsb}\path|anna.kramer, leonie.kunz, christine.reiss, nicole.sator,|
%\urldef{\mailsc}\path|erika.siebert-cole, peter.strasser, lncs}@springer.com|    
\newcommand{\keywords}[1]{\par\addvspace\baselineskip
\noindent\keywordname\enspace\ignorespaces#1}

\begin{document}

\mainmatter  % start of an individual contribution

% first the title is needed
\title{Reducing Wasted Evaluations in Cartesian Genetic Programming}

% a short form should be given in case it is too long for the running head
%\titlerunning{Lecture Notes in Computer Science: Authors' Instructions}

% the name(s) of the author(s) follow(s) next
%
% NB: Chinese authors should write their first names(s) in front of
% their surnames. This ensures that the names appear correctly in
% the running heads and the author index.
%
\author{Brian W.~Goldman\and William F. Punch}
%\author{Anonymous Author\and Anonymous Author}
%
%\authorrunning{Lecture Notes in Computer Science: Authors' Instructions}
% (feature abused for this document to repeat the title also on left hand pages)

% the affiliations are given next; don't give your e-mail address
% unless you accept that it will be published
\institute{BEACON Center for the Study of Evolution in Action,\\
Michigan State University, U.S.A.\\
brianwgoldman@acm.org, punch@msu.edu}
%\institute{Department\\ Organization\\ Email}


%
% NB: a more complex sample for affiliations and the mapping to the
% corresponding authors can be found in the file "llncs.dem"
% (search for the string "\mainmatter" where a contribution starts).
% "llncs.dem" accompanies the document class "llncs.cls".
%

%\toctitle{Lecture Notes in Computer Science}
%\tocauthor{Authors' Instructions}
\maketitle


\begin{abstract}
Cartesian Genetic Programming~(CGP) is a form of Genetic Programming~(GP)
where a large proportion of the genome is identifiably unused
by the phenotype.  This can lead mutation to create offspring that are 
genotypically different but phenotypically identical, and therefore do
not need to be evaluated.  We investigate theoretically and empirically
the effects of avoiding these otherwise wasted evaluations, and provide
evidence that doing so reduces the median number of evaluations to solve
four benchmark problems, as well as reducing CGP's sensitivity to the mutation
rate.  The similarity of results across the problem set in combination with the
theoretical conclusions supports the general need for avoiding these unnecessary
evaluations.

%The abstract should summarize the contents of the paper and should
%contain at least 70 and at most 150 words. It should be written using the
%\emph{abstract} environment.

\keywords{cartesian genetic programming, mutation}
\end{abstract}


\section{Introduction}
%TODO Consider change to "best results"
In Genetic Programming~(GP) the most common metric for measuring algorithm complexity
is the number of fitness evaluations required to solve
a black box problem.  This metric assumes that the cost of evaluating an
individual will dominate search times on real world problems.  Under this
metric, search algorithms that best exploit evaluation information and avoid
unnecessary evaluations will receive the best results.  Cartesian Genetic
Programming~(CGP), a branch of GP, has a particular structure that allows
for the detection of some phenotypically identical genotypes without
requiring fitness evaluation.  No previous work has investigated if this
property can be exploited to meaningfully improve CGP efficiency.  After discussing the
potential theoretical gains of avoiding these evaluations, we propose and
investigate three novel techniques designed to improve CGP's
search.


\section{Cartesian Genetic Programming}
\label{sec:cgp}
Cartesian Genetic Programming is a variant of genetic programming in which
individuals encode directed acyclic graphs (DAG).
An in-depth explanation of CGP is provided in~\cite{miller:2011:chapter2}, but for
our purposes the important features are:
\begin{itemize}
  \item DAGs are represented by a collection of nodes and genes specifying output locations.
  \item Nodes contain genes describing both what function they perform
  and how they connect to other nodes.
  \item Offspring are created using mutation.
  \item Offspring replace parents if they are of the same or greater fitness.
\end{itemize}
CGP's encoding method does not require a node
to connect to an output location, which is required for it to
affect the phenotype of the individual.
These unconnected nodes and their genes are commonly called ``inactive''
and the connected nodes and their genes are called ``active.''
As the topology of the DAG is constantly evolving, the number and location of
active nodes can change from parent to child.
There have been numerous studies on the usefulness of inactive nodes~\cite{vassilev2000neutrality,yu:2001:neutrality}
including an argument that it is desirable to have up
to $95\%$ of the genome inactive~\cite{miller:2006:redundancy}.

CGP's canonical variation operator is point mutation.  Unfortunately the details
of this operator are often ambiguously stated.  For instance, in some papers this
operator chooses a set number of genes at random to be mutated~\cite{miller:2011:chapter2},
while in other papers each gene can be mutated with a certain probability,
allowing any number of genes to be mutated at once~\cite{harding:2012:mtcgp}.  Also, mutation in CGP can either
force a mutated gene's value to change~\cite{harding:2012:mtcgp,miller:2011:chapter2},
or randomly reset it to any valid value including the genes previous value~\cite{miller:2006:redundancy}.
As much of our analysis hinges on
the behavior of mutation, we specify mutation as follows:
when performing mutation, all genes use the mutation rate as the
probability that they are changed to a different value.
%Only if a gene has exactly one valid value\footnote{Genes can have just a single valid
%value in the case of the first node's connection genes in single input problem
%or if there is only a single valid function.}
%will mutation leave the gene unchanged.

The most common form of CGP uses a $1+\lambda$ strategy, where all offspring are
mutants of the parent.  The most fit offspring replaces the parent only if its
fitness is no worse than the parent.  Ties between offspring are broken by random selection
and ties between the offspring and the parent are awarded to the offspring:  in this manner
inactive genes are allowed to drift, which has been shown to significantly improve
performance~\cite{yu:2001:neutrality}.  We will refer to this standard form of
mutation and replacement as \emph{Normal}.

\section{Wasted Evaluations}
Preventing search from repeatedly evaluating the same search point can improve
any search that is limited by the number of evaluations it can perform.
In an evolutionary search employing indirect phenotype encoding, as CGP does,
the same search point can be represented by multiple different genotypes
making duplicate detection more difficult.  While it does not catch all
identical phenotypes, if two CGP individuals are actively identical (contain
the same active genes) then their phenotypes must be identical.
Miller~\cite{miller:2011:chapter2} theorized
that this may lead to wasted evaluations as any number of actively identical
individuals can be evaluated even though they will receive identical fitnesses.  He did not,
however, provide any analysis for how frequently these unnecessary evaluations may occur.

\begin{figure}
  \centering
  \includegraphicsfit{Probability}
  \caption{Probability of no active gene being mutated for different mutation rates} 
  \label{fig:probability}
\end{figure}
\subsection{Detection}
Before evaluating a CGP individual, it is common practice to determine which
nodes are active allowing inactive nodes to be ignored during
evaluation, leading to a reduction in runtime~\cite{vasicek:2012:efficient}.
By storing the set of active genes as part of the individual, it is trivial
to determine when active genes are mutated.  More generally, it takes $O(N)$ additional time to determine
if two randomly chosen individuals contain different active genes, where $N$ is
the number of active nodes, assuming at least one of the individuals has been
evaluated.  

\begin{comment}
To exploit this knowledge to its fullest, all newly created individuals could
be compared against all previously evaluated individuals to determine if they
need to be evaluated.  While this can be optimized by storing a hash table of
actively unique individuals, the likelihood that two unrelated individuals are
actively identical may be too low to warrant the memory overhead required by being
this thorough.
\end{comment}

%A reasonable heuristic to reduce this overhead is to only check if
A reasonable heuristic to avoid duplicate evaluations is to determine if
a parent's active genes are identical to its offspring.  This will catch the
majority of cases, as the probability of two individuals being actively identical
decreases as the number of mutation applications
between them increases.  This also has the advantage that
extra individuals need not be stored in memory.  Due
to these advantages, we only compare an offspring against its parent
to detect unnecessary evaluations.


\subsection{Frequency of Offspring Actively Identical to their Parent}
\label{sec:freq}
Using the definition of point mutation given in Section~\ref{sec:cgp},
it is possible to quantify the probability that an offspring is actively
identical to its parent.
\begin{equation}
  \left (1 - mutation\_rate  \right )^{active\_genes}
\label{eq:waste}
\end{equation}
\begin{comment}
First, the number of active genes in each individual
is given by Equation~\ref{eq:active}.  This shows how the number of active genes
can change greatly with the number of active nodes and the arity used for each node.
\begin{equation}
  active\_genes = active\_nodes\cdot\left(arity + 1 \right) + output\_length
  \label{eq:active}
\end{equation}
\end{comment}
Equation~\ref{eq:waste} shows the probability that none of the active genes are
mutated at the given mutation rate.  To help display the effects of this relationship,
Figure~\ref{fig:probability} illustrates the probability that an offspring is actively
identical to its parent for various mutation rates and number of active genes.
Figure~\ref{fig:probability} shows that as the mutation rate and the number of active genes decreases,
the probability of an offspring being actively identical to a parent greatly increases.
While the other versions of mutation discussed in Section~\ref{sec:cgp} will change the details
of this equation, the overall relationship between the variables will be the same.

\begin{comment}
Using the other versions of mutation in Section~\ref{sec:cgp} does not change the
overall relationship of the number of active genes and the mutation rate to the
frequency of wasted evaluations.  If mutated genes are not forced to change to a
new value the probability of actively identical offspring will increase.  While this is intuitively
straightforward, determining the exact amount of increase is complex as the probability
that a randomly reset gene returns to its current value depends on which gene is being mutated.
If mutation is changed to have a set percentage of the genome mutated, the relationship
changes to use the percentage of the genome that is active, but it will still follow the general
shape of Figure~\ref{fig:probability}.
\end{comment}

\section{Methods to Avoid Wasting Evaluations}
\subsection{Skip}
The most straightforward method to use existing evaluation results is to
set the offspring's fitness to the parent's
if they are actively identical. This method will be referred to as \emph{Skip}.
As \emph{Skip} makes no modification to evolutionary mechanisms, any problem
solvable by \emph{Normal} will be solvable by \emph{Skip}, and vice versa. The
primary difference is only that \emph{Normal} will always use at least as many
evaluations as \emph{Skip}, with the potential to use significantly more.  As
a result, \emph{Skip} may solve problems \emph{Normal} cannot if given a
maximum number of evaluations.

A significant theoretic advantage to this technique is that it lowers CGP's
sensitivity to the mutation rate.  In \emph{Normal}, as the mutation rate
decreases the potential for wasted evaluations
increases creating an evaluation penalty.  As the mutation rate increases the expected
number of modified active genes increases making small
changes in phenotype less probable. From this, \emph{Normal} will likely
be most effective when the mutation rate
is high enough to expect at least one active gene to be modified by
mutation, but low enough to allow the small phenotypic moves required
to reach optimal values.
Unlike \emph{Normal}, \emph{Skip} can use mutation rates expected to produce
frequent offspring that are not actively different from their parents without
penalty.  As such \emph{Skip} can use very fine grained mutation allowing it to potentially
hill climb much more effectively.

\subsection{Accumulating Mutation}
As was stated in Section~\ref{sec:cgp}, some of CGP's success stems
from allowing genetic drift of inactive genes~\cite{yu:2001:neutrality}.
%Inactive gene modification can build structures which can potentially be made active
%by later mutations.
This raises the following question: will evolution in CGP improve
if inactive genes are allowed to change at an increased rate?

To test this possibility we propose a method of mutation called \emph{Accumulate}.
The primary concept behind this technique is to allow inactive genes to drift
multiple generations worth of distance in a single actual generation of CGP.
For our example we will refer to one of the offspring produced by the parent as $F_0$.  If $F_0$
is actively identical to its parent, $F_0$ produces a mutant offspring $F_1$ in the same generation.
This chain continues with $F_i$ mutating to produce $F_{i+1}$ until an individual
is produced that has one or more active genes different from the original parent.
This actively different individual is called $F_n$.  $F_n$ is then evaluated.
If $F_n$ is no worse than the original parent, it replaces $F_0$ and CGP continues
evolution and selection as normal.  If $F_n$ is worse than the original parent
$F_{n-1}$ replaces $F_0$.  Thus accumulate's offspring can have zero to multiple
active genes changed.

By chaining mutations each $F_i$ will probabilistically have more changes to
inactive genes than $F_{i-1}$ when compared with the original parent.  By using
$F_{n-1}$ or $F_n$ instead of $F_0$, \emph{Accumulate} can drift faster than
\emph{Normal} and \emph{Skip}.  This can be advantageous on problems that benefit
from high exploration of the neutral space, with the potential drawback that
previously active useful structures are less likely to survive to be reintegrated.

\subsection{Single Active Mutation}
%An alternative method for avoiding unnecessary evaluations is to
%modify mutation to ensure offspring
%are actively different from their parents. 
\emph{Skip}
and \emph{Accumulate} have the potential, similar to \emph{Normal}, to generate large numbers of individuals
with no active differences, which can be computationally expensive even if they
are not evaluated. To avoid
this potential overhead, we propose a method to ensure a \emph{Single} active
gene is mutated every time an offspring is generated.
%
Creating an offspring using \emph{Single} is an iterative process.  Until an
active gene has been mutated, select a gene at random in the individual and
mutate its value.  This has the following properties:
\begin{itemize}
\item Exactly one active gene is mutated for all offspring.
\item Zero or more inactive genes can be mutated.
\item A gene that is active will be mutated more frequently than one that is not.
\item No mutation rate is required.
\end{itemize}
Many of these properties make \emph{Single} distinct from the other described mutation
methods.  By forcing only a single active gene to change, \emph{Single} may have
better results on problems where incremental improvements are possible with single
gene changes and worse results on problems where larger changes are necessary.
This is in contrast to all of the other methods that have the potential to change
multiple active genes.
Limiting mutation to a single active gene change does not, however, prevent \emph{Single}
from making large phenotypic changes, which are still possible by connecting
in inactive nodes.

\emph{Single} has the property that the effective mutation rate is always
$\frac{1}{a}$ for active genes and $\frac{1}{a + 1}$ for
inactive genes, resulting in $\frac{n - a}{a + 1} + 1$ expected mutations, where
$a$ is the number of active genes and $n$ is the total number of genes.  This
means that the number of inactive genes does not affect the probability of an
inactive gene being mutated, just the expected number of inactive genes that are
mutated.  Furthermore, as the number of active genes approaches zero, the expected
number of mutations increases dramatically, even though it is still limited to
changing a single active gene.  As the evolved number of active genes is likely to be
highly problem dependent, the behavior of \emph{Single} will likely change significantly
from problem to problem, with some problems having very little changes to inactive
genes, and others changing a great deal.  It is also possible that this will create
a selective pressure on the number of active genes which may only become apparent
if multiple parents are used.

In comparison with \emph{Skip}, \emph{Single} is even easier to apply to new problems
as it completely removes the need to set the mutation rate, but does so at the
potential expense of lower maximum effectiveness.  In comparison with \emph{Accumulate},
\emph{Single} mutates inactive genes less often with respect to active genes, making it
better at preserving useful inactive structures but worse at exploring the inactive
space.

\section{Experimental Setup}
In order to empirically test the suggested benefits of \emph{Skip},
\emph{Accumulate}, and \emph{Single} in comparison with \emph{Normal}, we evaluated their performance on
four benchmarks: Parity, Multiplier, Binary Decode and Binary Encode.
Both Parity and Multiplier have been used extensively to test
CGP~\cite{walker:2008:cgpmodules,miller:2006:redundancy}.
The 3-bit even parity is included as it is the most commonly used benchmark for CGP and to
help provide some level of comparability with previous work.  In order to try a
more challenging binary problem, we chose to do the 3-bit multiplier, which has
6 inputs and 6 outputs.

Binary Decode is effectively a demultiplexer, where $N$
input lines specify which of the $2^N$ output lines should have a value of one, with
all others set to zero.  For example, if $101$ is the input to a 3-bit Binary Decode
problem, the output is $00100000$.  Binary Encode is the reverse of Binary Decode.
Given $2^N$ input lines, where exactly one line has a bit set to one and all others
are zero, the binary value index of the one bit is output using $N$ lines.
An example of this mapping is that $00100000$
should return the output of $101$.  The reason we use Binary Decode and Binary Encode
as benchmarks is their asymmetric number of inputs to outputs and their relatively
small number of test cases.  For example, on the 16 to 4 Binary Encode and the 4 to 16 Binary
Decode (the instances we use) only 16 test cases are needed.

All problems used the operators \{and, or, nand, nor\} and determine fitness as the average
percentage of correct bits across all possible inputs.  While this is a common function set used
by CGP for these problems,~\cite{walker:2008:cgpmodules} used a different set of operators
for Multiplier.  Preliminary testing shows our operator set makes Multiplier harder than theirs,
but, as always, conclusions drawn by comparing experiments using different operator sets should
be minimal.

While a few of these benchmarks may be overly simple, with parity especially
being considered for obsolescence~\cite{mcdermott:2012:benchmarks}, they are used here
for the specific reason of their simplicity.  The primary
difference between \emph{Normal} and \emph{Skip}
is not that one can solve harder problems than the other, rather it is that \emph{Skip} is thought
to solve problems in less evaluations than \emph{Normal}.  Being able to compare each method's
median number of evaluations until success makes for the most logical metric in this case,
so these problems are useful in that they can be solved often.  While much of the
previous work on CGP uses Koza's \emph{computational effort} to analyze experimental
results,~\cite{christensen:2002:comp_effort} explains how that can introduce
significant underestimates, especially when using small population sizes.

In order to compare with the effective standard CGP, we use the $1+4$
evolutionary strategy.  We use a
genotype size of 3000 nodes, similar to those found optimal in~\cite{miller:2006:redundancy}.
Each run was allowed up to
10 million evaluations before being terminated, and 50 runs of each experiment
were performed.  To help focus the differences between each mutation method,
all starting individuals on the same problem were identical for the same run number,
meaning that the 50 different individuals \emph{Normal} started with were identical
to the 50 starting individuals for \emph{Skip}, \emph{Accumulate}, and \emph{Single}.
For statistical purposes this means all results should be compared as paired tests.
As the number of evaluations to success is likely not normally distributed, we choose
the Wilcoxon signed-rank test to compare each mutation method with \emph{Normal}.
As an attempt to cover a wide range of mutation rates, the 16 mutation rates listed
in Figure~\ref{fig:probability} were used for each method.  These cover from 0.0001
to 0.8 on a log scale, and should encompass all previous levels of mutation used
with CGP.

\begin{comment}
We had intended to include Koza-1 and Pagie-1 from~\cite{mcdermott:2012:benchmarks}
as sample regression problems, but we had difficulty achieving acceptable success
rates to allow for comparisons here.  Future work may be needed to understand why
these problems were significantly harder when using our configuration settings
than the four binary problems. 
\end{comment}

\section{Results}
To summarize the results of testing each mutation rate, Figures~\ref{fig:parity},
\ref{fig:multiply}, \ref{fig:encode}, and \ref{fig:decode} show the median number of evaluations
required over 50 runs to solve the Parity, Multiply, Binary Encode, and Binary Decode problems.  Note
that both axes are presented on a log scale.  In the event that the median run of a configuration
did not find the optimal value before termination, the value is not displayed in the graph.  This can be
clearly observed in Figure~\ref{fig:multiply} for mutation rates greater than 0.01.  As \emph{Single}
does not use a mutation rate, its data is assumed constant for all mutation rates.
%
Table~\ref{table:results} shows fine details
about the best configurations found for each mutation method.  This includes the mutation rate
with the lowest median evaluations until success and the median evaluations it required, the median absolute
deviation from that median, the single tailed P-value received from comparing with the best \emph{Normal}
configuration using the Wilcoxon signed-rank test, and the median number of active genes in the solution.
Note that the number of active genes is a different measure from the number of active nodes, which has
been reported in other work, and is used here for compatibility with Equation~\ref{eq:waste}.
%
A complete package of our results as well as the source code used in testing
is available from our website.\footnote{\url{https://github.com/brianwgoldman/ReducingWastedEvaluationsCGP}}

\begin{figure}
  \centering
  \includegraphicsfit{parity}
  \caption{Median evaluations to success for each method on the Parity problem} 
  \label{fig:parity}
\end{figure}

\begin{figure}
  \centering
  \includegraphicsfit{multiply}
  \caption{Median evaluations to success for each method on the Multiply problem} 
  \label{fig:multiply}
\end{figure}

\begin{comment}
\begin{figure}
  \centering
  \includegraphicsfit{quartic}
  \caption{Mean Evaluations to Success for each method on the Quartic problem} 
  \label{fig:quartic}
\end{figure}
\end{comment}

\begin{figure}
  \centering
  \includegraphicsfit{encode}
  \caption{Median evaluations to success for each method on the Binary Encode problem} 
  \label{fig:encode}
\end{figure}

\begin{figure}
  \centering
  \includegraphicsfit{decode}
  \caption{Median evaluations to success for each method on the Binary Decode problem} 
  \label{fig:decode}
\end{figure}

\section{Discussion}
Even though the Parity, Multiplier, Binary Encode, and Binary Decode problems have
very different numbers of inputs and outputs, different input to output ratios, 
different evolved numbers of active genes (see Table~\ref{table:results}), and
different levels of difficulty, Figures~\ref{fig:parity}, \ref{fig:multiply},
\ref{fig:encode}, and \ref{fig:decode} present surprisingly similar results.  
\begin{comment}
On all four
problems \emph{Normal} has a fairly clear quadratic shape which at its lowest point
approaches or sometimes crosses the line of \emph{Single}.  As the mutation rate increases
beyond the optimal value for \emph{Normal} all techniques except \emph{Single} share
very similar if not identical progressions.  Below \emph{Normal}'s optimal mutation
rate, \emph{Normal} gets significantly worse while \emph{Skip} and \emph{Accumulate}
diverge from \emph{Normal} by maintaining similar qualities as the optimal value.
At exceptionally low mutation rates \emph{Normal} takes over an order of magnitude
more evaluations to find success than all of the other techniques.

The shape of these features is as expected.
\end{comment}
  The steep increases in median evaluations
for \emph{Normal} corresponds with the prediction that overly high mutation rates will
significantly increase the number of evaluations until success, 
while low mutation rates will create a significant portion of
wasted evaluations.  On Multiply, Binary Encode, and Binary Decode, overly high mutation rates
prevent \emph{Normal} from reaching the optimum.
Parity is the only problem CGP could solve using exceptionally high mutation rates, with even
a slight improvement at the very top of the tested range.  This likely happens as the Parity problem
we are using is so simple that a CGP using effectively random search is able to solve it.
\emph{Skip} and \emph{Accumulate}'s tendency to mirror \emph{Normal}'s effectiveness as the
mutation rate increases is because the probability of creating actively identical offspring
approaches zero.  As the divergence point is always near
\emph{Normal}'s optimum mutation rate, it can be inferred that \emph{Normal}'s significant decrease in
quality when using lower mutation rates is likely caused by wasted evaluations.
As \emph{Skip} also decreases
somewhat in quality, some of the impaired fitness can also likely be attributed
to improbability of making large enough genotypic changes to escape local optima
with small mutation rates.

All four figures suggest that \emph{Skip} and \emph{Accumulate}
have a lower sensitivity to the mutation rate than \emph{Normal}.
As the mutation rate goes below the optimal value,
\emph{Skip} and \emph{Accumulate} encounter a relatively minor increase in their median number of evaluations
when compared with \emph{Normal}. As confirmation of Equation~\ref{eq:waste}
we recorded the predicted number of wasted evaluations and found
the aggregate predicted waste was nearly identical to the differences between \emph{Normal} and \emph{Skip}.

\emph{Single}, the only method without a mutation rate, is clearly not as generally effective as
the other methods where mutation rate can be optimized.  However, when the optimal mutation rate is
not known, \emph{Single} does show significant advantages.  When compared against the entire
space of tested mutation rates, \emph{Single} almost always requires
the least number of evaluations to succeed.  Below the optimum mutation rates for each, \emph{Skip}
and \emph{Accumulate} are reasonable approximations of \emph{Single}, but come with the disadvantage
that they are more computationally expensive.
%
As the number of evaluations required to solve a problem increases, \emph{Single}
appears to improve in performance relative to \emph{Normal}.  Parity required the
least number of evaluations to be solved, followed by Binary Encode and Binary Decode,
with Multiply requiring by far the most.
This ordering mirror's \emph{Single}'s relative scaling with \emph{Normal}, 
as it goes from a lower peak performance than \emph{Normal}
on Parity and Binary Encode to surpassing it on Binary Decode and Multiply.

\begin{table}
	\centering
	\caption{Comparisons of Best Mutation Rate for Each Mutation Method on Each Benchmark.
	  All results are from 100\% successful configurations.}
	\begin{tabular}{r|c|c|c|c|c|c|c|c|c|c|c|}
	  \cline{3-6}
		 \multicolumn{2}{r|}{}    & \textbf{\emph{Normal}} & \textbf{\emph{Skip}} & \textbf{\emph{Accumulate}} & \textbf{\emph{Single}} \\ \cline{2-6} 
		                    & Mutation Rate & 0.02 & 0.02 & 0.01 & N/A \\ \cline{2-6}
	       \textbf{3-bit} & Median Evaluations & 1,018 & 959 & 928 & 1358 \\ \cline{2-6}
        \textbf{Parity} & Median Absolute Deviation & 614 & 564 & 416 & 733 \\ \cline{2-6}
	                      & P value  & N/A & 0.0000 & 0.4875 & 0.0790 \\ \cline{2-6}
	                      & Median Final Active Genes & 250 & 250 & 262 & 223 \\ \hhline{~=====}

		                    & Mutation Rate & 0.004 & 0.002 & 0.0008 & N/A \\ \cline{2-6}
	       \textbf{3-bit} & Median Evaluations & 503,424 & 398,354 & 421,859 & 340,165 \\ \cline{2-6}
	  \textbf{Multiplier} & Median Absolute Deviation & 191,037 & 184,979 & 165,218 & 151,063 \\ \cline{2-6}
	                      & P value  & N/A & 0.0212 & 0.3632 & 0.0065 \\ \cline{2-6}
	                      & Median Final Active Genes & 627 & 629 & 684 & 620 \\ \hhline{~=====}

                        & Mutation Rate & 0.004 & 0.004 & 0.008 & N/A \\ \cline{2-6}
	     \textbf{16 to 4} & Median Evaluations & 24,057 & 19,732 & 21,563 & 26,031 \\ \cline{2-6}
	      \textbf{Encode} & Median Absolute Deviation & 8,829 & 7,425 & 7,449 & 10,170 \\ \cline{2-6}
	                      & P value  & N/A & 0.0000 & 0.0864 & 0.3578 \\ \cline{2-6}
	                      & Median Final Active Genes & 472 & 472 & 499 & 510 \\ \hhline{~=====}

	                      & Mutation Rate & 0.004 & 0.002 & 0.004 & N/A \\ \cline{2-6} \cline{2-6}
       \textbf{4 to 16} & Median Evaluations & 73,351 & 70,023 & 65,347 & 69,564 \\ \cline{2-6}
        \textbf{Decode} & Median Absolute Deviation & 17,889 & 17,036 & 18,411 & 21,131 \\ \cline{2-6}
	                      & P value  & N/A & 0.415 & 0.3276 & 0.2398 \\ \cline{2-6}
	                      & Median Final Active Genes & 1,117 & 1,027 & 1,050 & 1,090 \\ \cline{2-6}

  \end{tabular}
  \label{table:results}
\end{table}

Using Table~\ref{table:results}, it is clear that \emph{Normal} is never the best option.
In all cases \emph{Skip} and \emph{Accumulate} can obtain better results,
with even \emph{Single} doing better on Binary Decode and Multiply.  Furthermore, \emph{Skip} always
had a lower median absolute deviation than \emph{Normal}, with \emph{Accumulate} lower
on all problems except Binary Decode.  The statistical analysis is less clear as the deviations
are so large, but \emph{Skip}'s improvements are definitely significant on all problems except
Binary Decode, with \emph{Single} statistically better than \emph{Normal} on Multiply.
As support that our configuration
for \emph{Normal} is valid to compare against, our peak quality is approximately 6 times
better than results published in~\cite{walker:2008:cgpmodules} for their standard CGP and ECGP
on Parity.
%TODO Discuss length stuff

While the true complexity measure for a GP implementation is how many evaluations
it requires to solve a problem, we also performed preliminary timing tests to ensure
our new techniques were less computationally complex than the evaluations they avoid.
When using the best found mutation rate on Multiply, \emph{Normal} required the most time (4597 minutes)
to complete all 50 runs, with \emph{Skip} using 9\% less time, \emph{Accumulate} 16\% less,
and \emph{Single} 29\% less.

\section{Conclusions and Future Work}
CGP's current mutation operator has the potential to create offspring which can
be identified as phenotypically identical to their parents without invoking
the evaluation function.  Evaluating these offspring creates waste, as we can
assign their fitness to be identical to their parent.  Through careful definition
of how the mutation operator works, we have shown how the probability of having
wasted evaluations is dependent on the mutation rate and the number of active genes.
We predicted and then provided empirical evidence that avoiding this waste can
reduce how many evaluations CGP requires to solve problems as well as reduce
CGP's sensitivity to the mutation rate parameter.

We proposed \emph{Skip} and \emph{Accumulate} as two methods for what to do when
actively identical offspring are detected, and \emph{Single} as a method to ensure
they are never created.  On four fairly diverse benchmark problems we showed that
\emph{Skip} and \emph{Accumulate} do at least as well as \emph{Normal} using a
variety of mutation rates and are frequently vastly more efficient than \emph{Normal}.
As there was little discernible difference in their overall quality, for now we suggest \emph{Skip}
be used as a replacement to \emph{Normal}, as it is less complex than \emph{Accumulate},
although future experimentation may suggest otherwise.  In situations where the mutation
rate cannot be optimized, we suggest \emph{Single}, as while it did not achieve
the best peak performance on all problems, it has the least overhead with the widest applicability.
Furthermore, \emph{Single} achieved the best results on the hardest test problem, suggesting more
extensive testing of this method on difficult problems should be performed.

Even though the results appear to be problem independent, \emph{Skip}, \emph{Accumulate},
and \emph{Single} should be tested on a wider range of problem classes to make sure.
More advanced techniques for handling or avoiding wasted evaluations may also improve
results, such as a modified version of \emph{Single} to allow multiple active genes
to be mutated at once.  Nevertheless, we feel that the theoretic and empirical evidence
sufficiently supports a need to avoid wasting evaluations on actively identical
individuals.

\bibliographystyle{splncs}
\bibliography{../main}
\end{document}
