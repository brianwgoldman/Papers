% This is "sig-alternate.tex" V2.0 May 2012
% This file should be compiled with V2.5 of "sig-alternate.cls" May 2012
%
% This example file demonstrates the use of the 'sig-alternate.cls'
% V2.5 LaTeX2e document class file. It is for those submitting
% articles to ACM Conference Proceedings WHO DO NOT WISH TO
% STRICTLY ADHERE TO THE SIGS (PUBS-BOARD-ENDORSED) STYLE.
% The 'sig-alternate.cls' file will produce a similar-looking,
% albeit, 'tighter' paper resulting in, invariably, fewer pages.
%
% ----------------------------------------------------------------------------------------------------------------
% This .tex file (and associated .cls V2.5) produces:
%       1) The Permission Statement
%       2) The Conference (location) Info information
%       3) The Copyright Line with ACM data
%       4) NO page numbers
%
% as against the acm_proc_article-sp.cls file which
% DOES NOT produce 1) thru' 3) above.
%
% Using 'sig-alternate.cls' you have control, however, from within
% the source .tex file, over both the CopyrightYear
% (defaulted to 200X) and the ACM Copyright Data
% (defaulted to X-XXXXX-XX-X/XX/XX).
% e.g.
% \CopyrightYear{2007} will cause 2007 to appear in the copyright line.
% \crdata{0-12345-67-8/90/12} will cause 0-12345-67-8/90/12 to appear in the copyright line.
%
% ---------------------------------------------------------------------------------------------------------------
% This .tex source is an example which *does* use
% the .bib file (from which the .bbl file % is produced).
% REMEMBER HOWEVER: After having produced the .bbl file,
% and prior to final submission, you *NEED* to 'insert'
% your .bbl file into your source .tex file so as to provide
% ONE 'self-contained' source file.
%
% ================= IF YOU HAVE QUESTIONS =======================
% Questions regarding the SIGS styles, SIGS policies and
% procedures, Conferences etc. should be sent to
% Adrienne Griscti (griscti@acm.org)
%
% Technical questions _only_ to
% Gerald Murray (murray@hq.acm.org)
% ===============================================================
%
% For tracking purposes - this is V2.0 - May 2012

\documentclass{sig-alternate}
%\linespread{2.5}
\usepackage{url}
\usepackage{verbatim}
\newcommand{\includegraphicsfit}[1]
{\includegraphics[width=\columnwidth,height=\textheight,keepaspectratio]{#1}}
\begin{document}
%
% --- Author Metadata here ---
\conferenceinfo{GECCO'13,} {July 6-10, 2013, Philadelphia, Pennsylvania, USA.}
\CopyrightYear{2013}
\crdata{TBA}
\clubpenalty=10000
\widowpenalty = 10000
%\CopyrightYear{2007} % Allows default copyright year (20XX) to be over-ridden - IF NEED BE.
%\crdata{0-12345-67-8/90/01}  % Allows default copyright data (0-89791-88-6/97/05) to be over-ridden - IF NEED BE.
% --- End of Author Metadata ---

\title{Length Bias and Search Limitations in Cartesian Genetic Programming}
\subtitle{[Genetic Programming Track]}
%
% You need the command \numberofauthors to handle the 'placement
% and alignment' of the authors beneath the title.
%
% For aesthetic reasons, we recommend 'three authors at a time'
% i.e. three 'name/affiliation blocks' be placed beneath the title.
%
% NOTE: You are NOT restricted in how many 'rows' of
% "name/affiliations" may appear. We just ask that you restrict
% the number of 'columns' to three.
%
% Because of the available 'opening page real-estate'
% we ask you to refrain from putting more than six authors
% (two rows with three columns) beneath the article title.
% More than six makes the first-page appear very cluttered indeed.
%
% Use the \alignauthor commands to handle the names
% and affiliations for an 'aesthetic maximum' of six authors.
% Add names, affiliations, addresses for
% the seventh etc. author(s) as the argument for the
% \additionalauthors command.
% These 'additional authors' will be output/set for you
% without further effort on your part as the last section in
% the body of your article BEFORE References or any Appendices.

\numberofauthors{2} %
\begin{comment}
\author{
% 1st. author
\alignauthor
Brian W. Goldman\\
       \affaddr{BEACON Center for the Study of Evolution in Action}\\
       \affaddr{Michigan State University, U.S.A.}\\
       \email{brianwgoldman@acm.org}
% 2nd. author
\alignauthor
William F. Punch\\
       \affaddr{BEACON Center for the Study of Evolution in Action}\\
       \affaddr{Michigan State University, U.S.A.}\\
       \email{punch@msu.edu}
}
%\end{comment}
%\begin{comment}
\author{
% 1st. author
\alignauthor
Anonymous\\
       \affaddr{Group}\\
       \affaddr{Organization}\\
       \email{email@site.com}
% 2nd. author
\alignauthor
Anonymous\\
       \affaddr{Group}\\
       \affaddr{Organization}\\
       \email{email@site.com}
}
%\end{comment}

\maketitle
\begin{abstract}
%Understanding how configuration decisions can interact with individual representation and
%evolutionary mechanisms in Cartesian Genetic Programming (CGP) to cause limitations
%on search is vital to proper algorithm application.  
In this paper we examine how Cartesian Genetic Programming's (CGP's) method
for encoding directed acyclic graphs (DAGs) and its mutation operator bias the
effective length of individuals as well as the distribution of inactive nodes
in the genome.  We investigate these biases experimentally using two CGP variants as comparisons: \emph{Reorder},
a method for shuffling node ordering without effecting individual evaluation, and
\emph{DAG}, a method for removing the concept of node position.
Experiments were performed on four problems tailored to highlight potential search
limitations, with further testing on the 3-bit multiplier problem.

Unlike previous work, our experiments show that CGP has an innate parsimony pressure
that makes it very difficult to evolve individuals with a high percentage of active nodes.  This bias is
particularly prevalent as the length of an individual increases.  Furthermore, these problems are compounded by CGP's positional
biases which can make some problems effectively unsolvable.  Both \emph{Reorder}
and \emph{DAG} appear to avoid these problems and outperform \emph{Normal} CGP
on preliminary benchmark testing. Finally, these new techniques require more
reasonable genome sizes than those suggested in current CGP, with some evidence
that solutions are also more terse.

\end{abstract}

% A category with the (minimum) three required fields
\category{Computing Methodologies}{Artificial Intelligence}{Search Methodologies}
%A category including the fourth, optional field follows...
%\category{D.2.8}{Software Engineering}{Metrics}[complexity measures, performance measures]

\terms{Algorithms}

\keywords{Cartesian Genetic Programming, Neutrality, Bloat}
\\
\\
\section{Introduction}
In all black box search, search designers must make assumptions about the properties
of the problem in order to outperform random search.  When these assumptions
match the application, search time can be greatly reduced.  As such, understanding how design decisions
effect search effectiveness or even prevent search from finding solutions is very important.
%
In Cartesian Genetic Programming (CGP) the maximum genome size of an individual
is set at initialization.  While this
requires the maximum genome size to be larger than the problem's solution needs,
the question remains; how will the maximum genome size effect evolution's
speed at finding solutions.  As larger genomes inherently represent larger search
spaces, intuition suggests that using the lowest possible maximum size will have the most efficient search.
However, in CGP a mounting body of evidence suggests the optimal
maximum size is much larger than the actual solutions evolved~\cite{miller:2006:redundancy}.
We examine the cause of this discrepancy by applying CGP and a few
modified versions of CGP to problems which test specific edge cases in representation.
The result is a better understanding of the search limitations of CGP.

\section{CGP}
\label{sec-cgp}
In Cartesian Genetic Programming (CGP) individuals use a fixed length string of integers to
encode directed acyclic graphs (DAGs).  
Each node in the DAG is encoded using multiple genes: one specifying a function and
the rest specifying node input connections.
Graph output is determined by additional genes which can connect to any node or
program input.
To prevent cycles,
nodes may only connect to problem inputs or nodes that precede them in the genome.
Early versions of CGP limited valid connections further through user specified graph
topologies.
While specifying topology is still useful
when evolving circuits with known hardware limitations, the less restricted form
allows CGP to represent all possible
DAGs of size \emph{N}, where \emph{N} is the number of nodes in the genome.
%
The most common evolutionary mechanism in CGP is a $1+\lambda$ population
where each generation the best offspring replaces the parent if it is no less
fit.  CGP almost exclusively uses a form of point mutation to create offspring,
with no form of crossover commonly being employed.  As such, these are the methods
considered in our analysis.
For a more detailed description of CGP, see~\cite{miller:2011:chapter2}.

%\subsection{Graph Size, Bloat, and Inactive Genes}
While CGP imposes a maximum graph size \emph{N}, the minimum
effective graph size is variable, as there can be unused portions
of the genome.  These inactive nodes exist because both output locations and
internal connections can be changed by evolution.  A trivial example is when all outputs
are connected directly to problem inputs the entire genome becomes
inactive.  Thus the active nodes are those nodes used when computing output.
%With all
%of these features combined, it should be clear that CGP is capable of evolving
%variable sized DAGs with a bounded maximum size.

A pernicious problem in the Genetic Programming (GP) community is individual bloat, in which solutions
become larger with time without substantive improvement in quality~\cite{luke:2006:bloat}.
Experimental evidence has so far suggests that CGP is able to avoid
bloat, as the evolved number of active
nodes does not grow during evolution~\cite{miller:2001:bloat} nor does it grow significantly
as the maximum number of nodes is raised~\cite{miller:2006:redundancy}.

%\subsection{Neutral Search}
The existence of inactive genes provides CGP with an explicit form of genetic
introns, allowing the evolutionary process to perform neutral search.
CGP's replacement strategy facilitates neutral space
exploration by ensuring offspring replace their parents if they are no less fit.
This allows the population to drift,
which has been shown to be beneficial~\cite{yu:2001:neutrality}.  Furthermore,
there is evidence that as much as 95\% of the genotype should be inactive
to allow for sufficient neutral search~\cite{miller:2006:redundancy}.

However, this approach to neutral search may have disadvantages.  In response to a previous claim
about neutrality's beneficial effects on needle in a haystack style
problems, Collins~\cite{collins:2006:haystacks} showed that on problems with no search
gradient, efficient random sampling can significantly outperform CGP with neutral search.
Collins argued that the benefits CGP claimed to be receiving through neutral search
may be better attributed to higher maximum genome sizes causing an increase in
the total number of active genes.

\section{Length and Positional Biasing}
\label{sec-length-biasing}
Despite CGP's ability to represent all possible DAGs
of size \emph{N}, there is little evidence that it uniformly
searches the space of DAGs. In fact,~\cite{payne:2009:bias} showed that certain types of structures
are more difficult to evolve than others.  In particular, larger structures may be more difficult to
evolve as they are less common in the search space. 
Tree based GP has similar issues,
with strong evidence that specific types of trees are more difficult to
evolve even though they are common in the search space~\cite{daida3:2003:treebias}.
We suggest that
the mechanics used to initialize and evolve connection genes in CGP
impose significant biases on the structures of the DAGs evolved, with more bias against
larger genotypes than should be expected by just representation frequency.

When initializing and mutating a connection gene, its value is randomly set to
either a preceding node in the genome or to an input.  As was discussed in~\cite{collins:2006:haystacks},
this means the expected location of a new connection is $\frac{i}{2}$ where \emph{i}
is the number of nodes and inputs that precede the mutated node.  If all nodes had
a single connection gene, we would expect random individuals to have a length of $\log_2N$.
Precise calculation of the expected length is made difficult
due to node overlap, but the relationship will remain logarithmic to the total genome size.

The probability that a node is active changes greatly depending on where
in the genome it is encoded.  Consider nodes near the end of the genome.
The highest index node expected to be active is given by Equation~\ref{eq-active-end}.
\begin{equation}
  \left ( N+I \right )\frac{O}{O + 1}
  \label{eq-active-end}
\end{equation}
Here, \emph{N} is the number of nodes in the genome, \emph{I} is the number of input locations,
and \emph{O} is the number of output locations.  This equation is derived from the fact that
if numbers are drawn uniformly at random and then ranked, the mean value for each rank
will be evenly spaced in the range.  As a result, the expected highest output location will be
the same as if \emph{O} locations were placed evenly on the range $N + I$.
On problems with a single
output (such as the commonly used parity problem) this means, on average, nodes
in the second half of the genome are inactive.  
%Problems with
%more outputs will be less effected as Equation~\ref{eq-active-end} approaches $N + I$
%as \emph{O} grows.
Some CGP implementations reduce this problem by
fixing the output locations to the last \emph{O} nodes in the genome, which has the effect
of scaling \emph{O} in Equation~\ref{eq-active-end} by the arity of the nodes in use.  However, this does not
remove the probability for large sections of inactive nodes near the end of the genome.

Conversely, nodes near the beginning of the genome are almost always active.
Equation~\ref{eq-incoming} shows how the expected number of incoming connections
increases as the number of nodes following node \emph{i} increases and the
arity (\emph{a}) used increases.  This relationship is non-linear monotonic, with input locations
and nodes near the inputs likely to have more than one incoming connection while nodes
near the output location will likely have less than one incoming connection.
\begin{equation}
  \sum_{j=i+1}^{N} \frac{a}{j}
  \label{eq-incoming}
\end{equation}
Equation~\ref{eq-incoming} is derived from the fact that all nodes following \emph{i}
in the genome have \emph{a} connections which each have a $\frac{1}{j}$ chance of
connecting to \emph{i}.

The probability that a node is active increases as the number of incoming connections
increases, as each could potentially come from an active node.  This creates a self
reinforcing cycle:  The closer a node is to the inputs the more incoming connections
it can receive from other nodes with lots of incoming connections.
As a result, nodes near the inputs are almost
always active while nodes near the outputs are almost never active.  Furthermore,
the output of nodes near the inputs is likely to be used by multiple active nodes.

Having many active nodes near the start and few near the end has important
implications on CGP's ability to perform neutral search.
If there are no inactive nodes preceding a given
node, it cannot benefit from neutral evolution.  Nodes that have a high percentage
of inactive nodes preceding them will conversely have a high probability of attaching
to inactive nodes, but will likely be inactive themselves.

\begin{figure}
  \centering
  \includegraphicsfit{ThoughtExperiment}
  \caption{Thought experiment illustrating limitations in \emph{Normal} CGP.
           Arrows represent the flow of information.} 
  \label{fig-thought}
\end{figure}

How this positional bias of active genes effects search can be examined
using a thought experiment, shown in Figure~\ref{fig-thought}.  
Each of the three Cases given are of the second best phenotype, in which \emph{Z}
takes its input from \emph{X}, while in the optimal phenotype \emph{Y} is applied
on \emph{X}'s output before input to \emph{Z}.  In Case 1 there are inactive nodes
between \emph{X} and \emph{Z}, so CGP is able to use drift to insert \emph{Y}.
Unfortunately, because active nodes cluster heavily near the start of the genome,
this is unlike to happen if even a moderate percentage of the genome is active.
In Case 2, \emph{X} and \emph{Z} are adjacent nodes, or equivalently all nodes between
\emph{X} and \emph{Z} are active.  CGP can only solve this case by recreating \emph{X}
and/or \emph{Z} somewhere else in the genome using drift, converting this back
to Case 1.  That process is highly unlikely as it involves drift of active genes.
In Case 3, \emph{Y} already exists in the genome, but in the wrong order relative to \emph{Z}.
CGP can solve this by either reinventing \emph{Y} as in Case 1 or moving \emph{Z} as in Case 2.
From a phenotypic standpoint, the first two cases should be equally difficult with the third
being easier, as \emph{Y} already exists.  However, due to CGP's positional biases, Case 2 is much harder than Case 1,
as active nodes must be reinvented before \emph{Y} can be evolved and added between \emph{X} and \emph{Z}.
Furthermore, Case 3 is just as hard as Case 1, as even though \emph{Y} already exists, \emph{Z} cannot connect to it.
Case 2 and 3 become even more difficult if \emph{X}, \emph{Y}, and \emph{Z} are thought of as
collections of nodes used to produce a single output, as recreating these complex structures
increases the difficulty.  They also become more common as interwoven structures in the
genome act the same as adjacent nodes in this experiment.

\section{Avoiding Search Limitations}
\label{sec-modifying-cgp}
In order to experimentally examine how the mechanisms of CGP bias search, we propose two CGP variants.
The first
is \emph{Reorder}, which attempts to mitigate positional bias by performing phenotypically
neutral node shuffling.  The second is \emph{DAG}, which allows nodes to connect
to any other node in the genome as long as no cycle is created.  For clarity, we will
refer to standard CGP as \emph{Normal}.  Note that both variants are capable of
representing the exact same set of graphs as \emph{Normal}.  They only differ
in how graphs can be modified by evolution.

\subsection{Reorder}
\label{sec-reorder}
\emph{Reorder} works identically to normal CGP with a single change:
Before producing offspring, the parent's nodes are shuffled without changing the phenotype of the individual.
That is, we perform neutral search on the parent's node ordering.
This process begins by building bidirectional node dependency sets, such that
each node's input and output sets can be easily accessed.  First the
set of \emph{addable} nodes is created, which initially contains all of the
inputs.  The new node ordering is then determined iteratively using the following
steps:
\begin{enumerate}
  \item Select and remove a node $\alpha$ at random from the \emph{addable} set.
  \item If $\alpha$ is not an input, map it sequentially to the next genome location.
  \item For each node $\beta$ which takes output from $\alpha$:
  \begin{enumerate}
    \item Mark $\beta$'s dependency on $\alpha$ as satisfied.
    \item If $\beta$'s dependencies have all been satisfied, add it to \emph{addable}.
  \end{enumerate}
\end{enumerate}
Iteration ends
when all nodes have been given new locations in the sequence.  At this point, nodes can be moved
into their new locations, with their connections being updated using the ordering
map.  Once reordered, nodes maintain the normal CGP requirement that all nodes
must take their inputs from nodes preceding them in the genome, meaning that
the mutation and evaluation process remains the same.  As all connections and functions are
preserved during this process, there is no change in the result obtained by executing
an individual.  \emph{Reorder} is just a neutral permutation of the node ordering.

Repeating the thought experiment in Figure~\ref{fig-thought} with
\emph{Reorder} shows how it can differ from \emph{Normal}.  Because inactive nodes
can be shuffled between \emph{X} and \emph{Z}, Case 1 and 2 become identical.
Case 3 can be
solved by \emph{Reorder} without recreating \emph{Y}, as it can
be shuffled to precede \emph{Z}.

Without breaking connections, a node can be moved forward such that
it transitively depends on all preceding nodes.  A node can also be moved back until all following nodes
transitively depend on it.  Furthermore, the distance between a node and
its input with the highest amount of transitive dependency will be on average smaller
than in \emph{Normal}.  This is illustrated
when considering graphs with a node arity of one.  In \emph{Normal} and at initialization
of \emph{Reorder}, a node is on average $\frac{i}{2}$ indexes away from where it gets its input and $2i$ indexes
from where its output goes.  In \emph{Reorder}, nodes will on average be equidistant to
their their inputs and outputs.  In this case, the expected distance is related to the total number
of nodes in the dependency chain.  This means that an increased number of active nodes can
occupy higher index locations as they are pushed to the center between their inputs and outputs.

This does not directly translate into \emph{Reorder} being able to create more active graphs,
or mean that \emph{Reorder} is without length bias.  When creating new connections,
the expected location is still $\frac{i}{2}$, which means given enough mutations,
the individual can revert back to all of the expectations for \emph{Normal}.
In general however, \emph{Reorder} will allow nodes on which
many nodes transitively depend to increase how many nodes can connect to them, while simultaneously
allowing nodes with only a few transitive dependencies to gain more dependencies.

A final artificial bias in both \emph{Reorder} and \emph{Normal} is
that all connections in a node will tend to have the same transitive
dependencies.  This is because if node $\beta$ is already transitively dependent
on node $\alpha$, then $\alpha$ is always a valid connection when generating new connections in $\beta$.
This is because all transitive dependencies to a node always precede that node, even
when using \emph{Reorder}.  Nodes that $\beta$ is not already transitively dependent
on have a probability of being unreachable, and therefore are connected to less frequently.

\subsection{DAG}
\label{sec-dag}
The concept behind \emph{DAG} is to relax the requirement that nodes must connect
forward in the genome.
To accomplish this, only two portions of CGP have to be
modified: connection gene mutation and evaluation ordering.
In \emph{Normal}, when a connection gene is mutated, its value is randomly changed
to any input or node index less than the mutating node's index.  In
\emph{DAG}, when a connection gene is chosen to be mutated, new connections are
tried randomly and kept if they do not create a cycle.

At the start of each mutation, a mapping is developed to store which nodes are
transitively dependent on the mutating node, and which are not.  Initially this map stores that
the input indexes are not dependent on the mutating node, and
that the node is dependent on itself.  Nodes are then tested in a random
order for transitive dependence.  This is performed by recursively applying
the following rules:
\begin{itemize}
  \item If the node's dependency is already known, return the stored information.
  \item If this node takes input from a dependent node, it is dependent.
  \item If none of its inputs are dependent, it is not dependent.
\end{itemize}
As information is collected, the map is updated to ensure a node is never checked
for dependency twice.  If the randomly chosen node is not dependent on the mutating
node, it is used as the new connection value. While this algorithm can result in
each mutation needing to traverse a large portion of the graph searching for
dependencies, the growth is at worst linear.  Furthermore, our reason for
introducing \emph{DAG} is to analyze limitations in \emph{Normal}, not necessarily as
a computationally efficient alternative.

To determine node ordering during evaluation, \emph{DAG}
performs a recursive walk of the active nodes starting from the output
locations, recording for each node its bidirectional dependencies on other active nodes.
Next, \emph{DAG} builds the ordering of active nodes similar to
the \emph{Reorder} method in Section~\ref{sec-reorder}, ensuring only active nodes are added
and that they are added only after all of their dependencies are in the list.
Using this ordering, which is only built once for each individual, any number of
input sequences can be evaluated with identical run time as \emph{Normal}.

Removing the forward-only requirement found in \emph{Normal} and \emph{Reorder}
allows \emph{DAG} to avoid positional bias.
This is because any node can connect to any structure
that exists in the graph as long as its not already part of that structure.  
In contrast, \emph{Reorder} can only connect to nodes currently shuffled
in front of the mutating node, while in \emph{Normal} structures may have
to be reinvented to achieve the same connection.  \emph{DAG} can also
insert inactive nodes between active nodes, as there is
no concept of adjacency.

As to individual length, \emph{DAG} has a much higher expected percentage of active nodes
than \emph{Normal} or \emph{Reorder}.  When mutating, a connection can be made to
any non-dependent node and input.  Using an arity of one as an example, the
expected number of active nodes is $\frac{N}{I+1}$ where \emph{I} is the number
of inputs.  Higher arity will increase the active proportion, but overlap makes
precise estimations difficult.  Note that unlike \emph{Normal}, the number of
active nodes in \emph{DAG} scales linearly with the total number of nodes.  Also,
since \emph{DAG} still uses \emph{Normal}'s method of initialization, the actual
individuals evolved may be less active than expected.

While \emph{DAG}'s reduced search biases do not necessarily result in better results
on specific problems, it should generally make \emph{DAG} effective on a wider
range of applications.
Also note that while evolution of complete DAGs has been done previously~\cite{niehaus:2007:ggp},
our \emph{DAG} method mirror's CGP more faithfully and allows for more direct comparison.

\section{Tailored Problems}
\label{sec-tailored}
To test the search biases of CGP and our variants,
we propose a few benchmark problems specifically
tailored to highlight certain types of bias.
%When performing
%experiments with any evolutionary system, it is necessary to specify all configuration
%information in order to improve replicability of results.
The following experimental configuration is used across all experiments reported here.
%In order to simplify
%matters, all experiments reported here use the same configuration.
Keeping with existing CGP literature, we opted for a $1+4$ strategy using
mutation as the only variation operator.  To avoid tuning the mutation
rate, we use the \emph{Single} mutation method described in~\cite{goldman:2013:cgpwaste},
which mutates genes at random until exactly one active gene is changed.  
Each run of each experiment used the same initial individual for all algorithms.
All experiments were replicated 50 times, resulting in 50 different starting individuals per experiment.
Having all algorithms start with the same initialization was done to reduce noise.
Each
run was given 10,000,000 evaluations before being terminated, and runs were stopped
immediately on success.

For all problems except \emph{Neutral} (Section~\ref{sec-neutral}), we provide a plot of the median
evaluations until success using different problem sizes.  If the median run
was unsuccessful, no information is plotted.  Note that both axes in these plots
use a log scale.  In order to ensure a good sample of scalability we used 20
evenly spaced problem sizes $\left [ 5 .. 100 \right ]$ at increments of 5.
This information was then used to calculate each algorithms' order of complexity
in regards to problem size, with results given in Table~\ref{table-growth}.
A complete package of our results as well as the source code used in testing
is available from our website.\footnote{\url{https://name.of.site.removed.com/anonymous-CGP}}

\subsection{Neutral Exploration}
\label{sec-neutral}

\begin{figure}
  \centering
  \includegraphicsfit{neutral}
  \caption{Genome sizes required to achieve different mode
           numbers of active nodes on the \emph{Neutral} problem} 
  \label{fig-neutral}
\end{figure}


\begin{figure}
  \centering
  \includegraphicsfit{length_frequencies_100}
  \caption{Frequency of evolving individuals of different active lengths
           on the \emph{Neutral} problem} 
  \label{fig-length}
\end{figure}

To examine how CGP and our variants bias the active length of individuals
based on the maximum genome size, we first experiment using the \emph{Neutral} problem.
Similar to the problems used in~\cite{miller:2001:bloat,collins:2006:haystacks},
\emph{Neutral} has a completely flat landscape, such that all individuals
evaluate to the same fitness.  In CGP this means parents are replaced each
generation with a randomly chosen offspring. Thus there is no evolutionary pressure, only drift.
This problem highlights what kinds of individuals each variant is biased toward
creating, independent of the evolutionary pressure.
In less complex representations, such as those in binary genetic
algorithms, you would expect all representable
search points to have an equal probability of being seen given enough evaluations.  As CGP's representation
involves complex decoding, \emph{Neutral} can help examine the likelihood of searching
different parts of the search space.  
In our experiments, \emph{Neutral} has a single input and a single output,
with a node arity of two.  Furthermore, the function set used is of size one, meaning
\emph{Single} mutation must change an active node's connection gene or the output location
in order to produce new offspring.

\begin{table}
	\centering
	\begin{tabular}{|r|c|c|c|}
	  \hline
& \textbf{\emph{Normal}} & \textbf{\emph{Reorder}} & \textbf{\emph{DAG}} \\ \hline
    \textbf{Mode} & 15      & 20      & 45 \\ \hline
 \textbf{95\% CI} & [5, 31] & [6, 44] & [10, 67] \\ \hline
\textbf{<99.99\%} & 45      & 61      & 81 \\ \hline
 \textbf{Longest} & 59      & 76      & 92 \\ \hline
	\end{tabular}
	\caption{Frequency of different length individuals on \emph{Neutral}. 
	         Includes the most common length (mode), 95\% confidence interval (CI),
	         length that 99.99\% of individuals were below, and the longest individual found.}
	\label{table-length}
\end{table}


We performed two types of experiments on the \emph{Neutral} problem.  The first
tests how each variant's evolved number of active nodes changes
with genome size.  The results provided in Figure~\ref{fig-neutral} can be used
to predict how large of a genome size should be used
given a user's rough estimate of the number of active nodes required by the problem.  
This also assumes that search will perform best when the most commonly searched length
is close to the optimal individual length.  Figure~\ref{fig-neutral} shows that \emph{Normal}
required the largest genome sizes to achieve each mode number of active nodes.  In
terms of scaling, \emph{Normal} needs a genome size of $O(n^{1.79})$ with
$n$ the desired mode number of active nodes.  While \emph{Reorder} only reduced
this to $O(n^{1.53})$, \emph{DAG} scaled much better at $O(n^{1.18})$.  This implies
that as true problem size increases, we should expect \emph{Normal} to require nearly
quadratically increasing genome sizes, while \emph{DAG} scales nearly linearly.

The second experiment looked at the distribution of evolved lengths using a fixed
genome size of 100. This
information is shown in Figure~\ref{fig-length} as a probability density curve,
with numeric results in Table~\ref{table-length}.
The results for \emph{Normal} match our predictions in Section~\ref{sec-reorder};
most individuals were highly inactive, with a mode number of
active nodes of only 15.  The probability of generating an individual drops quickly
as the number of active nodes deviates from the mode, with none of the 0.5 billion
generated individuals having more than 59 active nodes, and only one in 10,000 above 45.  This is in contrast
to the actual number of graphs of each size, which increases exponentially.
This means that \emph{Normal} is biased toward small graphs even more than
Figure~\ref{fig-length} indicates.
\emph{Reorder} and \emph{DAG} show less of a bias, with \emph{DAG}'s mode of 45
close to the prediction made in Section~\ref{sec-dag}.  The implications of these
biases is that each is most likely to find solutions fastest when the number of
active nodes desired by the problem best matches the algorithms search bias, as
otherwise the algorithm will add artificial pressure to search different sized
graphs.  The steepness of the bias reflects how much pressure will be applied.
Combined with the length scaling results in Figure~\ref{fig-neutral}, we would
predict \emph{Normal} to both require a larger genome size than \emph{Reorder} and \emph{DAG},
as well as search a much smaller subset of the search space.

%TODO Consider returning to this problem after having done \emph{Depth} and
%\emph{Flat}, using the optimal versions of those two problems as the
%starting individual.



\subsection{Breadth}

\begin{table}
	\centering
	\begin{tabular}{|r|c|c|c|}
	  \hline
& \textbf{\emph{Normal}} & \textbf{\emph{Reorder}} & \textbf{\emph{DAG}} \\ \hline
\textbf{\emph{Breadth}} & 3.24 & 2.06 & 2.09 \\ \hline
  \textbf{\emph{Depth}} & 7.72 & 3.66 & 4.17 \\ \hline
   \textbf{\emph{Flat}} & 1.86 & 1.87 & 1.94 \\ \hline
	
	\end{tabular}
	\caption{$O(n^x)$ for each variant on each problem, where $n$ is problem size and $x$ is given in
	         the table.}
	\label{table-growth}
\end{table}


The \emph{Breadth} problem tests CGP's ability to evolve
graphs that connect to all inputs using the minimum possible number of nodes.
The landscape for this test is a problem with \emph{N} nodes which has $N+1$
inputs and a single output.  The optimal structure is a tree with all nodes in the genome
active and all inputs used exactly once as leaves.  To evaluate
an individual, all nodes execute the binary \emph{OR} function,
and possible program inputs are all strings of length $N+1$ with exactly one bit
set.  As a result, the fitness of an individual is equal to the number of inputs
connected to the output.  This problem is similar to the tree graph proposed
in~\cite{payne:2009:bias}, except the problem scales such that all nodes
must always be active, multiple inputs are used,
and the fitness function involves individual execution instead of graph edit distance.

Experimental results for the \emph{Breadth} problem are given in Figure~\ref{fig-breadth}.
This shows that both \emph{Reorder} and \emph{DAG} scale better than \emph{Normal} when solving \emph{Breadth},
but have similar behavior to each other.
From Table~\ref{table-growth} we see that the number of evaluations \emph{Normal} requires
to solve the \emph{Breadth} problem increases cubically while the other variants scale
quadratically.  A likely cause of this growth discrepancy is that the solution to \emph{Breadth} requires
all nodes in the genome to be active, which \emph{Normal} does with difficulty.
The fact that \emph{Reorder} does not share this issue even though its
search bias on \emph{Neutral} was relatively similar implies that \emph{Reorder} is better
able to adapt to problem pressures than \emph{Normal}.  While \emph{DAG} scaled
slightly worse than \emph{Reorder}, the difference is not significant.

\begin{figure}
  \centering
  \includegraphicsfit{breadth}
  \caption{Median evaluations until success for different sizes of the \emph{Breadth} problem} 
  \label{fig-breadth}
\end{figure}

\subsection{Depth}
The \emph{Depth} problem requires CGP to evolve graphs with the maximum possible node
reuse.  In this problem, all nodes use the same function, which takes the minimum of
their inputs, increments it by one, and returns that value.  The fitness function is
equal to the value returned, and each individual is tested on a single input of zero.
As a result, the maximum fitness is achieved when all nodes are connected in a chain
such that all nodes are active and all connections in each node go to the previous node in the graph.
Where \emph{Breadth} required no node reuse, \emph{Depth} requires the maximum amount of node
reuse.  Also in contrast to \emph{Breadth}, \emph{Depth} has a unique solution for
\emph{Normal} and \emph{Reorder}, with \emph{DAG} having multiple ways of representing
the same graph by encoding nodes in different orders.  This problem is similar to the linear graph proposed
in~\cite{payne:2009:bias}, except the problem scales such that all nodes
must always be active and the fitness function involves individual execution instead of graph edit distance.

Figure~\ref{fig-depth} and Table~\ref{table-growth} both show how \emph{Normal}
is effectively unable to solve this problem.  \emph{Normal} was only able to solve
the smallest three problem sizes of the twenty tried, with an experimental growth
rate of $O(n^{7.72})$.  By comparison, \emph{Reorder} and \emph{DAG} solved all
or almost all, respectively, of the problems tried, with far lower growth rates.
As \emph{Breadth} also required all nodes to be active, the reason for \emph{Normal}'s
breakdown must be something else about \emph{Depth}.  A likely explanation is that
\emph{Normal} requires structures to be evolved in a precise location in the genome,
in that a node with all connections set to $i$ is only part of the optimal solution
if that node is at position $i+1$.  If it is not, it must be reinvented at $i+1$ to
move to optimality.  This is especially problematic in the following case.  Consider
a fully optimal individual, except that $i+1$ is inactive and $i+2$ is active and
all of its connections go to $i$.  The only way to improve this individual is to
set all of $i+1$'s connections to $i$ and change all of $i+2$'s connections to $i+1$.
What makes this difficult is the absence of fitness gradient, meaning all changes can
only happen by drift and that changes to $i+2$ can only happen after $i+1$ is relatively
correct.  Otherwise the individual will decrease in fitness.  This problem is further
exacerbated by \emph{Normal}'s mutation method, which on average connects $i$ to $\frac{i}{2}$,
meaning high index nodes are rarely connected to adjacent indexes.
\emph{Reorder} avoids this problem by allowing structures to be moved to where they are
needed, while simultaneously decreasing the length of connections.  \emph{DAG} has
no concept of position, and therefore cannot have this problem.

While \emph{Reorder} and \emph{DAG} have very similar behaviors in Figure~\ref{fig-depth},
Table~\ref{table-growth} shows that \emph{Reorder} scaled slightly better.  A likely
cause of this is that \emph{Reorder} has a higher probability of making nodes
with uniform connections as compared to \emph{DAG}.  This is true
because nodes with no transitive dependency on the mutating node
and can still sometimes be unreachable by mutation in \emph{Reorder}.  Conversely, nodes
that the mutating node is transitively dependent on will always be reachable.  In \emph{DAG}
there is no bias toward connecting to already transitively dependent nodes.
As a result, \emph{Reorder} is able to leverage this
search bias to better make nodes with uniform connection genes, improving search speed.

\begin{figure}
  \centering
  \includegraphicsfit{depth}
  \caption{Median evaluations until success for different sizes of the \emph{Depth} problem} 
  \label{fig-depth}
\end{figure}

\subsection{Flat}
As a counter point to \emph{Breadth} and \emph{Depth}, which require entirely
active genomes, we propose \emph{Flat}.  Unlike the other problems, \emph{Flat}
evaluates an individual's genome directly, with the fitness of
an individual equal to the number of connections made directly to the input.
In this case the optimal individual is one where all nodes and the output location
are connected to the only input location.  This tests how well CGP is able to
create genomes with the least amount of internal structure.
Similar to \emph{Depth}, \emph{Flat} has exactly one solution. 
In~\cite{payne:2009:bias} some evidence was found that solutions
that appear less often in the search space may be harder to find.  If this is the reason
\emph{Depth} was more difficult than \emph{Breadth}, then \emph{Flat} should also
be difficult to solve.  As all nodes now
participate in evaluation regardless of connections, they are all considered active
when performing \emph{Single} mutation.
%Note that a more standard fitness function
%cannot be used as an inverse to \emph{Breadth} and \emph{Depth} as all three methods
%can equally make fully inactive individuals by directly connecting the output
%to the input at a probability of $\frac{1}{N+1}$.

On this problem, all three algorithms have approximately the same overall shape, shown
in Figure~\ref{fig-flat}, and have similar complexities, shown in Table~\ref{table-growth}.
\emph{DAG} does appear to be slower than the other two, although the difference may
just be by a constant scalar.  This makes sense as the probability of a connection
mutating to an input is $\frac{1}{N}$ for \emph{DAG} and $\frac{1}{i}$ for
\emph{Normal} and \emph{Reorder}, with $i\leq N$.  In \emph{DAG}, if there are nodes with
transitive dependencies on the mutating node, then the probability
of connecting to the input increases.  However, it can never surpass that for \emph{Normal}
and \emph{Reorder}.  More generally, these results support the idea that \emph{Reorder}
and \emph{DAG} do not have a harder time making highly unconnected structures
than \emph{Normal}, and that solution rarity in the search space is not why \emph{Normal}
did so poorly on \emph{Depth}.

\begin{figure}
  \centering
  \includegraphicsfit{flat}
  \caption{Median evaluations until success for different sizes of the \emph{Flat} problem} 
  \label{fig-flat}
\end{figure}

\section{Benchmarking}
\label{sec-benchmarking}
Understanding length biases and search limitations allows users to
solve new search problems easier.  In order to examine how our predicted search biases
interact with standard CGP landscapes, we tested all three algorithms on the 3-bit
multiplier problem.  In this problem there are six inputs and six outputs with
a function set of \{AND, OR, NAND, NOR\}.  Instead of scaling problem difficulty
as in Section~\ref{sec-tailored}, each algorithm was tested with scaling genome
sizes.  We used the range $\left [ 10 .. 8000 \right ]$, taking 21 points on an
exponential scale.  Other run parameters were kept the same.
The median number of evaluations to success is shown in
Figure~\ref{fig-multiply} on a log-log scale.  The results from the best genome
size are given in Table~\ref{table-multiply}.

As predicted, both the graphic and tabular results show that \emph{Reorder} and
\emph{DAG} can perform effective search with much smaller genome sizes than \emph{Normal}:
The best tested genome size for both novel
variants had lower median evaluations to success than \emph{Normal}'s best, which
also required a genome twenty times larger.  \emph{Reorder} was statistically
better than \emph{Normal} under the single tailed Mann-Whitney U test. While \emph{DAG} was not
statistically better than \emph{Normal}, the likely reason is \emph{Normal}'s exceptionally large median
absolute deviation.  We believe this deviation is caused by \emph{Normal}'s occasional need to
reinvent useful structures in different genome locations.  In the unlikely event that
active nodes are evolved with the right amount of inactive nodes between them, and in the
right order relative to each other, \emph{Normal} is likely going to receive far better
results.  In cases were these properties do not hold, \emph{Normal} will
have to recreate nodes in new locations, often with no fitness gradient, significantly
increasing run time.  As the number of nodes in the genome increases, the probability
that inactive nodes exist between two active nodes increases.  This may be part of
the reason \emph{Normal} requires such a large genome size to perform well.
Also, more inactive space in the genome may facilitate structure recreation in new locations.
As a final advantage to \emph{Reorder} and \emph{DAG}, we see that both evolved
solutions with half as many median active nodes as \emph{Normal}.  A potential
reason for this is \emph{Normal}'s need to leverage introns in order to perform
neutral search more than the other methods.

\begin{table}
	\centering
	\begin{tabular}{|c|c|c|c|}
	  \hline
& \textbf{\emph{Normal}} & \textbf{\emph{Reorder}} & \textbf{\emph{DAG}} \\ \hline
\textbf{Genome} & 4000   & 200     & 200 \\ \hline
  \textbf{MES} & 272,324 & 213,660 & 249,031 \\ \hline
  \textbf{MAD} & 125,997 & 79,335  & 86,551 \\ \hline
  \textbf{P value} & N/A & 0.030   & 0.117 \\ \hline
  \textbf{Active} & 255  & 118     & 120 \\ \hline

	\end{tabular}
	\caption{Best configuration of each algorithm on 3-bit Multiplier.
	Includes maximum genome size, Median Evaluations to Success (MES), Median Absolute Deviation (MAD)
	one tailed P value comparison with \emph{Normal}, and median number of active
	nodes in the final solutions found.}
	\label{table-multiply}
\end{table}

\emph{Reorder} and \emph{DAG} are not without drawbacks.  In Figure~\ref{fig-multiply} we see that
\emph{Normal} may underperform at peak performance and at lower genome sizes, but
as the genome size increases it surpasses the other methods.  \emph{DAG} is especially
sensitive to increasing genome sizes, performing worse than \emph{Normal} on sizes above 600.
\emph{Reorder} also falls behind \emph{Normal} once sizes exceed 2000.  This
could indicate that the novel methods are more sensitive to genome size.
While this is true, we believe it to be less of a flaw and more of an unavoidable side
effect of these techniques being able to evolve more complex graphs than \emph{Normal}.
%TODO Consider expanding here.

\begin{figure}
  \centering
  \includegraphicsfit{multiply}
  \caption{Median evaluations until success on the Multiply problem using different genome sizes} 
  \label{fig-multiply}
\end{figure}

\section{Conclusions and Future Work}
\label{sec-conclusion}
Existing literature suggests CGP performs best when maximum genome sizes
far exceed actual solution size~\cite{miller:2006:redundancy}, supported by
evidence that inactive genes improve neutral search~\cite{yu:2001:neutrality},
and that CGP doesn't suffer from bloat~\cite{miller:2001:bloat}.  We, however, find evidence
that the need for highly inactive genomes is likely due to limitations in CGP's
search mechanisms.  When no selective pressure on size exists, CGP highly favors
short individuals, with this search bias continuing even when the problem structure
calls for longer individuals.  Furthermore, when combined with CGP's use of node
positioning, problems requiring significant proportions of the genotype to be
active were found to be difficult or impossible for CGP to solve.  We showed theoretical
and experimental evidence that the blame for this problem lies with the way CGP's mutation
interacts with its representation, as existing mechanisms impose artificial search limitations
not inherent to the search problem.  \emph{Reorder}, which shuffled node positions
without changing node expression, and \emph{DAG}, which removed all concept of node
position, were both able to avoid these limitations and more fully utilize
genome space.  As these methods were able to achieve better results with less
inactive genes, we conclude that while inactive genes are useful, the number required
by \emph{Normal} CGP is a result of the inherent length bias and search limitations in its
representation and evolutionary mechanisms, and not because that many inactive
nodes are needed for neutral search.  Similarly, CGP's resistance to bloat is likely
no more than the work of a previously unexamined form of parsimony pressure, the strength
of which is determined by the maximum number of nodes.  The drawback of \emph{Reorder} and
\emph{DAG} is that in removing this parsimony pressure, they are more able to utilize
all genes in the genome, making them more sensitive to the genome size.  Yet this
may prove to be an advantage, as it may allow CGP to solve more complex problems without
explosively increasing the genome size.

Further analysis of \emph{Reorder} and \emph{DAG} should be performed in light
of the evidence found here.  Both appear to be valid ways of making CGP more flexible
and powerful, but wider testing needs to be performed to solidify that claim.  As
an important aspect of those tests, run time comparisons need to be performed, as
both increase the computational expense of CGP, with \emph{DAG} especially having
the potential for significant overhead.  It is also possible either could benefit
from further modification, such as using different initialization or method for mutation.

More analysis on the tailored problems will likely improve the clarity of results.  Specifically
\emph{Neutral} would likely benefit from further scalability testing to see how search frequencies change
with respect to genome size, especially at larger genome sizes.  Tests with \emph{Breadth} and \emph{Depth} to see how many inactive
nodes are needed before \emph{Normal} surpasses \emph{Reorder} and \emph{DAG}, if ever, may
also help highlight exactly what proportion of inactive nodes \emph{Normal} needs to solve
problems.

%
% The following two commands are all you need in the
% initial runs of your .tex file to
% produce the bibliography for the citations in your paper.
\bibliographystyle{abbrv}
\bibliography{../main}  % sigproc.bib is the name of the Bibliography in this case
% You must have a proper ".bib" file
%  and remember to run:
% latex bibtex latex latex
% to resolve all references
%
% ACM needs 'a single self-contained file'!
%
%APPENDICES are optional
\balancecolumns
%\balancecolumns % GM June 2007
% That's all folks!
\end{document}
