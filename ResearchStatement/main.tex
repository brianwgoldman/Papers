\documentclass[a4paper, 11pt]{article}

\usepackage{fancyhdr}
\usepackage[margin=1in]{geometry}
\usepackage{titlesec}
\usepackage{url}

\titleformat{\section}
{\normalfont\Large\bfseries}{}{0pt}{}
\titleformat{\subsection}
{\normalfont\itshape}{}{0pt}{}

%\linespread{2.5}
\begin{document}
\thispagestyle{empty}

\pagestyle{fancy}
\lhead{Brian W. Goldman}
\rhead{Research Statement}

\begin{center}
{\LARGE \bf RESEARCH STATEMENT}\\
\vspace*{0.1cm}
{\normalsize Brian W. Goldman (brianwgoldman@acm.org)}
\end{center}

\noindent
Machine learning (ML) has made great advances in the last few years, and is gaining wide
acceptance as a way to automate expertise at low cost. Yet most of ML's practical application has
been by billion dollar companies to help them sell more products or reduce spending on labor.
I believe machine learning has an untapped potential to help the poorest and most disadvantaged members
of society by improving the effectiveness of underfunded social programs. However, effectively utilizing
ML in these areas will require overcoming challenges not seen in most industrial uses: Sparsity of data,
malicious manipulation of features, and a need to understand what unintentional biases exist in the model.



%NOTES:
%Small grants to prevent homelessness -> reduce overhead http://science.sciencemag.org/content/353/6300/694
%Social Workers overworked -> automatically filter cases
%Automated Legal Help https://www.engadget.com/2016/06/29/ai-laywer-shoots-down-160000-parking-tickets/
%Policing -> risk assessment http://fivethirtyeight.com/features/prison-reform-risk-assessment/

\section{Previous Work}
Many ML and Artificial intelligence algorithms require expert configuration
to achieve high quality performance. This can make it very difficult for
the benefits of these tools to spread beyond academics and well funded
research departments.

The primary focus of my dissertation was the creation of the
Parameter-less Population Pyramid (P3), an optimization algorithm
that requires no problem specific configuration~\cite{goldman:2014:p3,goldman:2015:fastp3,goldman:2016:p3hiff}.
The guiding principle behind its design is to exploit all previous
knowledge the algorithm has about the problem before resorting to further stochastic exploration.
Each iteration, P3 creates a new high quality solution using local search and a model of
problem structure. The model is built by examining the mutual information between variables
in previously found high quality solutions~\cite{goldman:2012:ltga}.
By storing multiple populations of solutions, P3 is able to progressively guide search
to better and better areas without becoming stuck or requiring large initial random search.
P3 is able to find the global optimum in less time than other techniques, as well as
maintain a better ``current best'' solution as optimization progresses, across a diverse set of problems.
This quality is on top
of the fact that previous algorithms used unrealistically ideal configurations while
P3 requires no configuration.

By exploiting additional information on a subset of problems, I was able to further
improve P3's efficiency by two orders of magnitude~\cite{goldman:2015:GBO}.
Working with a collaborator, we were able to develop a generic method using the same principles
that solves many of common problems in the optimization literature in $O(N)$ time~\cite{whitley:2016:mkl}.

\section{Transitioning Focus}
Throughout my Ph.D. I focused on abstract theoretical work. While enjoyable, I began to feel my time would be
better spent focusing on how I could do more than just satisfy my intellectual curiosity.
My postdoc at Colorado State University helped me make this transition, as the topic was treatments for
Post Traumatic Stress Disorder (PTSD).

PTSD is the result of extreme stress causing the body to develop a new
normal, in which the body's hormones are in a constant state of stress response. The illness is resistant
to treatment as natural homeostatic processes reinforce this state. One way to create a lasting improvement
is to disrupt the complex interaction of hormones and return them to a healthy state. To understand how to
do this, our collaborators needed to understand what steady states are possible. By modeling this problem as a search for local optima,
I was able to apply my previous work for understanding optimization landscapes~\cite{goldman:2016:hyperplane}.
Unlike naive enumeration, which would have required decades of computing, my technique was able to find
all steady states in our collaborators' most complete model in seconds.


Recently my interests have centered on deep neural networks (DNNs). As a side project I
investigated new ways of training the output layer of decapitated DNNs, and I am currently in the
process of drafting a publication on my findings. At Google I recently completed the internal course
for ML, and plan to bring a new ML service into production next year.

\section{Future Work}
A recent study in Science~\cite{evans:2016:homelessness} found that
giving a one time payout of about \$1,000 to someone on the brink of homelessness
can keep them out of shelters for two years. Furthermore, this saves an estimated
\$20,000 in welfare, policing, and other expenses. Unfortunately the cost
of ensuring only the needy receive benefits creates almost \$10,000 in overhead.
I believe that the proper application of ML can help improve the effectiveness
of this program by improving fund allocation and automatically filtering applications
to reduce operation expenses, resulting in more money going to those who need it.


Social workers are overworked and underpaid. When
surveyed\footnote{\url{https://www.theguardian.com/society/2010/oct/06/overworked-social-workers-children-risk}},
``95\% admitted that the increased pressure meant children's health and safety was at risk because social workers
were having to make key decisions based on insufficient information.'' ML has repeatedly shown
its ability to detect patterns and make decisions at or above human expert quality. With careful
training, I believe ML could be used to help social workers make better decisions and rescue children
from abusive homes.

Unlike traditional product applications, using ML for social action
requires that we know the model is making decisions for the right reasons.
Much of my work in optimization has been in understanding how complex systems work~\cite{goldman:2015:cgpanalysis,goldman:2016:p3hiff},
and ML will require such analysis to better understand its effectiveness.
Social action ML will also require learning
from comparatively small training sets, which will likely result in new advancements
in how to repurpose existing models from other tasks. However, if we can rise to the challenge
I believe ML can help those who need it most.




%\vspace{0.5cm}
%\newpage

\small

\bibliographystyle{abbrv}
\bibliography{../main}

\end{document}

