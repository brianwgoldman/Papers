\documentclass[a4paper, 11pt]{article}

\usepackage{fancyhdr}
\usepackage[margin=1in]{geometry}
\usepackage{titlesec}

\titleformat{\section}
{\normalfont\Large\bfseries}{}{0pt}{}
\titleformat{\subsection}
{\normalfont\itshape}{}{0pt}{}

%\linespread{2.5}
\begin{document}
\thispagestyle{empty}

\pagestyle{fancy}
\lhead{Brian W. Goldman}
\rhead{Research Statement}

\begin{center}
{\LARGE \bf RESEARCH STATEMENT}\\
\vspace*{0.1cm}
{\normalsize Brian W. Goldman (brianwgoldman@acm.org)}
\end{center}

\noindent
For many problems in many disciplines, it is easier to determine the quality of a solution
than to find the best possible solution. For example, this is true of NP-Hard problems,
finite element analysis, chemical design, and physics simulations.
While coming from very different fields, these problems share
three critical characteristics:
\begin{itemize}
\item There is a defined set of possible solutions.
\item We can measure the quality of a proposed solution.
\item There is no tractable method for finding the best solution.
\end{itemize}
I design algorithms for optimizing problems with these characteristics. Instead of
attempting to create specialized heuristics for every problem,
my goal is to discover and utilize general principles which can
then be applied to a large set of problems without further expert knowledge.

\section{Algorithm Design}
Many generic optimization algorithms require problem specific
configuration. These algorithms are therefore of limited generality
as expert knowledge is required to properly configure them.

The primary focus of my dissertation was the creation of the
Parameter-less Population Pyramid (P3), an optimization algorithm
that requires no problem specific configuration~\cite{goldman:2014:p3,goldman:2015:fastp3}.
P3 is comprised of three parts: Stochastic local search, entropy based model building, and
population layering. The guiding principle behind its design is to exploit all previous
knowledge the algorithm has about the problem before resorting to further stochastic exploration.
Each iteration, P3 creates a new high quality solution using local search and a model of
problem structure. The model is built by examining the mutual information between variables
in previously found high quality solutions~\cite{goldman:2012:ltga}.
By storing multiple populations of solutions, P3 is able to progressively guide search
to better and better areas without becoming stuck or requiring large initial random search.
P3 is able to find the global optimum in less time than other techniques, as well as
maintain a better ``current best'' solution as optimization progresses across a diverse set of problems.
This quality is on top
of the fact that the comparison algorithms used unrealistically ideal configurations while
P3 requires no configuration.

By making slightly more information about the problem available to optimization, I was
able to develop an even more efficient version of P3~\cite{goldman:2015:GBO}. Consider
that many problems can be expressed as the sum of functions which use only a subset
of problem variables. For example, Boolean Satisfiability is often expressed as the sum
of clauses. The Maximum Cut of a graph is equal to the sum of checking each edge.
Using this additional information, I was able to reduce P3's runtime by two
orders of magnitude. I was also able to construct an efficient method
for finding all local optima in an arbitrary problem~\cite{goldman:2016:hyperplane},
and with a collaborator create a generic method which solves many common
problems in the optimization literature in $O(N)$ time~\cite{whitley:2016:mkl}.

\section{Interdisciplinary Applications}
In order to apply these generic optimization methods, I have formed a number of
interdisciplinary collaborations. I first began a collaboration with a mechanical engineer
at NASA. He was in the process of designing a lightweight pallet for carrying sensitive
equipment to other planetary bodies. Using finite element analysis he could measure
the quality of a given pallet, but he had no direct method for discovering the ``best'' pallet.
P3 was able to find a solution which used less mass than the one found by the industry standard
tool, and did so using 40\% of the time.

%TODO Statistical Physics somewhere

Working with a chemist, we set out to
design better OLED molecules, which underly modern electronics displays.
Given a molecule, it is possible to calculate its phosphorescence when excited. However,
there is no direct method for finding molecules which maximize this property.
By applying P3 and other generic optimization methods we found some interesting
molecules for future analysis.

Determining the placement of turbines in a wind farm is another candidate for
generic optimization. Aerospace engineers have developed models which can calculate
how the wake of each turbine can effect the electrical output of the entire field.
While turbines are commonly arranged in grids, we found using optimization methods
that fields with less regular layouts can be significantly more efficient.

A challenging task in machine learning is the training of large scale neural networks.
I have recently been in collaboration with an evolutionary biologist
who studies natural learning to develop better methods for training neural networks.

\section{Future Work}
In recently submitted work~\cite{whitley:2016:mkl}, we determined that many real-world
instances of common NP-Hard problems contained a surprising amount of regularity.
I believe this regularity can be exploited to significantly reduce the space
which needs to be searched for high quality solutions. To that end, I plan to develop
a problem independent technique for search space compression.

A current limitation of P3 is the focus on single objective optimization. For many
problems, there are many competing objectives which should be optimized. For example,
strength, weight, and cost in an engineering task. Optimization gains significant
benefits from treating this problem as having three separate goals instead of attempting
to combine them into a single objective. I believe that integrating P3 with the concepts
of multi-objective optimization will result in a powerful tool for real-world optimization.

Finally, and most critically, I plan to continue exploring interdisciplinary collaborations.
As a tool for problem solving, optimization algorithms are only as useful as the problems they
can solve. Some specific targets for future collaborations include better battery design,
structural engineering tasks, and land use management.



%\vspace{0.5cm}
%\newpage

\small

\bibliographystyle{abbrv}
\bibliography{../main}

\end{document}

