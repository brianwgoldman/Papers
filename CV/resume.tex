%TODO Add: Voted Leader of the year
%______________________________________________________________________________________________________________________
% @brief    LaTeX2e Resume based on the example from Kamil K Wojcicki
\documentclass[margin,line]{resume}
\usepackage{verbatim}

%______________________________________________________________________________________________________________________
\begin{document}
\name{\Large Brian W. Goldman}
\begin{resume}

    %__________________________________________________________________________________________________________________
    % Contact Information
    \section{\mysidestyle Contact\\Information}

    6195 Gossard                            \hfill phone: (314) 313-1281          \vspace{0mm}\\\vspace{0mm}%
    East Lansing, Michigan, 48823                          \hfill email: brianwgoldman@acm.org          %\vspace{0mm}%\\\vspace{0mm}%
    %__________________________________________________________________________________________________________________
    % Research Interests
    \section{\mysidestyle Research\\Interests}
    Evolutionary Computation, Artificial Intelligence, Machine Learning

    %__________________________________________________________________________________________________________________
    % Education
    \section{\mysidestyle Education}
    \textbf{Michigan State University}, East Lansing, Michigan \vspace{2mm}\\\vspace{1mm}%
    \textsl{Doctor of Philosophy in Computer Science \& Engineering} \hfill \textbf{ August 2012 -- Present}\vspace{-3mm}\\\vspace{-1mm}%
    \begin{list2}
        \item Thesis Advisor:  Dr. William F. Punch
        \item GPA: 4.0/4.0
        \item ABD, Expected Graduation: June 2015
    \end{list2}\vspace{-1.5mm}
    \textbf{Missouri S\&T}, Rolla, Missouri \vspace{2mm}\\\vspace{1mm}%
    \textsl{Master of Science in Computer Science} \hfill \textbf{ August 2010 -- May 2012}\vspace{-3mm}\\\vspace{-1mm}%
    \begin{list2}
        \item Thesis Advisor:  Dr. Daniel R. Tauritz
        \item GPA: 4.0/4.0
        \item Summa Cum Laude
    \end{list2}\vspace{-1.5mm}
    \textsl{Bachelor of Science in Computer Science} \hfill \textbf{ August 2006 -- July 2010}\vspace{-3mm}\\\vspace{-1mm}%
    \begin{list2}
        \item Minor in Mathematics
        \item GPA: 3.84/4.0 in major, 3.81/4.0 overall
        \item Summa Cum Laude
    \end{list2}\vspace{-1.5mm}



    %__________________________________________________________________________________________________________________
    % Research
    \section{\mysidestyle Research\\Experience}
    %----------------------P3-------------------
    \textsl{Parameter-less Population Pyramid} \hfill \textbf{August 2013 -- Present}\\
    Combined efficient local search with statistical linkage learning to create a
    parameter-less black box optimization algorithm which outperforms optimally tuned
    state-of-the-art competitors. Presented at GECCO 2014, winning Best Paper in Genetic
    Algorithms Track. Submitted to Evolutionary Computation Journal.
    
    \vspace{-2mm}
    B. W. Goldman and W. F. Punch,
    ``Parameter-less Population Pyramid,''
    \textsl{In Proceedings of the 16th Annual Conference on Genetic and Evolutionary Computation}, pp. 785--792, 2014
        
    %----------------------CGP------------------
    \textsl{Cartesian Genetic Programming} \hfill \textbf{August 2012 -- August 2013}\\
    Examined how representation and evolutionary operators interacted to cause degenerate
    optimization behavior, leading to the proposal of improvements.  Presented at EuroGP 2013
    in Vienna Austria, GECCO 2013 in Amsterdam, The Netherlands.
    
    \vspace{-2mm}
    B. W. Goldman and W. F. Punch,
    ``Analysis of Cartesian Genetic Programming's Evolutionary Mechanisms,''
    \textsl{IEEE Transactions on Evolutionary Computation}, In Press
    
    \vspace{-2mm}
    B. W. Goldman and W. F. Punch,
    ``Length Bias and Search Limitations in Cartesian Genetic Programming,''
    \textsl{In Proceedings of the 15th Annual Conference on Genetic and Evolutionary Computation}, pp. 933--940, 2013

    \vspace{-2mm}
    B. W. Goldman and W. F. Punch,
    ``Reducing Wasted Evaluations in  Cartesian Genetic Programming,''
    \textsl{In Proceedings of the 16th European Conference on Genetic Programming}, pp. 61--72, 2012

    %--------------------Avida---------------------------------------
    \textsl{Artificial Life} \hfill \textbf{March 2013 -- March 2014}\\
    Used Avida, a widely acclaimed platform for digital evolution, to study the evolution
    of prey mimicry of poisonous species when under predation.

    \vspace{-2mm}
    K. D. S. Lehmann,  B. W. Goldman, I. Dworkin, D. M. Bryson, A. P. Wagner,
    ``From Cues to Signals: Evolution of Interspecific Communication via Aposematism and Mimicry in a Predator-Prey System,''
    \textsl{PloS one}, vol. 9, no. 3, e91783, 2014.    

    \pagebreak
    
    %---------------------Benchmarks--------------------------------
    \textsl{Improving Genetic Programming Research} \hfill \textbf{August 2012 -- December 2012}\\
    Utilizing a survey of genetic programming researches and in honor of the fields twentieth anniversary,
    proposed consensus suggestions for how to raise the quality of future publications.

    \vspace{-2mm}
    D. R. White, J. McDermott, M. Castelli, L. Manzoni, B. W. Goldman,
    G. Kronberger, W. Jaskowski, U.-M. O’Reilly, and S. Luke,
    ``Better GP Benchmarks: Community Survey Results and Proposals,''
    \textsl{Genetic Programming and Evolvable Machines}, vol. 14, no. 1, pp. 3--29, 2013.    

    %----------------------LTGA-------------------------------
    \textsl{Linkage Tree Genetic Algorithm} \hfill \textbf{January 2012 -- May 2012}\\
    Performed a deep analysis of the multiple variations of this recently created automatic linkage
    discovery technique, with suggestions for improved methods of testing. Presented at GECCO 2012 in Philadelphia, Pennsylvania.

    \vspace{-2mm}
    B. W. Goldman and D. R. Tauritz,
    ``Linkage tree genetic algorithms: variants and analysis,''
    \textsl{In Proceedings of the 14th Annual Conference on Genetic and Evolutionary Computation}, pp. 625--632, 2012

    \textsl{Supportive Coevolution} \hfill \textbf{January 2011 -- May 2012}\\
    A new form of coevolution in which the primary species directly encodes prospective problem solutions,
    and is supported by other species tasked with dynamically evolving better ways to evolve the primary species.
    Presented at GECCO 2012 in Philadelphia, Pennsylvania.
    
    \vspace{-2mm}
    B. W. Goldman and D. R. Tauritz,
    ``Supportive Coevolution,''
    \textsl{In Proceedings of the 14th Annual Conference Companion on Genetic and Evolutionary Computation}, pp. 59--66, 2012.

    \textsl{Self-Configuring Crossover} \hfill \textbf{January 2010 -- June 2011}\\
    Dynamically evolving novel crossover operators during evolutionary algorithm optimization 
    using linear genetic programming and self-adaptation.  Presented at GECCO 2011 in Dublin, Ireland.
    
    \vspace{-2mm}
    B. W. Goldman and D. R. Tauritz,
    ``Self-Configuring Crossover,''
    \textsl{In Proceedings of the 13th Annual Conference Companion on Genetic and Evolutionary Computation}, pp. 575--582, 2011.

    \textsl{Dynamic Parameters} \hfill \textbf{January 2010 -- February 2011}\\
    Dynamically changing evolutionary algorithm parameters during run time can achieve better solution quality
    than optimally set static parameters for a wide variety of parameters.
    Presented at GECCO 2011 in Dublin, Ireland.

    \vspace{-2mm}    
    B. W. Goldman and D. R. Tauritz,
    ``Meta-Evolved Empirical Evidence of the Effectiveness of Dynamic Parameters'',
    \textsl{In Proceedings of the 13th Annual Conference Companion on Genetic and Evolutionary Computation}, pp. 155--156, 2011.
    
    %__________________________________________________________________________________________________________________
    % Mentoring Experience
    \section{\mysidestyle Mentoring\\Experience}

    \textsl{Extending Self-Configuration} \hfill \textbf{August 2012 -- May 2014}\\
    Mentoring undergraduate student research project investigating ways to improve evolutionary
    self configuration and expanding to new operators.
    
    \vspace{-2mm}
    N. R. Kamrath, B. W. Goldman, and D. R. Tauritz,
    ``Using Supportive Coevolution to Evolve Self-Configuring Crossover,''
    \textsl{In Proceedings of the 15th Annual Conference Companion on Genetic and Evolutionary Computation}, pp. 1489--1496, 2013.

    %__________________________________________________________________________________________________________________
    % Teaching Experience
    \section{\mysidestyle Selected\\Teaching\\Experience}
    \textbf{Michigan State University}, East Lansing, Michigan \vspace{2mm}\\\vspace{1mm}%
    \textsl{CSE232: Introduction to Programming II} \hfill \textbf{May 2014 -- July 2014}\\
    Lead instructor for course, in charge of lecturing, creating course content, and assessments.
    Topics include the C++ programming language up to dynamic memory, templating, and user created data structures.

    \textsl{CSE232: Introduction to Programming II} \hfill \textbf{August 2013 -- May 2014}\\
    Primary instructor for lab section,
    topics include the C++ programming language up to dynamic memory, templating, and user created data structures.

    \pagebreak

    \textbf{Missouri S\&T}, Rolla, Missouri \vspace{2mm}\\\vspace{1mm}%
    \begin{comment}
    \textsl{CS328: Object Oriented Numerical Methods} \hfill \textbf{January 2012 -- present}\\
    Grader for intensive C++ course,
    topics include advanced programming techniques and how to efficiently solve large systems of equations.
    \end{comment}
    \textsl{CS54: Introduction to C++ Lab} \hfill \textbf{January 2011 -- December 2011}\\
    Primary instructor for introductory computer science course,
    topics include programming basics up to inheritance, style, editors, and how to interact with Unix.
    \begin{comment}
    \textsl{CS387: Parallel Computing} \hfill \textbf{January 2011 -- May 2011}\\
    Teaching assistant for parallel computation course,
    topics include MPI, cluster computing, GPUs, Monte Carlo experiments, and high performance computing.
    \end{comment}
    
    \textsl{CS348: Evolutionary Computation} \hfill \textbf{August 2010 -- December 2010}\\
    Teaching assistant for evolutionary computation course,
    topics include genetic algorithms, evolutionary strategies, genetic programming, and learning classifier systems.

    %__________________________________________________________________________________________________________________
    % Professional Experience
    \section{\mysidestyle Recent\\Professional\\Experience}

    \textbf{Los Alamos National Laboratories}, Los Alamos, New Mexico \vspace{2mm}\\\vspace{1mm}%
    \textsl{Advanced Computing Solutions Program} \hfill \textbf{May 2012 -- August 2012}\\
    Investigated intrusion detection techniques using automated network behavior analysis.
    
    \textbf{Sandia National Laboratories}, Albuquerque, New Mexico \vspace{2mm}\\\vspace{1mm}%
    \textsl{Center for Cyber Defenders Graduate Student Intern} \hfill \textbf{May 2011 -- August 2011}\\
    Developed covert command and control communication methods, created security audit tools for a
    surveillance camera system, and designed an educational game to teach secure system design.
    
    \textbf{Dynetics}, Huntsville, Alabama \vspace{2mm}\\\vspace{1mm}%
    \textsl{Computer Analyst 3} \hfill \textbf{May 2010 -- August 2010}\\
    Worked with department of defense applications to improve unmanned aerial vehicle data analysis. 

    \begin{comment}
    \textbf{AT\&T}, Saint Louis, Missouri \vspace{2mm}\\\vspace{1mm}%
    \textsl{Summer Technical Intern} \hfill \textbf{May 2008 -- August 2008}\\
    Developed automated service detection to ensure all pertinent applications installed on a server
    were restarted at server reboot.  Modernized legacy software performing phone number database queries.
    \end{comment}
    
    %__________________________________________________________________________________________________________________
    % Honors and Awards
    \section{\mysidestyle Honors and\\Activities}
    ``Best Paper - Genentic Algorithms Track'' for GECCO 2014 ``Parameter-less Population Pyramid''\\% 
    Program Committee Member, GECCO 2012--present Genetic Algorithms Track \vspace{1mm}\\%
    Voted 2012 Leader of the Year, Missouri S\&T Computer Science Department \vspace{1mm}\\%
    %Chair of ACM SIG for AI competition game development (SIG-Game), 2011--present \vspace{1mm}\\%
    Google AI challenge, ranked 25$^{th}$ out of 4619 world wide, 6$^{th}$ in USA, 2010 \vspace{1mm}\\%
    Missouri S\&T Human versus Computer chess tournament, two first place AIs, 2009 and 2011 \vspace{1mm}\\%
\begin{comment}    
    ACM SIG-Game Competition testing and balancing using evolving AI, 2009--present \vspace{1mm}\\%
    ACM SIG-Game first place AI, 2008          \vspace{1mm}\\%
    University of Illinois at Urbana-Champaign MechMania AI competition second place AI, 2007  \vspace{1mm}\\%
    Missouri S\&T Alumni Scholarship 2006--2010             %\vspace{1mm}\\%

    %__________________________________________________________________________________________________________________
    % Computer Skills
    \section{\mysidestyle Programming} 
    Python, C, C++, C\#, Matlab, Linux shell scripting, \LaTeX, Java, CUDA.

%______________________________________________________________________________________________________________________
\section{\mysidestyle References} 

\begin{tabular}{@{}p{6cm}p{6cm}}
\textbf{Dr. Daniel R. Tauritz} (Advisor)       &  \textbf{Professor Fikret Ercal}                   \\
Associate Professor                &  Professor                       \\
Department of Computer Science     & Department of Computer Science \\
Missouri S\&T                        &  Missouri S\&T                      \\
Rolla, Missouri                      &  Rolla, Missouri        \\
phone: \textsl{(573) 341-7218}    &  phone: \textsl{(573) 341-4857}     \\
e-mail: \textsl{tauritzd@mst.edu}   &  e-mail: \textsl{ercal@mst.edu}    \\
\end{tabular}

\begin{tabular}{@{}p{6cm}p{6cm}}
\textbf{Mr. Dan Thomsen}                \\%&  \textbf{Mr Sean Loye}                    \\
Senior Scientist                        \\      %&  Systems Engineer                         \\
Sandia National Laboratories            \\      %&  Hewlett Packard                          \\
Albuquerque, New Mexico                 \\ %&  Milton, Queensland, Australia            \\
phone: \textsl{(612) 789-0559}    \\%&  phone: \textsl{available on request}     \\
e-mail: \textsl{available on request}   \\%&  e-mail: \textsl{available on request}    \\
\end{tabular}

\end{comment}
%______________________________________________________________________________________________________________________
\end{resume}
\end{document}


%______________________________________________________________________________________________________________________
% EOF

