%%%%%%%%%%%%%%%%%%%%%%%%%%%%%%%%%%%%%%%%%
% Plain Cover Letter
% LaTeX Template
% Version 1.0 (28/5/13)
%
% This template has been downloaded from:
% http://www.LaTeXTemplates.com
%
% Original author:
% Rensselaer Polytechnic Institute 
% http://www.rpi.edu/dept/arc/training/latex/resumes/
%
% License:
% CC BY-NC-SA 3.0 (http://creativecommons.org/licenses/by-nc-sa/3.0/)
%
%%%%%%%%%%%%%%%%%%%%%%%%%%%%%%%%%%%%%%%%%

%----------------------------------------------------------------------------------------
%	PACKAGES AND OTHER DOCUMENT CONFIGURATIONS
%----------------------------------------------------------------------------------------

\documentclass[11pt]{letter} % Default font size of the document, change to 10pt to fit more text

\usepackage{newcent} % Default font is the New Century Schoolbook PostScript font 
%\usepackage{helvet} % Uncomment this (while commenting the above line) to use the Helvetica font

% Margins
\topmargin=-1in % Moves the top of the document 1 inch above the default
\textheight=8.5in % Total height of the text on the page before text goes on to the next page, this can be increased in a longer letter
\oddsidemargin=-10pt % Position of the left margin, can be negative or positive if you want more or less room
\textwidth=6.5in % Total width of the text, increase this if the left margin was decreased and vice-versa

\let\raggedleft\raggedright % Pushes the date (at the top) to the left, comment this line to have the date on the right

\begin{document}

%----------------------------------------------------------------------------------------
%	ADDRESSEE SECTION
%----------------------------------------------------------------------------------------

\begin{letter}{
Josh Bongard \\
Morphology, Evolution \& Cognition Laboratory \\
University of Vermont} 

%----------------------------------------------------------------------------------------
%	YOUR NAME & ADDRESS SECTION
%----------------------------------------------------------------------------------------

\begin{center}
\large\bf Brian W. Goldman \\ % Your name
%\vspace{20pt} \hrule height 1pt % If you would like a horizontal line separating the name from the address, uncomment the line to the left of this text
brianwgoldman@acm.org \\ (314) 313-1281 \\ Department of Computer Science and Engineering \\ Michigan State University % Your address and phone number
\end{center} 
\vfill

\signature{Brian W. Goldman} % Your name for the signature at the bottom

%----------------------------------------------------------------------------------------
%	LETTER CONTENT SECTION
%----------------------------------------------------------------------------------------

\opening{Dear Dr. Bongard,} 
 
I am writing to apply for the Postdoctoral Associate position in your laboratory.
I am currently an all but dissertation Ph.D. student in BEACON and the Department of Computer Science and Engineering at Michigan State University,
and my defense is scheduled for July 1st, 2015. My primary research is in evolutionary computation,
focusing on advanced optimization methods.
In pursuing my research I have produced four journal articles and seven peer reviewed conference papers,
the most recent of which winning Best Paper in the Genetic Algorithms track of GECCO 2014.
%My publications have already received some attention from the field, with 54 citations and an h-index of 4.

My work in Genetic Programming (GP) has focused on how to improve the quality of evolutionary search.
Working with Cartesian Genetic Programming (CGP), a variant of GP which allows for
directed acyclic graphs and multiple program outputs, I discovered flaws in how the genetic operators interacted
with the genetic architecture. This lead me to propose modified operators which significantly improved search
efficiency and solution compactness.

An issue facing the field of GP is its reliance on simplistic and outdated benchmark problems. As part of a multi-institutional
collaboration I helped propose new tools for comparing proposed algorithms. By testing GP methods on more challenging
and more diverse problems we hope to direct the field toward even more success in tackling real-world applications.

More recently I have worked to develop the Parameter-less Population Pyramid (P3) boolean optimization algorithm. P3
utilizes advanced statistical modeling techniques to automatically detect epistatic relationships between the problem variables,
allowing for highly efficient crossover. In combination with local search and a novel population layering strategy, P3 is able
to outperform current state-of-the-art, perfectly tuned, optimization algorithms without receiving any problem specific configuration.
This work won Best Paper in the Genetic Algorithm's track of GECCO 2014, and lead to a collaboration with NASA mechanical engineers
to perform finite element design.

Through graduate coursework I have experience with many Machine Learning tools, such as Bayesian classifiers,
Support Vector Machines, Neural Networks, Random Forests, and Particle Swarm Optimization.
I received perfect grades in Pattern Recognition, Data Mining, Evolutionary Computation, Computational Intelligence,
and Artificial Intelligence.

I am very interested in the opportunity to apply my knowledge to the critically important domain of climate change offered by your
position. I believe this represents an excellent opportunity for new research
in advanced modeling which will in turn be applicable to a broad range of applications.

\closing{Thank you for your consideration,}


%\encl{Curriculum vitae, employment form} % List your enclosed documents here, comment this out to get rid of the "encl:"

%----------------------------------------------------------------------------------------

\end{letter}

\end{document}
