% This is "sig-alternate.tex" V2.0 May 2012
% This file should be compiled with V2.5 of "sig-alternate.cls" May 2012
%
% This example file demonstrates the use of the 'sig-alternate.cls'
% V2.5 LaTeX2e document class file. It is for those submitting
% articles to ACM Conference Proceedings WHO DO NOT WISH TO
% STRICTLY ADHERE TO THE SIGS (PUBS-BOARD-ENDORSED) STYLE.
% The 'sig-alternate.cls' file will produce a similar-looking,
% albeit, 'tighter' paper resulting in, invariably, fewer pages.
%
% ----------------------------------------------------------------------------------------------------------------
% This .tex file (and associated .cls V2.5) produces:
%       1) The Permission Statement
%       2) The Conference (location) Info information
%       3) The Copyright Line with ACM data
%       4) NO page numbers
%
% as against the acm_proc_article-sp.cls file which
% DOES NOT produce 1) thru' 3) above.
%
% Using 'sig-alternate.cls' you have control, however, from within
% the source .tex file, over both the CopyrightYear
% (defaulted to 200X) and the ACM Copyright Data
% (defaulted to X-XXXXX-XX-X/XX/XX).
% e.g.
% \CopyrightYear{2007} will cause 2007 to appear in the copyright line.
% \crdata{0-12345-67-8/90/12} will cause 0-12345-67-8/90/12 to appear in the copyright line.
%
% ---------------------------------------------------------------------------------------------------------------
% This .tex source is an example which *does* use
% the .bib file (from which the .bbl file % is produced).
% REMEMBER HOWEVER: After having produced the .bbl file,
% and prior to final submission, you *NEED* to 'insert'
% your .bbl file into your source .tex file so as to provide
% ONE 'self-contained' source file.
%
% ================= IF YOU HAVE QUESTIONS =======================
% Questions regarding the SIGS styles, SIGS policies and
% procedures, Conferences etc. should be sent to
% Adrienne Griscti (griscti@acm.org)
%
% Technical questions _only_ to
% Gerald Murray (murray@hq.acm.org)
% ===============================================================
%
% For tracking purposes - this is V2.0 - May 2012

\documentclass{sig-alternate}
%\linespread{2.5}
\usepackage{url}
\usepackage{verbatim}
\newcommand{\includegraphicsfit}[1]
{\includegraphics[width=\columnwidth,height=\textheight,keepaspectratio]{#1}}

\usepackage{bm}
\begin{document}
%
% --- Author Metadata here ---
\conferenceinfo{GECCO'14,} {July 12--16, 2014, Vancouver, BC, Canada.}
\CopyrightYear{2014}
\crdata{978-1-4503-1963-8/13/07}
\clubpenalty=10000
\widowpenalty = 10000
%\CopyrightYear{2007} % Allows default copyright year (20XX) to be over-ridden - IF NEED BE.
%\crdata{0-12345-67-8/90/01}  % Allows default copyright data (0-89791-88-6/97/05) to be over-ridden - IF NEED BE.
% --- End of Author Metadata ---

\title{Parameterless Population Pyramid}
\subtitle{[Genetic Algorithms Track]}
%
% You need the command \numberofauthors to handle the 'placement
% and alignment' of the authors beneath the title.
%
% For aesthetic reasons, we recommend 'three authors at a time'
% i.e. three 'name/affiliation blocks' be placed beneath the title.
%
% NOTE: You are NOT restricted in how many 'rows' of
% "name/affiliations" may appear. We just ask that you restrict
% the number of 'columns' to three.
%
% Because of the available 'opening page real-estate'
% we ask you to refrain from putting more than six authors
% (two rows with three columns) beneath the article title.
% More than six makes the first-page appear very cluttered indeed.
%
% Use the \alignauthor commands to handle the names
% and affiliations for an 'aesthetic maximum' of six authors.
% Add names, affiliations, addresses for
% the seventh etc. author(s) as the argument for the
% \additionalauthors command.
% These 'additional authors' will be output/set for you
% without further effort on your part as the last section in
% the body of your article BEFORE References or any Appendices.

\numberofauthors{2} %
\begin{comment}
\author{
% 1st. author
\alignauthor
Brian W. Goldman\\
       \affaddr{BEACON Center for the Study of Evolution in Action}\\
       \affaddr{Michigan State University, U.S.A.}\\
       \email{brianwgoldman@acm.org}
% 2nd. author
\alignauthor
William F. Punch\\
       \affaddr{BEACON Center for the Study of Evolution in Action}\\
       \affaddr{Michigan State University, U.S.A.}\\
       \email{punch@msu.edu}
}
%\end{comment}
%\begin{comment}
\author{
% 1st. author
\alignauthor
Anonymous\\
       \affaddr{Group}\\
       \affaddr{Organization}\\
       \email{email@site.com}
% 2nd. author
\alignauthor
Anonymous\\
       \affaddr{Group}\\
       \affaddr{Organization}\\
       \email{email@site.com}
}
%\end{comment}

\maketitle
\begin{abstract}
TODO
\end{abstract}

% A category with the (minimum) three required fields
%\category{Computing Methodologies}{Artificial Intelligence}{Search Methodologies}
\category{TODO}{TODO}{TODO}
%A category including the fourth, optional field follows...
%\category{D.2.8}{Software Engineering}{Metrics}[complexity measures, performance measures]

\terms{Algorithms}

\keywords{Linkage Learning, Local Search, TODO}

\section{Introduction}
TODO Parameterless optimization, model building, etc.

TODO Talk about ALPS~\cite{hornby:2006:alps}.

TODO Talk about LO-LIMD~\cite{posik:2011:parameterless} and how it was parameter-less,
but is incapable of handling overlapping linkages.

\section{The P3 Algorithm}
\label{sec-p3}
Before going into detail about how each aspect of P3 works, consider the example
evolution presented in Figure~\ref{fig-p3}.  At the start of optimization there
is no known problem information, so solution \#1 is randomly generated.  This
is then brought to a local optima using a hill climber (Section~\ref{sec-hillclimber})
creating solution \#2, which is added to the pyramid (Section~\ref{sec-pyramid}) level
$P_0$.  As there is no further exploitable information, \#3 is randomly generated
and optimized to create \#4, which is added to $P_0$.  \#2 and \#4 are used to
define a crossover (Section~\ref{sec-crossover}), which improves \#4 to create
\#5. As \#5 used information from $P_0$ is it added to the next level of the pyramid, $P_1$.
Again P3 lacks information to continue, so \#6 is randomly generated and improved into
\#7.  Unlike \#2 and \#4, \#7 is not added in this example because it happens to be identical to
a solution already in the pyramid.  Even though \#7 is not unique, it still crosses with
$P_0$, which does make a unique improvement \#8 that is added to $P_1$ and mixed with that
level to create \#9.  As before, the process creates \#10, \#11, and \#12 using
random generation, hill climbing, and crossover, respectively.  In this example \#13
is not added to $P_2$ because it not a strict improvement in fitness over \#12.  There
is not enough information in $P_2$ to improve \#13 further, so the process would begin
yet again with a random solution.

\begin{figure}
  \centering
  \includegraphicsfit{P3_big_fail}
  \caption{Example of the first steps of P3's optimization described in Section~\ref{sec-p3}.
  \#7 is not added because it is a duplicate of a solution already in the pyramid.  \#13 is
  not added because it is not a fitness improvement over \#12.}
  \label{fig-p3}
\end{figure}

\subsection{Hill Climber}
\label{sec-hillclimber}
The terminology used to describe different methods of making minor modifications
to a solution, evaluating the quality of those changes, and iteratively constructing
new solutions, is full of duplicated, ambiguously defined terms.  As such we shall
describe what is likely an existing algorithm and give it a descriptive name that
may not match all of its previous incarnations.

For our purposes, we use the First Improvement Hill Climber (FIHC), a method of exploiting
local information through a series of modify-evaluate-accept iterations, to start
from randomly initialized solutions and end with single change local optimal results.
FIHC uses the following algorithm.  Consider all gene loci $x_i$ in a random order.
For a given $x_i$ try all possible alternative values for the gene $x_i$, evaluating
solution quality with each.  If any alternative value is a strict fitness improvement
over the current value for $x_i$, replace $x_i$ with the improved value.  If no
change is made to the solution, stop, otherwise iterate.

To avoid wasting evaluations, FIHC keeps track of all loci tested since the last
change was made to the solution.  If the algorithm returns to the same loci without
any changes, it moves on to the next loci without performing any evaluations.

FIHC solves all functions of unitation in $O(N)$ time.  This is in contrast to
methods like Steepest Ascent Hill Climber~\cite{bosman:2011:lsbbo, goldman:2012:ltga}
which require $O(N^2)$ as changes are only accepted after all neighbor solutions
are evaluated.  In general FIHC is designed to find optima using a minimum number
of evaluations.  Furthermore, because FIHC tests loci in a random order it does
not cause any positional bias to the search.


\subsection{Pyramid}
\label{sec-pyramid}
Unlike many EC techniques, P3 does not have a single population of solutions.
Instead to maintains a pyramid like structure, such that each level of the pyramid
has a population of solutions, with each level representing something akin to different
generations of evolution.  The exact relationship between the levels is described
in Section~\ref{sec-alltogether}.

The levels represent independent sets of unique solutions, such that
$\forall_{i, j \in P, i \neq j} P_i \cap P_j = \emptyset$.  This relationship
is maintained constructively, such that new solutions can only be added to a level
if they do not exist in any level of the pyramid.  This uniqueness check can be
achieved in constant time by maintaining a hash of all solutions in the pyramid.

Along with storing the solutions, each level also maintains a table of pairwise
gene value frequencies.  This is also built constructively, such that initially
all frequencies are set to 0.  Each time a solution is added to a level, all
pairs of loci are queried for their gene values.  The table is then updated,
incrementing the count for each observed pairing.  This requires $O(N^2)$ operations
(no evaluations) each time a solution is added to a level of the pyramid, but is
constant in the number of solutions currently in the level.

\subsection{Crossover}
\label{sec-crossover}
P3 uses a crossover method derived from that defined for LTGA~\cite{thierens:2013:ltgahiff}
(Section~\ref{sec-ltga}),
in that it uses gene value entropy to construct clusters of genes that should have their values
conserved during crossover.  There are two primary steps in
defining this type of crossover: cluster creation and cluster usage.

During cluster creation, P3 attempts to create a binary tree of clusters such that the leaves
represent individual gene loci and each internal node represent the subset of loci created
by joining the subsets encoded in its children.  These clusters represent detected linked
loci in the genome, and are used when performing crossover.
Using the table of observed frequencies described in Section~\ref{sec-pyramid}, Equation~\ref{eq-distance}
efficiently estimates the linkage between clusters of loci using the average pairwise
entropy, as given in Equation~\ref{eq-entropy}.
\begin{equation}
  D(C_i,C_j) = \frac{1}{\left | C_i \right |\cdot \left |C_j \right|}\sum_{c_i \in C_i}\sum_{c_j \in C_j} 
  2 - \frac{H(C_i) + H(C_j)}{H(C_i \cup C_j)}
  \label{eq-distance}
\end{equation}
\begin{equation}
  H(C) = -\sum_{s\in S} p_c(s)\log(p_c(s))
  \label{eq-entropy}
\end{equation}

Starting from the leaves, such that each loci appears exactly once, the binary tree
can be constructed using an agglomerative method.  Each iteration determine the
pairwise linkage between each cluster of loci using Equation~\ref{eq-distance},
and merge the two closest into a new cluster.  The process ends when there is only
a single cluster containing all loci.  As our clustering algorithm provides the proper
criteria this algorithm requires $O(N^2)$ total time to correctly construct the linkage tree
from the frequency table~\cite{gronau:2007:upgma}.

Not all of the clusters in the linkage tree are worth using during crossover.  The most
obvious is a cluster containing all the loci.  If we performed a crossover such that all
of the loci must remain together, no new individual would be created.  We further remove
clusters that are completely contained within some other cluster, such that if
$D(C_i,C_j) = 0$, $C_i$ and $C_j$ are not used when performing crossover.  $D(C_i,C_j)$ only
returns $0$ when the loci in $C_i$ perfectly predict the bits in $C_j$ and vice versa,
and from this we conclude that the settings for those two loci should never be disrupted.
When used on particularly small populations this has the potential to greatly reduce the number
of clusters used during crossover.  For instance, if $|P_i| = 1$ no clusters exist, and if $|P_i| = 2$
at most two clusters exist: one where the solutions agree and one where they disagree.

Given the useful set of clusters, we can now define how they are used.  P3 uses
crossover to recombine known information in a level of the pyramid with a candidate
solution.  For each cluster, do the following.  Select a random donor from the population.
Copy the donor's gene values for all loci in the cluster into the candidate solution.
Evaluate the candidate.  If the fitness of the candidate has been reduced, revert the candidate's
gene values.  When choosing the donor P3 ensures there is at least one gene value
different between the donor and the candidate solution.  This can potentially require
up to $O(|P_i|)$, but is on average more efficient than allowing donations that
do not change any gene values.

To complete a single crossover event, all of the useful clusters are tested, each
requiring at most a single evaluation.  Clusters are tested in smallest first order,
with clusters of equal size ordered randomly.  This ensures smaller building blocks
are optimized before large changes are made to the genome, which helps preserve
diversity.  As cluster building and application ordering are not dependent on
the position of loci in the genome, there is no positional bias in the search points
evaluated.

\subsection{All Together}
\label{sec-alltogether}
TODO Pieces working together.  HC removes pairwise noise, which allows crossover to detect
clusters.  Levels allow realization of linkage at high levels, reduction in noise.

\section{Comparison Algorithms}

\subsection{Random Restart First Improvement Hill Climber}
TODO Need a citation either here or Section~\ref{sec-hillclimber}.

As P3 utilizes a hill climber to perform optimization, it is necessary to show P3's
added complexity is able to outperform using the hill climber alone.  Therefore
as the first comparison we define a complete optimization method based of off the
First Improvement Hill Climber defined in Section~\ref{sec-hillclimber}.  This definition
is extended to match P3's style of restarting. Whenever the hill climber
obtains a solution that cannot be improved using a single bit flip, a new solution
is randomly generated and optimized.

If P3's quality relies solely on starting at random points and applying hill climber,
this comparison algorithm should perform even better than P3.  Similar to P3 this
algorithm is parameterless, makes no assumptions about gene ordering, and will
find the global optima given a sufficient (potentially exponential) number of evaluations.

\subsection{Linkage Tree Genetic Algorithm}
\label{sec-ltga}
The closest relative to P3 in existing literature is the Linkage Tree Genetic Algorithm
(LTGA).  There have been a number of different variations proposed to LTGA since
its original publication~\cite{thierens:2010:ltga}, so for comparison we have chosen
to use the variant described in the most recent publication~\cite{thierens:2013:ltgahiff}.
This algorithm is also the current state of the art, outperforming other algorithms
in black box optimization across numerous benchmarks.

LTGA works by iteratively improving a population of solutions in a generational manner.
Each generation the half of the solutions (chosen using standard tournament selection)
are used to construct a linkage tree just as described in Section~\ref{sec-crossover}.
Each solution in the population is then crossed with the entire population using that
linkage tree, with the results put into the next generation.  This process continues
until the population fails to change between generations, signaling convergence.

Unlike P3, LTGA does not make use of a hill climber, and there is evidence that
doing so results in decreased efficiency~\cite{bosman:2011:lsbbo}.  LTGA applies
clusters in least linked first order whereas P3 uses smallest first.  Also, LTGA
does not search for donors until one containing different genes is found.  Instead
if the donated genes are identical, it simply skips the evaluation.  LTGA
requires a population size parameter, maintains a fixed population size across
generations, and does not prevent duplicate individuals in a population.  LTGA
does not add new genetic information after initialization, meaning that if
required diversity is not contained in the original population or lost in future
generations, LTGA will fail to find the global optimum.

In order to set the population size parameter to solve a class of problems, we use
the bisection method~\cite{goldman:2012:ltga}, which determines the minimum population
size required to meet a specified success criteria.  As P3 will run until the global optima
is found or the evaluation limit is reached, we want to ensure LTGA has a similar requirement
on its success.  Therefore using the Rule of Three~\cite{jovanovic:1997:ruleofthree} we
can provide a bound on the probability of failure $\frac{3}{k+1}$ with $95\%$ confidence,
where $k$ is a number of runs without a failure.  Here we state that a population size
is successful if it performs 100 runs without failing to find the global optimum, meaning
the probability of that population failing to solve problems of the same class is bounded above by
$3\%$.  Note that these 100 runs are considered training, and the results from the bisection
runs are not included during the comparative testing.  Furthermore, when testing on problem
classes, bisection is performed on a different set of instances than testing.

Other variants of LTGA have more closely resembled P3~\cite{goldman:2012:ltga},
but as those variants are not in active use, we did not use them for comparison.
However, no LTGA variant has been able to remove the population size parameter
or the problem of premature convergence.

\subsection{\bm{$(1+(\lambda,\lambda))$}}
As a final comparison we chose the $(1+(\lambda,\lambda))$ algorithm~\cite{doerr:2013:lambdalambda}
which is currently the best theory supported crossover method.  Unlike P3 and LTGA,
$(1+(\lambda,\lambda))$ does not use a population of solutions or any model building
to determine problem variable linkage.  Instead it uses mutation to produce $\lambda$
offspring, selecting the best to recombine with the original parent to produce $\lambda$
more offspring using uniform crossover.  In the original work a method for controlling
the value of $\lambda$ is provided based on offspring success, meaning there are no
parameters to be set in this algorithm.

In order to make $(1+(\lambda,\lambda))$ a viable method of optimizing problems
with multiple local optima, we modified the original algorithm.  Primarily, if
during search $\lambda \ge N$, search is restarted from a random individual with
$\lambda=1$.  This is done because when $\lambda \ge N$ the mutation rate is greater
than or equal to $100\%$, nullifying search.  Furthermore, this point is only reached
when the algorithm has stalled for a significant number of generations.

We also made minor modifications to selection and to prevent wasting evaluations.
If the best offspring produced in a generation is a mutant offspring, select that
over any of those produced using crossover.  If there is a fitness tie between the
best offspring in a generation, select the one with the maximum hamming distance
to the parent.  This encourages drift over plateaus.  While the original paper
discusses a ``mod'' version of the $\lambda$ control strategy for how to handle
when offspring of equal fitness are produced, we found this conflicted with our
method of restarting.  As such we use the original control strategy, where $\lambda$
is increased if the best offspring in a generation is not strictly better than its parent.

\section{Test Problems}
TODO Mention that since all of our algorithms are variable order insensitive
and have no bias toward creating 1s over 0s, all will have identical results if
problem variables are randomly reordered or XOR'd.

TODO Chose only problems with known global optima to allow comparison on time to
global optima.

\subsection{Single Instance Problems}
Well defined test problem landscapes can help when performing exact analysis of
algorithms.  These problems can be thought of as testing the edge cases of optimizers,
presenting them with unnatural but interesting problem features.

The classic example for this kind of test problem is the Deceptive Trap problem.
For this benchmark the genome is broken up into non-overlapping problems of $k$
genes, with the fitness of a single ``trap'' given by Equation~\ref{eq-deceptivetrap}.
\begin{equation}
   trap(t) = \left\{
     \begin{array}{rl}
       k-1-t, &  t<k\\
       k,   &  t = k
     \end{array}
   \right.
  \label{eq-deceptivetrap}
\end{equation}
In this equation $t$ represents the number of bits set in the trap.  For each trap
there is one local optima (all bits set to 0) and one global optima (all bits set to 1).
This problem requires exponential time to solve unless there is some method to effectively
move complete traps between solutions, making this an excellent crossover test problem.

However, when using a hill climber and linkage learning, the Deceptive Trap problem
may be trivially solved~\cite{goldman:2012:ltga}.  As such the Deceptive Step Trap
problem was defined, given in Equation~\ref{eq-deceptivesteptrap}.
\begin{equation}
   step\_trap(t) = \left \lfloor \frac{(k-s)\pmod{s} + trap(t)}{s} \right \rfloor
  \label{eq-deceptivesteptrap}
\end{equation}
This problem leverages the original Deceptive Trap problem definition, with the
addition of fitness plateaus of configurable size $s$.  This creates an enormous
number of local optima and makes detection of linkages much more challenging.

While both of these problems have difficult to solve subproblems, there non-overlapping
nature and single optimal setting may not capture aspects of real world problem solving.
Therefore the Hierarchical If and only If (HIFF) problem is often used to test an
algorithms ability to handle these characteristics~\cite{thierens:2013:ltgahiff}.
In this problem the genome is broken up hierarchically into a complete binary tree,
with each gene representing a single bit, and each internal node representing the union
between all genes in that nodes subtree.  A node in the tree contributes to the fitness
only if all of the genes in its subset have the same value (IE are all 0 or are all 1).
The amount contributed is equal to the number of bits in the subset.  As a result
this problem involves creating larger and larger blocks of single value runs which
only score if the entire block is optimized.

Our final single instance problem is borrowed from the real valued optimization domain.
The Discretized Rastrigin used the standard Rastrigin evaluation function (Equation~\ref{eq-rast}).
\begin{equation}
  An + \sum_{i=1}^{n}\left [ x_i^2-A\cos (2\pi x_i) \right ] \forall x\in [-5.12,5.12]
  \label{eq-rast}
\end{equation}
This function creates a highly multimodal landscape which is completely separable
and has a single global optima.  In order to apply our algorithms to this complex problem,
we discretized the representation using 10 bit gray coded numbers evenly distributed
on the range $[-5.12,5.12)$.

\subsection{Randomly Generated Problem Classes}
By enforcing only certain characteristics of a search space while allowing others
to be generated randomly, we can perform tests on classes of problems instead
of specific instances.  This can improve the generality of results as experiments
then predict the expected quality over the entire class even if only a subset
of possible instances are actually used during experimentation.  Unlike single
instance problems, the challenge with randomly generated problems is often how
to know when the global optimum is reached.

NK landscapes define a class of problems such that the fitness of a single gene's
value is dependent on $k$ other genes in the genome.  Finding the solution which
maximized fitness on an arbitrary NK landscape requires exponential time, and
are therefore a good candidate for heuristic techniques.  Nearest Neighbor NK
landscapes are a subset of NK landscapes such that each gene is dependent on the $k$
genes following it in the genome.  The advantage of this subset is that they can be
solved in polynomial time~\cite{wright:2000:solvingnk}.  For our experiments we
used wrapping neighborhoods, such that genes near the end of the genome depend on
genes at the start of the genome.

Ising Spin Glasses are another large class of $NP-Hard$ problems.  These problems
involve minimizing the interaction terms in a graph based on edge weights by changing
the sign of graph nodes.  Similar to with NK landscapes, there is a subset of Ising problems
known as $2D\pm J$ Ising Spin Glasses.  This subset restricts graph interactions to a
toroidal two dimensional grid with edge weights of $\{-1, +1\}$.  This subset can
be solved in polynomial time~\cite{saul:1994:spinglass}, and there are existing
tools for solving arbitrary instances.\footnote{\url{http://www.informatik.uni-koeln.de/spinglass/}}

Our final problem class is a subset of the Maximum Satisfiability (MAXSAT) problem.
This problem consists of a series of disjunctive clauses, such that each clause contains
three problem variables, some of which are negated. While
MAXSAT requires exponential time to solve, we have devised a method of generating
instances that ensures each has at least one solution that satisfies all clauses.
First, a random target solution is generated.  For each clause, set the sign of each
variable in the clause such that that clause will be satisfied by the target solution.
To ensure there is no bias to variable signs in the generated problem there is a $\frac{1}{6}$ probability
all clause signs match the target, a $\frac{1}{6}$ probability two of the three signs match,
and a $\frac{4}{6}$ probability only one sign matches.  By setting the clause signs
in this way we ensure the target solution is always able to satisfy all clauses,
meaning all problems have a known best fitness.  Note that its possible there to be
other solutions that satisfy all clauses.

\section{Experimental Results}
TODO Cover all 7 problems as a group, focusing on the two ``types'' of problems.
Make it explicit that you outperform by a constant factor on the static problems
and by an order of complexity on generated problems.

TODO Mention 1.5 trillion evaluations and 150,000 runs (average 10 million evaluations per run).

TODO Include graphic for 7 problem results.

TODO Include table comparing results on largest version of each problem.

TODO Mention source code is available\footnote{\url{https://name.of.site.removed/anonymous}}.

\section{Promising Theory}
TODO Easy proof on unimodal (no plateaus) and all functions of unitation due to HC.
Sketch of how to prove runtime on Deceptive Trap, HIFF, suggestions toward all
possible non-overlapping functions.

Due to P3's persistent integration of random solutions, it always has a non zero
probability of reaching the global optimum.  Because it meets this prerequisite
(unlike LTGA), it is likely possible to derive expected running times for P3 on
a number of problems.  While we do not present a rigorous solution to any particular
problem here, we wish to illuminate potential avenues for future proofs.

The traditional One Max class, as well as all functions of unitation, will be
trivially solved by the hill climber in at most $N+1$ evaluations.  These come
from the initial evaluation of the random solution, plus the $N$ evaluations required
to test each gene for improvement.  While more difficult to bound, any unimodal
search space without fitness plateaus will be solved by a single application
of the hill climber.  Notably we expect a bound on Leading Ones of $O(N^2)$ time.

Yet unlike many theory based algorithms, we believe P3 can achieve bounds on more
complex and interesting landscapes.  Consider a sketch for a proof on the Deceptive
Trap problem.  Each restart will require at most $2N$ evaluations to evaluate the
random individual and optimize it to a local optima using the hill climber.  In
$P_0$ all solutions will either have all 1s or all 0s.  As a result $D(C_i, C_j) = 0$
if all $C_i$ and $C_j$ come from the same trap.  Therefore all crossover clusters
will move only complete traps between solutions, and once there is sufficient
diversity in the population there will be a crossover for each individual trap.
When performing a crossover of a single trap there can only be a single alternative
value, meaning if the candidate solution has all 0s the donor must have all 1s, or vice
versa.  Therefore if $P_i$ contains the optimum solution for a trap in at least one
individual and has a crossover that moves that trap, all solutions produced by crossing
$P_i$ with a candidate solution will have the optimized value of that trap.  Extending to
all traps we can say that once any $P_i$ has at least one solution with the optimized
version of a trap for all traps the very next crossover with $P_i$ will produce the global optimum.
The only remaining portion of the proof is to determine how many evaluations are required to reach this point,
which is relatively trivial.

Using loose requirements, this can likely be extended to any arbitrary fully decomposable
problem. Assuming that for sufficiently large population sizes the entropy between
bits in a single subproblem is lower than the entropy of bits between subproblems,
P3 will construct a cluster capable of moving exactly one subproblem's gene values
during crossover.  As crossover cannot decrease the fitness of a solution each
crossover will have a non zero probability of improving subproblem solution quality.
The probability of low fitness local optima for a subproblem reaching high levels
is therefore low, which increases the probability of donors only having high quality
solutions to subproblems.  Combined this is expected to create a runtime potentially
exponential in subproblem size, but polynomial in problem size.

This proof sketch can likely be extended to the HIFF problem, with minor modifications.
Each application of the hill climber ensures all subproblems of size 2 will be solved.
Once $P_0$ has a sufficient number of solutions it will correctly cross all subproblems
independently, such that all future solutions crossed with $P_0$ will have all subproblems
of size 4 solved.  While $P_1$ can contain solutions that do not correctly solve some 2 bit
subproblems (added before $P_0$ reached critical mass), the probability of these solutions
being selected as donors decreases rapidly (solutions added after $P_0$ reaches critical mass).
As such each level effectively serves to optimize the next largest subproblem size, leading to
convergence using a logarithmic number of levels.


\section{Conclusions and Future Work}
TODO No parameters required, beats tuned LTGA.  Good at solving hard classes
of problems, not just single instances of a problem as it scales to problem difficulty.
Future work in theory.  Also readily applicable to real world problems as no
problem information is required.

%
% The following two commands are all you need in the
% initial runs of your .tex file to
% produce the bibliography for the citations in your paper.
\bibliographystyle{abbrv}
\bibliography{../main}  % sigproc.bib is the name of the Bibliography in this case
% You must have a proper ".bib" file
%  and remember to run:
% latex bibtex latex latex
% to resolve all references
%
% ACM needs 'a single self-contained file'!
%
%APPENDICES are optional
\balancecolumns
%\balancecolumns % GM June 2007
% That's all folks!
\end{document}
