% This is "sig-alternate.tex" V2.0 May 2012
% This file should be compiled with V2.5 of "sig-alternate.cls" May 2012
%
% This example file demonstrates the use of the 'sig-alternate.cls'
% V2.5 LaTeX2e document class file. It is for those submitting
% articles to ACM Conference Proceedings WHO DO NOT WISH TO
% STRICTLY ADHERE TO THE SIGS (PUBS-BOARD-ENDORSED) STYLE.
% The 'sig-alternate.cls' file will produce a similar-looking,
% albeit, 'tighter' paper resulting in, invariably, fewer pages.
%
% ----------------------------------------------------------------------------------------------------------------
% This .tex file (and associated .cls V2.5) produces:
%       1) The Permission Statement
%       2) The Conference (location) Info information
%       3) The Copyright Line with ACM data
%       4) NO page numbers
%
% as against the acm_proc_article-sp.cls file which
% DOES NOT produce 1) thru' 3) above.
%
% Using 'sig-alternate.cls' you have control, however, from within
% the source .tex file, over both the CopyrightYear
% (defaulted to 200X) and the ACM Copyright Data
% (defaulted to X-XXXXX-XX-X/XX/XX).
% e.g.
% \CopyrightYear{2007} will cause 2007 to appear in the copyright line.
% \crdata{0-12345-67-8/90/12} will cause 0-12345-67-8/90/12 to appear in the copyright line.
%
% ---------------------------------------------------------------------------------------------------------------
% This .tex source is an example which *does* use
% the .bib file (from which the .bbl file % is produced).
% REMEMBER HOWEVER: After having produced the .bbl file,
% and prior to final submission, you *NEED* to 'insert'
% your .bbl file into your source .tex file so as to provide
% ONE 'self-contained' source file.
%
% ================= IF YOU HAVE QUESTIONS =======================
% Questions regarding the SIGS styles, SIGS policies and
% procedures, Conferences etc. should be sent to
% Adrienne Griscti (griscti@acm.org)
%
% Technical questions _only_ to
% Gerald Murray (murray@hq.acm.org)
% ===============================================================
%
% For tracking purposes - this is V2.0 - May 2012

\documentclass{sig-alternate}
%\linespread{2.5}
\usepackage{url}
\usepackage{verbatim}
\newcommand{\includegraphicsfit}[1]
{\includegraphics[width=\columnwidth,height=\textheight,keepaspectratio]{#1}}
\begin{document}
%
% --- Author Metadata here ---
\conferenceinfo{GECCO'14,} {July 12--16, 2014, Vancouver, BC, Canada.}
\CopyrightYear{2014}
\crdata{978-1-4503-1963-8/13/07}
\clubpenalty=10000
\widowpenalty = 10000
%\CopyrightYear{2007} % Allows default copyright year (20XX) to be over-ridden - IF NEED BE.
%\crdata{0-12345-67-8/90/01}  % Allows default copyright data (0-89791-88-6/97/05) to be over-ridden - IF NEED BE.
% --- End of Author Metadata ---

\title{Parameterless Population Pyramid}
\subtitle{[Genetic Algorithms Track]}
%
% You need the command \numberofauthors to handle the 'placement
% and alignment' of the authors beneath the title.
%
% For aesthetic reasons, we recommend 'three authors at a time'
% i.e. three 'name/affiliation blocks' be placed beneath the title.
%
% NOTE: You are NOT restricted in how many 'rows' of
% "name/affiliations" may appear. We just ask that you restrict
% the number of 'columns' to three.
%
% Because of the available 'opening page real-estate'
% we ask you to refrain from putting more than six authors
% (two rows with three columns) beneath the article title.
% More than six makes the first-page appear very cluttered indeed.
%
% Use the \alignauthor commands to handle the names
% and affiliations for an 'aesthetic maximum' of six authors.
% Add names, affiliations, addresses for
% the seventh etc. author(s) as the argument for the
% \additionalauthors command.
% These 'additional authors' will be output/set for you
% without further effort on your part as the last section in
% the body of your article BEFORE References or any Appendices.

\numberofauthors{2} %
\begin{comment}
\author{
% 1st. author
\alignauthor
Brian W. Goldman\\
       \affaddr{BEACON Center for the Study of Evolution in Action}\\
       \affaddr{Michigan State University, U.S.A.}\\
       \email{brianwgoldman@acm.org}
% 2nd. author
\alignauthor
William F. Punch\\
       \affaddr{BEACON Center for the Study of Evolution in Action}\\
       \affaddr{Michigan State University, U.S.A.}\\
       \email{punch@msu.edu}
}
%\end{comment}
%\begin{comment}
\author{
% 1st. author
\alignauthor
Anonymous\\
       \affaddr{Group}\\
       \affaddr{Organization}\\
       \email{email@site.com}
% 2nd. author
\alignauthor
Anonymous\\
       \affaddr{Group}\\
       \affaddr{Organization}\\
       \email{email@site.com}
}
%\end{comment}

\maketitle
\begin{abstract}

\end{abstract}

% A category with the (minimum) three required fields
%\category{Computing Methodologies}{Artificial Intelligence}{Search Methodologies}
\category{TODO}{TODO}{TODO}
%A category including the fourth, optional field follows...
%\category{D.2.8}{Software Engineering}{Metrics}[complexity measures, performance measures]

\terms{Algorithms}

\keywords{TODO}

\section{Introduction}
TODO~\cite{goldman:2012:ltga}.

%
% The following two commands are all you need in the
% initial runs of your .tex file to
% produce the bibliography for the citations in your paper.
\bibliographystyle{abbrv}
\bibliography{../main}  % sigproc.bib is the name of the Bibliography in this case
% You must have a proper ".bib" file
%  and remember to run:
% latex bibtex latex latex
% to resolve all references
%
% ACM needs 'a single self-contained file'!
%
%APPENDICES are optional
\balancecolumns
%\balancecolumns % GM June 2007
% That's all folks!
\end{document}
