% This is "sig-alternate.tex" V2.0 May 2012
% This file should be compiled with V2.5 of "sig-alternate.cls" May 2012
%
% This example file demonstrates the use of the 'sig-alternate.cls'
% V2.5 LaTeX2e document class file. It is for those submitting
% articles to ACM Conference Proceedings WHO DO NOT WISH TO
% STRICTLY ADHERE TO THE SIGS (PUBS-BOARD-ENDORSED) STYLE.
% The 'sig-alternate.cls' file will produce a similar-looking,
% albeit, 'tighter' paper resulting in, invariably, fewer pages.
%
% ----------------------------------------------------------------------------------------------------------------
% This .tex file (and associated .cls V2.5) produces:
%       1) The Permission Statement
%       2) The Conference (location) Info information
%       3) The Copyright Line with ACM data
%       4) NO page numbers
%
% as against the acm_proc_article-sp.cls file which
% DOES NOT produce 1) thru' 3) above.
%
% Using 'sig-alternate.cls' you have control, however, from within
% the source .tex file, over both the CopyrightYear
% (defaulted to 200X) and the ACM Copyright Data
% (defaulted to X-XXXXX-XX-X/XX/XX).
% e.g.
% \CopyrightYear{2007} will cause 2007 to appear in the copyright line.
% \crdata{0-12345-67-8/90/12} will cause 0-12345-67-8/90/12 to appear in the copyright line.
%
% ---------------------------------------------------------------------------------------------------------------
% This .tex source is an example which *does* use
% the .bib file (from which the .bbl file % is produced).
% REMEMBER HOWEVER: After having produced the .bbl file,
% and prior to final submission, you *NEED* to 'insert'
% your .bbl file into your source .tex file so as to provide
% ONE 'self-contained' source file.
%
% ================= IF YOU HAVE QUESTIONS =======================
% Questions regarding the SIGS styles, SIGS policies and
% procedures, Conferences etc. should be sent to
% Adrienne Griscti (griscti@acm.org)
%
% Technical questions _only_ to
% Gerald Murray (murray@hq.acm.org)
% ===============================================================
%
% For tracking purposes - this is V2.0 - May 2012

\documentclass{sig-alternate}
%\linespread{2.5}
\usepackage{url}
\usepackage{verbatim}
\newcommand{\includegraphicsfit}[1]
{\includegraphics[width=\columnwidth,height=\textheight,keepaspectratio]{#1}}

\usepackage{bm}
\begin{document}
%
% --- Author Metadata here ---
\conferenceinfo{GECCO'14,} {July 12--16, 2014, Vancouver, BC, Canada.}
\CopyrightYear{2014}
\crdata{978-1-4503-1963-8/13/07}
\clubpenalty=10000
\widowpenalty = 10000
%\CopyrightYear{2007} % Allows default copyright year (20XX) to be over-ridden - IF NEED BE.
%\crdata{0-12345-67-8/90/01}  % Allows default copyright data (0-89791-88-6/97/05) to be over-ridden - IF NEED BE.
% --- End of Author Metadata ---

\title{Parameterless Population Pyramid}
\subtitle{[Genetic Algorithms Track]}
%
% You need the command \numberofauthors to handle the 'placement
% and alignment' of the authors beneath the title.
%
% For aesthetic reasons, we recommend 'three authors at a time'
% i.e. three 'name/affiliation blocks' be placed beneath the title.
%
% NOTE: You are NOT restricted in how many 'rows' of
% "name/affiliations" may appear. We just ask that you restrict
% the number of 'columns' to three.
%
% Because of the available 'opening page real-estate'
% we ask you to refrain from putting more than six authors
% (two rows with three columns) beneath the article title.
% More than six makes the first-page appear very cluttered indeed.
%
% Use the \alignauthor commands to handle the names
% and affiliations for an 'aesthetic maximum' of six authors.
% Add names, affiliations, addresses for
% the seventh etc. author(s) as the argument for the
% \additionalauthors command.
% These 'additional authors' will be output/set for you
% without further effort on your part as the last section in
% the body of your article BEFORE References or any Appendices.

\numberofauthors{2} %
\begin{comment}
\author{
% 1st. author
\alignauthor
Brian W. Goldman\\
       \affaddr{BEACON Center for the Study of Evolution in Action}\\
       \affaddr{Michigan State University, U.S.A.}\\
       \email{brianwgoldman@acm.org}
% 2nd. author
\alignauthor
William F. Punch\\
       \affaddr{BEACON Center for the Study of Evolution in Action}\\
       \affaddr{Michigan State University, U.S.A.}\\
       \email{punch@msu.edu}
}
%\end{comment}
%\begin{comment}
\author{
% 1st. author
\alignauthor
Anonymous\\
       \affaddr{Group}\\
       \affaddr{Organization}\\
       \email{email@site.com}
% 2nd. author
\alignauthor
Anonymous\\
       \affaddr{Group}\\
       \affaddr{Organization}\\
       \email{email@site.com}
}
%\end{comment}

\maketitle
\begin{abstract}
TODO
\end{abstract}

% A category with the (minimum) three required fields
%\category{Computing Methodologies}{Artificial Intelligence}{Search Methodologies}
\category{TODO}{TODO}{TODO}
%A category including the fourth, optional field follows...
%\category{D.2.8}{Software Engineering}{Metrics}[complexity measures, performance measures]

\terms{Algorithms}

\keywords{TODO}

\section{Introduction}
TODO Parameterless optimization, model building, etc.

TODO Talk about ALPS~\cite{hornby:2006:alps}.

\section{The P3 Algorithm}
TODO Overview.  Use Figure~\ref{fig-p3}.

\begin{figure}
  \centering
  \includegraphicsfit{P3_big_fail}
  \caption{Example}
  \label{fig-p3}
\end{figure}

\subsection{Hill Climber}
TODO Describe First Improvement Hill Climber, including waste prevention, etc.

\subsection{Pyramid}
TODO Explain population levels, entropy tables $O(n^2)$, uniqueness

\subsection{Crossover}
TODO Explain how entropy becomes clusters efficiently $O(n^2)$~\cite{gronau:2007:upgma}, cluster ordering,
keep changes that don't reduce fitness, removal of zero entropy sub-clusters,
keep donating until a change is made.

\subsection{All Together}
TODO Pieces working together.  HC removes pairwise noise, which allows crossover to detect
clusters.  Levels allow realization of linkage at high levels, reduction in noise.

\section{Comparison Algorithms}

\subsection{Random Restart First Improvement Hill Climber}
TODO This is important to have as it shows how the pyramid is able to reach optima
the hill climber cannot.  Need a citation.

\subsection{Linkage Tree Genetic Algorithm}
TODO Discuss LTGA~\cite{thierens:2013:ltgahiff}, how it differs from P3, its limitation based on
initialization, requirement of population size.  Mention how some
of the ideas in P3 were tried in LTGA and seemed to make improvements~\cite{goldman:2012:ltga}.
Discuss bisection method and Rule of Three~\cite{jovanovic:1997:ruleofthree}.
Remember to say that the paper argues for $\frac{3}{N+1}$, so 100 trails is strictly
less than $3\%$ expected failure rate.

\subsection{\bm{$(1+(\lambda,\lambda))$}}
TODO Explain briefly how it works, that it is the best theory supported crossover method~\cite{doerr:2013:lambdalambda}.
Explain changes including: Keep best mutant if no worse than crossover offspring,
prevent evaluations if crossover offspring identical to either parent, if tied fitness
in crossover offspring take maximum hamming distance to parent.  If $\lambda \ge N$
restart with random individual.  Does not use ``mod'' version as restarting is preferable
when stuck on plateaus.

\section{Test Problems}

\subsection{Single Instance Problems}
TODO Deceptive Trap, Deceptive Step Trap~\cite{goldman:2012:ltga}, Hierarchical If
and Only If, Discretized Rastrigin.

\subsection{Randomly Generated Problem Classes}
TODO Nearest Neighbor NK with wrapping neighborhoods, Ising Spin Glasses on toroidal
grids with only -1, 1 spins~\cite{saul:1994:spinglass}.
Used\footnote{\url{http://www.informatik.uni-koeln.de/spinglass/}}, Maximum Satisfiability constructed with known global
optimum.

\section{Experimental Results}
TODO Cover all 7 problems as a group, focusing on the two ``types'' of problems.
Make it explicit that you outperform by a constant factor on the static problems
and by an order of complexity on generated problems.

TODO Mention 1.5 trillion evaluations and 150,000 runs (average 10 million evaluations per run).

TODO Include graphic for 7 problem results.

TODO Include table comparing results on largest version of each problem.

\section{Promising Theory}
TODO Easy proof on unimodal (no plateaus) and all functions of unitation due to HC.
Sketch of how to prove runtime on Deceptive Trap, HIFF, suggestions toward all
possible non-overlapping functions.

\section{Conclusions and Future Work}
TODO No parameters required, beats tuned LTGA.  Good at solving hard classes
of problems, not just single instances of a problem as it scales to problem difficulty.
Future work in theory.  Also readily applicable to real world problems as no
problem information is required.

%
% The following two commands are all you need in the
% initial runs of your .tex file to
% produce the bibliography for the citations in your paper.
\bibliographystyle{abbrv}
\bibliography{../main}  % sigproc.bib is the name of the Bibliography in this case
% You must have a proper ".bib" file
%  and remember to run:
% latex bibtex latex latex
% to resolve all references
%
% ACM needs 'a single self-contained file'!
%
%APPENDICES are optional
\balancecolumns
%\balancecolumns % GM June 2007
% That's all folks!
\end{document}
