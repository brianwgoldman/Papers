\documentclass[a4paper, 11pt]{article}

\usepackage{fancyhdr}
\usepackage[margin=1in]{geometry}
\usepackage{titlesec}
\usepackage{verbatim}

\titleformat{\section}
{\normalfont\Large\bfseries}{}{0pt}{}
\titleformat{\subsection}
{\normalfont\itshape}{}{0pt}{}

\usepackage{enumitem}
\setlist[itemize,enumerate]{noitemsep,nolistsep}
%\linespread{2.5}
\begin{document}
\thispagestyle{empty}

\pagestyle{fancy}
\lhead{Brian W. Goldman}
\rhead{Teaching Statement}

\begin{center}
{\LARGE \bf TEACHING STATEMENT}\\
\vspace*{0.1cm}
{\normalsize Brian W. Goldman (brianwgoldman@acm.org)}
\end{center}

\noindent
A critical skill in both teaching and research is the ability to convey
complex topics in meaningful terms to an uninformed audience. In both
contexts this requires an understanding of what a listener or reader
already knows, what they will have difficulty with, and how best to
connect the two. This is especially important when trying to overcome
the cultural boundaries which turn away underrepresented groups from
the field of computer science.

\section{Quotations from Student Evaluations}
\begin{itemize}
\item ``Brian is a great TA. I consistently attended the labs because I was always learning and benefiting.''
\item ``Brian was always very helpful and also found a way to make you figure out what you were doing on your own.''
\item ``Often explained things that were confusing in an easy way to understand.''
\end{itemize}

\section{Teaching Classes of Moderate Size}
Due to my previous success as a Lab Instructor, I was offered a
position as Lead Instructor for the second level undergraduate
programming course. This class is designed to introduce the students,
who learn Python in the first semester, to C++11. They learn
everything from basic syntax and STL tools to templates and how
to building their own data structures with dynamic memory.
I was given full autonomy to create course content, including
lectures, assignments, and assessments. To help with grading
assignments and lab work created by the 37 students, the
department assignment me two Lab Instructors.
I am proud to report that I received an average
of 3.81 out of 4 in post course student assessment.

In order to create an inclusive environment, I worked to
show my concern for each student's understanding. While
lecturing I frequently paused and prompted for questions,
and did my best to gauge whether students were understanding
the material. This would often result in me changing the pace
of the lecture to meet the students' comprehension, either
skipping slides or moving to the chalkboard to give further
examples. For those students who did not want to ask questions
in class, I utilized Piazza, an online tool where students
can post their questions anonymously. These questions could be
visible to everyone in the class, allowing students to answer
each others' questions, or only visible only to me. 

Recent studies have shown that women are more likely to stay
in computer science when courses are application oriented
as opposed to abstract problem solving. As such, when creating
assignments I worked hard to give them concrete,
real world significance. When learning about the ``map'' data
structure and the random number generator, I had the students
estimate the centrality of websites using random walks.
When learning the basics of object oriented programming,
I had them create a stock market simulator using real data.
For the first assignment, in which the students know so little
syntax that it is hard to do anything meaningful, I had the
students calculate the height of US debt when stacking different denominations of currency.

In classes and labs I focused on collaborative experiential learning.
Each class, to break up the lecture, I would give students
relevant programming tasks to complete in small groups.
These worksheets would prompt them to write code to solve a
problem or to explain the behavior of a block of code. The solution
required them to use the material I had just covered in lecture as
well as topics from the previous days. By allowing them to collaborate,
students were able to help each other understand the topics better.
The lab assignments had a similar design, such that students worked
in pairs to solve problems requiring similar skills as upcoming
homework assignments. In this way students saw material in five
contexts: Reading, Lecture, Worksheet, Lab, and Assignment.

I have also had the opportunity to give a handful of lectures in graduate level courses.
These invited talks focused on my research and how it related
to the general optimization principles students were learning.
As before I found the key to successfully teaching these students
was understanding how best to connect the new information to
what they already knew. I covered my research in terms they
were familiar with, contrasting them with algorithms they had
previously learned. As a result I received very positive
feedback from both the course instructor and the students.

\section{Teaching Classes of Large Size}
While my primary experiences have been with moderately sized
classes (25-50 students), I have taught a few lectures to much
larger classes (120+). Lectures of that size make it much more
difficult to interactively gauge student comprehension, making
alternative teaching tools all the more important. The small
group worksheet approach became even more critical as, unlike
the lecture, it is unaffected by class size. It also provided
me an opportunity to work with small groups of students to
answer specific questions or to clarify ideas, something impossible
during large scale lecture. As a teaching assistant in these
courses I saw the effectiveness of video tutorials and online
classroom management systems which allow students to work together to answer each other's questions.

Effective large scale education requires new approaches,
and computer science has some especially promising new
technologies. In the future I plan to provide students
unit tests and source code analysis tools, allowing them
to receive immediate and automated feedback. This could
be integrated with version control and continuous build systems,
simultaneously allowing these techniques to be used with even
the most introductory students and teaching them important
skills necessary for success in industry. Once set up these
methods can scale to accommodate any class size and support
distance and asynchronous learning.

\begin{comment}
\section{Outreach}
During my masters I was the chair of an ACM SIG set up to organize
the MegaMinerAI programming competition each semester.
In this competition small teams of students are given
24 hours to write programs which act as the player in a computer game.
These programs compete with those written by other students,
with the results visualized. The competition focuses on making
it as easy as possible to write useful AIs, and making the game
as visually interpretable as possible. These goals help create
a community building challenge which is open to creativity and
translates student's work into visual improvement.
The competition saw a significant growth in participation under
my direction, with nearly 100 competitors from a department of
only 400 students. For my work with the organization the department
awarded me “Leader of the Year.”
\end{comment}

\section{Future Goals}
While much of my teaching experience has been with introductory
programming courses, I am also interested in teaching higher level
and theory based courses. Artificial intelligence and machine learning
are common courses which coincide strongly with my research interests.
Algorithms, discrete math, and graph theory are topics that I believe
present exciting challgenges for keeping students engaged.
From previous work I have also acquired significant knowledge of
introductory and high performance C++ and would enjoy teaching classes
on those topics.

The best way to communicate information is on the receiver's
terms, regardless of if the context is teaching or research.
As such developing better methods for reaching both specific
individuals and broad audiences is essential to my career goals,
and is something I plan to continue improving.
\end{document}

