\chapter{Proposed Research}
The Parameter-less Population Pyramid (P3) was shown to be effective in Chapter~\ref{chap-benchmarking}
when applied to black-box optimization. Some of this effectiveness in due to P3's integration of
efficient local search and its ability to automatically detect and exploit variable epistasis.
In Chapter~\ref{chap-graybox} we presented the gray-box domain, which contains many interesting
real-world problems, and the improvements that can be made to local search in this domain.
This presents the following questions:

\begin{itemize}
\item How can hamming ball local search be integrated into a global optimization method?
\item What improvements can be made to P3's crossover operator when linkage is known?
\item How effective is a gray-box P3 algorithm in comparison with other gray-box techniques?
\end{itemize}

\section{Simple Memetic Algorithm Using Efficient Search}
%8.22154, 24.0066, 24.0077, 24.0016, 24.0108, 24.0032, 24.0095, 24.0035, 24.0102, 24.0071, 24.0078, 24.0061, 24.0023, 24.0055, 24.0016, 24.0022, 24.0094, 24.0045, 24.0064, 24.003, 24.0109, 24.0099, 24.0038, 4.78113, 5.47084, 24.0015, 24.0088, 24.0066, 24.0019, 20.6173, 24.0023, 18.9085, 24.0063, 24.0001, 24.0041, 24.0065, 24.002, 24.0017, 24.0029, 24.0096, 24.001, 24.0103, 24.0076, 24.0064, 2.73054, 24.0035, 24.0086, 24.0013, 24.0057, 24.0112
No current published research has examined how effective hamming ball hill climbing is at reaching the global
optimum. However, our preliminary experiments have found that even using large $r$ values when solving relatively small $N$
problems, hamming ball hill climbing is not sufficient to find the global optimum. When applied to 50 instances
of Nearest Neighbor NK using $N=200$, $K=5$, $r=4$, and limited to using less time than the slowest P3 run to solve that class (24 minutes),
the hamming ball hill climber only found the global optimum 6 times.

Therefore as a first step in exploring this new domain we propose to explore simple memetic algorithms~\cite{chen:2011:memetic}
that make use of this hill climber. The goal is to determine if combination with methods like uniform crossover
can result in improved search quality.

\begin{figure}
  \begin{algorithmic}[1]
  \Procedure{Iterate-SUT}{}
    \State Create random solution
    \State Apply hamming ball hill climber to solution
    \ForAll{$T_i \in T$}
      \If{$T_i$ is empty}
        \State $T_i \leftarrow$ solution
        \State \Return
      \EndIf
      \State Cross solution with $T_i$ to create $2^i$ offspring
      \State Apply hamming ball hill climber to each offspring
      \State solution $\leftarrow$ best of offspring, solution, and $T_i$
      \State $T_i \leftarrow$ empty
    \EndFor
    \State Add solution to end of $T$
  \EndProcedure
\end{algorithmic}
  \caption{One iteration of SUT optimization. $T$ is an
           ordered list of solution, each of which could be empty,
           awaiting a crossover partner.}
  \label{fig-sut}
\end{figure}


We tentatively refer to the first proposed algorithm of this type as the Simple Uniform crossover Tournament (SUT)
algorithm. Conceptually, SUT is meant to be the simplest way to combine uniform crossover with hamming ball hill climbing.
Solutions compete in an elimination tournament. Each contest in the tournament uses crossover between the solutions, with each
offspring subsequently brought to a local optimum using hamming ball hill climbing. The best solution of the parents and offspring then
advances up the tournament.

Similar to P3, this algorithm is entirely iterative and requires no parameters. Figure~\ref{fig-sut} provides the high level algorithm
for SUT. Each iteration starts by generating a random solution and improving it using the hamming ball hill climber. It then
iteratively climbs the tournament $T$, competing with each $T_i$ if $T_i$ has a stored solution. If $T_i$ is empty the solution
is stored until its competitor is found, and if the solution is not stored before reaching the top of the current bracket, a new
level is added. Adding a new level means that a complete binary tree of crossovers has been created, allowing this algorithm
to simulate an elimination tournament without knowing the size of the tournament beforehand. Furthermore, this algorithm should
never need to be restarted from scratch as new diversity is added and improved before being biased by previously found solutions.

When competing two solutions, crossover is used $2^i$ times, where $i$ is $[0..|T|-1]$. This means that the total number of crossovers
performed at each level is equal. For example, to have a single competition at $T_2$ requires two competitions at $T_1$ and four
competitions at $T_0$, with each level performing 4 crossovers.
Each crossover produces a single offspring using uniform crossover, which is then improved using
the hamming ball hill climber. The best of all offspring and the two parents then compete on fitness, with the best advancing
up the tournament.

\subsection{Preliminary Empirical Comparison}
%1.30066, 0.668016, 20.0872, 24.015, 6.84026, 0.620081, 1.43271, 10.9687, 0.625069, 24.0347, 2.84006, 1.57277, 2.81627, 7.82749, 0.63878, 1.52631, 1.31218, 1.41851, 0.650093, 2.79497, 3.01983, 0.28769, 24.0449, 0.292457, 0.316901, 2.957, 1.35011, 0.636912, 2.8645, 0.578051, 1.30444, 0.274158, 3.16518, 0.141998, 1.34107, 13.8565, 0.579429, 0.345084, 0.650234, 13.4685, 3.04961, 0.670062, 1.53499, 19.3596, 2.76979, 1.50853, 1.3763, 1.4152, 12.3199, 2.13644
When applied to the same problem and using the same $r$, SUT significantly outperforms the hamming ball hill climber.
On Nearest Neighbor NK $N=200$, $K=5$, $r=4$, SUT was successful 47 of 50 times when limited to 24 minutes of computation.
Its median time to success was 1.47 minutes, which is on par with P3's black-box median time of 1.27 minutes.
There is a good chance that this similarity will change on larger problems. Its possible that SUT will not
continue to be able to find the global optimum as $N$ increases. It is also possible that P3's use of complete solution
reevaluation and modeling may cause its runtime to grow faster than SUT. Further experimentation is required to know
with certainty which effect will dominate search time.

\section{Developing Efficient Linkage Aware Crossover}
The reason for recording variable entropy in P3 is to detect unknown variable linage in the black-box domain. However,
the gray-box domain makes high level variable linkage fully known. Therefore by leveraging this information it should
be possible to more efficiently construct crossovers that are even more accurate than the inferred black-box ones.

There are a number of possible avenues for future research here. First, should the goal be to derive a single
static linkage model for crossover or multiple random approximations? It may be possible to compute for
a given problem the set of all useful crossovers that best balance exploitation with exploration. As
this only needs to be done once per problem, the algorithm for creating the set can be relatively time consuming.
The drawback of such a design is that the set may be prohibitively large, require an accurate understanding of what
the best crossovers are, and repeated use of a static crossover set may adversely effect search.

Repeatedly generating new random crossover sets overcomes many of these issues. An individual set can be relatively small
as it does not need to contain all possibly useful crossovers. Simultaneously random generation allows the definition for
what types of crossovers should be used to be less rigorous, favoring instead a more inclusive definition. The drawback
is that generating random sets may be prohibitively expensive, it may unnecessarily create useless crossovers, and fail
to create difficult to detect but important crossovers.

Independent of if static or random crossovers are used, there are important research questions related to how variable epistasis
should best be leveraged in crossover. One possibility is that crossover should aim to preserve build blocks. For instance P3's
crossover attempts to cross entire traps on Deceptive Trap, as mixing values within a trap does not lead to the global optimum.
Under this principle it seems reasonable to assert that crossovers that ensure variables which share a non-linear relationship
are crossed as a group are better than those that do not. The task then is to decide how to handle subfunction overlap to allow
for sufficient exploration.

Conversely, it may be that intentionally mixing values within subfunctions leads to better exploration. When using a hamming
ball hill climber as a repair function, taking genes from multiple solutions for a subfunction may be beneficial. Consider
that on Deceptive Trap were $r\geq trap\_size/2$ the best crossover can be to split every trap in half. While the immediate
offspring may have low fitness, the hill climber is then able to reach either parent's values for the trap: If the parents
disagree for a trap, the hill climber can flip enough bits in a single move to return to either's value. Furthermore, it
also allows the hill climber to search for new high quality ways of solving that subfunction.

As an initial experiment, we have focused on recreating the P3 style of linkage tree without using entropy tables.
Figure~\ref{fig-sfx-tree} provides an algorithm for using subfunction information to create a linkage tree.
Initially all variables start in their own cluster. Each subfunction is then used in a random order such that
if that subfunction overlaps multiple top-level clusters it merges them to create a new cluster.

\begin{figure}
  \begin{algorithmic}[1]
  \Procedure{SubfunctionTree}{}
    \State $clusters \leftarrow [\{0\}, \{1\}, \{2\}, \dots, \{N-1\}]$
    \State $cluster\_number \leftarrow [0 .. N-1]$
    \ForAll{$subfunction \in shuffled(subfunctions)$}
      \State $to\_merge \leftarrow \emptyset$
      \ForAll{$b \in subfunction$}
        \State $to\_merge \leftarrow to\_merge \cup \{cluster\_number[b]\}$
      \EndFor
      \If{$|to\_merge| > 0$}
        \State $new\_cluster \leftarrow \emptyset$
        \ForAll{$i \in to\_merge$}
          \State $new\_cluster \leftarrow new\_cluster \cup clusters[i]$
        \EndFor
        \ForAll{$b \in new\_cluster$}
          \State $cluster\_number[b] \leftarrow |clusters|$
        \EndFor
        \State $clusters \leftarrow clusters + new\_cluster$
      \EndIf
    \EndFor
    \State Remove first $N~clusters$ and any cluster containing all variables
  \EndProcedure
\end{algorithmic}
  \caption{Algorithm used to convert a list of subfunctions into linkage tree clusters.}
  \label{fig-sfx-tree}
\end{figure}

The computational cost of this algorithm is $O(|cluster|)$ for each $cluster$ created.
Assuming that the problem is not separable, this algorithm will create
$\Omega(N)$ clusters. This is a safe assumption in the gray-box domain as it is more
efficient to solve each component independently.
In order to complete,
each bit of each subfunction must be referenced, requiring $O(Nc)$ time. As such this cost can
be amortized over the $\Omega(N)$ clusters to become $O(1)$.
Each time a new cluster is created
the $cluster\_number$ of all bits in that cluster must be updated, requiring $O(|cluster|)$ time.
Therefore using this algorithm detecting clusters is only a constant factor slower than using those clusters
to perform donations.

This methods has many desirable properties beyond efficiency. All crossovers always move at least one entire subfunction,
and clusters always differ by at least one entire subfunction. The probability two variables are linked in a
cluster increases as the number of subfunctions they share increases. This extends to indirect relationships, such that
the more paths of a given length between two variables the more likely they are to be kept in the same cluster.
Yet because this process is stochastic, any connected subset of subfunctions can be a cluster.

\subsection{Preliminary Empirical Comparison}
We tested P3 using the gray-box \Call{SubfunctionTree}{} in place of the black-box \Call{Cluster-Creation}{}
and using the gray-box hamming ball hill climber instead of black-box hill climbing. As an added efficiency
instead of recalculating all $delta$ values and assuming all moves are open each time a crossover donation is made,
only $delta$s and moves affected by the donation are updated. This means that hill climbing requires $O(|cluster|+I)$
time.

%0.0939752, 0.154129, 0.392246, 0.721644, 0.110155, 0.241687, 0.321031, 0.172651, 0.567039, 0.480878, 0.109734, 0.176248, 0.343879, 0.301121, 0.202199, 0.17966, 0.0330468, 0.117303, 0.0885791, 0.0588664, 0.594459, 0.101318, 1.86743, 0.0350668, 0.598237, 0.388794, 0.208924, 0.240061, 0.103984, 0.21943, 0.238308, 1.10106, 0.142684, 0.332108, 0.336863, 0.49791, 0.119978, 0.155924, 0.243874, 0.142354, 0.371339, 0.0854216, 0.168894, 0.419617, 0.128611, 1.9637, 0.288967, 0.142157, 0.103391, 0.307311
On Nearest Neighbor NK using
$N=200$, $K=5$, $r=4$ this found the global optimum in all 50 runs, with a median time to success of 0.23 minutes. This makes it the
most successful gray-box tested so far, and 6 times faster than applying black-box P3. Also, unlike the other
methods decreasing $r$ actually makes it more efficient, achieving a median time to success of 0.013 minutes (0.78 seconds) with $r=1$,
97 times faster than black-box P3.
%1.30066, 0.668016, 20.0872, 24.015, 6.84026, 0.620081, 1.43271, 10.9687, 0.625069, 24.0347, 2.84006, 1.57277, 2.81627, 7.82749, 0.63878, 1.52631, 1.31218, 1.41851, 0.650093, 2.79497, 3.01983, 0.28769, 24.0449, 0.292457, 0.316901, 2.957, 1.35011, 0.636912, 2.8645, 0.578051, 1.30444, 0.274158, 3.16518, 0.141998, 1.34107, 13.8565, 0.579429, 0.345084, 0.650234, 13.4685, 3.04961, 0.670062, 1.53499, 19.3596, 2.76979, 1.50853, 1.3763, 1.4152, 12.3199, 2.13644
\section{Comparison of Gray Box Methods}

Current research on these gray-box methods have been preliminary: a single size of a single problem class using
a single $r$, $k$, and $c$. To fully understand how to effectively perform search in this new domain, much more extensive testing
is necessary.

Just as when testing P3 in the black-box domain, an important step will be to identify a representative set
of gray-box problems. Planned test problems are NK landscapes, Ising Spin Glasses, and MAX-SAT. Beyond testing the
problems with known global optimum, we plan to try Unrestricted NK and Ising Spin Glasses with different linkage models.
This is critical as the behavior of many gray-box algorithms depend on the nature of the epistasis graph. In these domains
the focus will shift from finding the global optimum (which is unknown) to finding the best solution in a fixed amount of time.

Beyond the randomly generated instances, we plan to seek out real-world problems that can be expressed in this domain.
This will help solidify the conclusions about the applicability of these methods outside of the optimization community.
Similarly, we plan to compare the results of our gray-box methods with algorithms specifically designed for each problem
domain. This provides an interesting contention between the time required to develop problem-specific solvers and the
quality of using gray-box methods.

As the hamming ball hill climber has a user-specified parameter $r$, we will need to explore this parameter space.
Specifically we will want to ensure all comparisons use an appropriate value for $r$ that does not scale with
$N$. This process will likely involve brute force testing all small $r$ values for moderately large problems in each class
and then choosing the one that performs best. That $r$ value will then be used for all problem sizes for that problem class.

Another area of focus will be in how $N$ effects the relative effectiveness of the methods. One of the goals of moving
to a gray-box domain is how it impacts Big-$O$ complexity. As a result we plan to empirically verify the scalability
of each algorithm, as well as the wall clock time to find the global optimum. This is necessary as unlike the black-box
domain where evaluation is considered arbitrarily expensive, evaluation in the gray-box domain is perfectly efficient.

\section{Summary of Proposed Research Questions}
\begin{itemize}
\item How does the combination of uniform crossover with hamming ball hill climbing compare to hill climbing alone?
\item Does preserving subfunctions intact during crossover improve search efficiency?
\item How does black-box P3 compare with gray-box P3?
\item How do the more generic gray-box methods compare with problem-specific methods for solving real-world problems?
\end{itemize}
