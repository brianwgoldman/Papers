\chapter{Comparison Optimizers}
\label{chap-optimizers}

In order to fully understand the effectiveness of P3, we compare it with five advanced algorithms that
have related features to P3. The Random Restart Hill Climber defined by ~\cite{goldman:2014:p3} was chosen as an efficient form of repeated
local search. As P3 combines this hill climber with crossover,
comparing with local search alone shows the advantages of P3's overall approach.
The $(1+(\lambda, \lambda))$ algorithm~\citep{doerr:2013:lambdalambda} is the current best theory supported simple genetic algorithm
and its method of crossover is in some sense a macro-mutation just as in P3. hBOA and Parameter-less
hBOA are advanced model building search techniques that are effective at learning complex
problem structure, designed to achieve similar goals as P3's linkage learning but using very different
methods. Finally LTGA represents the current state-of-the-art in black-box search and is the origin
of P3's linkage learning and crossover methods.

Only hBOA and LTGA require any parameters, with each of these only requiring a population
size. This makes knowing the optimal behavior of these algorithms much more tractable. All of the
algorithms are also gene order independent, fitness scale invariant, and unbiased. This means,
for any problem, the order in which problem variables appear in the genome can be changed
without changing the behavior of the search. The fitness can also be manipulated in any fashion
as long as the rank ordering of solutions is unchanged. These algorithms are also unaffected by the meaning
assigned to each bit, such that inverting a predetermined random subset of genes before evaluation
will not impact search efficiency.

Our implementations of all of these algorithms as well as all of the population size information, raw results,
and processing scripts
are available from our website.\footnote{\url{https://github.com/brianwgoldman/FastEfficientP3}}

%%%%%%%%%% HC %%%%%%%%%%%%%%%%
\section{Random Restart Hill Climber}
\label{sec-hill-climber}
Perhaps the simplest black-box search heuristic is stochastic local
search, also know as hill climbing.
This optimization technique focuses on improving a single solution until it reaches a local
optimum. Here we use the first-improvement hill climber defined by \cite{goldman:2014:p3}
and given in Figure~\ref{fig-hc}. This algorithm works by flipping each bit in a random
order, keeping modifications when fitness is improved, until single bit flips cannot result in
further fitness improvements.

\begin{figure}
  \begin{algorithmic}[1]
  \Procedure{Hill-Climber}{}
    \State $options \leftarrow [0 \dots N-1]$\label{fig-hc-options}
    \State $tried \leftarrow \emptyset$
    \While{$|tried| < |options|$}
      \ForAll{$index \in shuffled(options)$}
        \If{$index \notin tried$}\label{fig-hc-tried}
          \State Flip bit $index$ in solution
          \If{solution's fitness increased}
            \State $tried \leftarrow \emptyset$
          \Else
            \State Revert change
          \EndIf
          \State $tried \leftarrow tried \cup \{index\}$
        \EndIf
      \EndFor
    \EndWhile
  \EndProcedure
\end{algorithmic}
  \caption{Hill climbing algorithm used to improve randomly generated solutions until no single
           bit change results in a fitness improvement.}
  \label{fig-hc}
\end{figure}

The hill climber requires an amortized cost of \BigO{1} operations per evaluation. In order to
terminate, at least one evaluation must be performed for each of the $N$ bits in the solution.
As such any operation that happens only once per search can be amortized over at least $N$
evaluations, covering the initialization of $options$ on Line~\ref{fig-hc-options}.
Line~\ref{fig-hc-tried}, which prevents wasted evaluations, can be called at most twice
per evaluation: once to add $index$ into $tried$ and once to prevent $index$ from being unnecessarily
evaluated again. The only way three or more calls could happen is if
no fitness improvement was made for the entire previous iteration, which contradicts the loop invariant.

Due to its nature, this hill climber cannot escape basins of attraction. Once a solution is reached
such that none of the single bit neighbors are fitness improvements, search stops. Thus
this algorithm requires a restart mechanism to solve multimodal problems. We have
chosen here to na\"{i}vely restart search from a random solution whenever a local optima is found. This
ensures that on all landscapes there is always a non-zero probability of search finding the global optimum.

%%%%%%%%%%%% Lambda Lambda %%%%%%%%%%%%%%%%
\section{$(1+(\lambda, \lambda))$}
\cite{doerr:2013:lambdalambda} presented the first genetic algorithm to provably show
the advantages of performing crossover on
% bwg: Instead of "the problem known as" how about "the One Max" problem?
the problem known as % wfp
One Max. 
% wfp could use a reference for OneMax
% bwg: I've looked, and One Max seems to be so common that no one cites a seminal paper for it.
This comparatively simple
algorithm maintains only a single individual and a self-controlled parameter $\lambda$.

Each iteration, the number of bits to flip is chosen from the binomial distribution $b\sim B(N, \frac{\lambda}{N})$,
where $N$ is the number of bits in the genome.
Next, $\lfloor\lambda\rfloor$ offspring are produced by flipping $b$ bits. The
best mutant then produces $\lfloor\lambda\rfloor$ offspring via uniform crossover with the original parent, such that each gene comes from the
mutant with probability $\frac{1}{\lambda}$. In the original algorithm the best
offspring produced by crossover then replaces the original parent if its fitness is no worse.
The $\lambda$ parameter, which is initialized to 1, is decreased if the offspring replaced
its parent and increased otherwise.

The original formulation was designed specifically for unimodal landscapes and as such were
not directly suitable for multimodal problems. \cite{goldman:2014:p3} extended $(1+(\lambda, \lambda))$
to include random restarts. As search stagnates, the $\lambda$ parameter increases in value. Eventually
this results in $\lambda \ge N$ causing mutation to always flip all bits of the individual.
As this prevents any future improvement, whenever $\lambda \ge N$ search is restarted from a random solution with $\lambda$
reset to 1.

A few other efficiency modifications were also made. If there is a tie in crossover offspring fitness,
whichever has a larger hamming distance from the parent is retained. This encourages drifting across plateaus.
The ``mod'' control strategy proposed by \cite{doerr:2013:lambdalambda} was not used as it conflicted with
the random restart strategy.
If a crossover individual is identical to either of its parents, it is not evaluated.
If mutation produces an offspring that is better than the best crossover offspring, it is used to compare
against the original parent.

%%%%%%%%%%%%%%%%%% HBOA %%%%%%%%%%%%%%%%%%%%
\section{Hierarchical Bayesian Optimization Algorithm}

\cite{pelikan:2006:hboa} used statistical principles in combination with a decision tree structure
to create the Hierarchical Bayesian Optimization Algorithm (hBOA). This method creates a model of
epistatic relationships between genes which is then used to stochastically generate new solutions.
Each generation a binary tournament with replacement is used to select $\mu$ solutions from
the population. These solutions are then used to build the model, which in turn is used to generate $\mu$ new
solutions. The new solutions are then integrated into the population using restricted tournament
replacement.

Conceptually, the model built by hBOA is trying to infer rules of the form ``Given that this
subset of genes are set to these values, how frequently is gene $x_i$ set to value v?'' This can
be represented using a directed acyclic decision forest, with each tree in the forest representing one gene
in the solution. In the decision tree $T_i$, which is used to set the value of gene $x_i$,
each internal node represents previous decisions on how to set
some other gene $x_j$, with the children of that node representing how the decision was made. The
leaves of each tree give the probability that $x_i$ should be set to 
one of the possible gene values.

The forest is constructed iteratively, with each tree initially containing a single leaf
and with each leaf storing a pointer for each selected solution. Each iteration the algorithm considers
all possible ways of splitting an existing leaf using another gene $x_j$, such that solutions in the
leaf are moved to the newly created leaves based on their value for $x_j$. The general goal is to
separate the solutions such that all solutions with $x_i = 0$ move to one leaf while solutions with
$x_j = 1$ move to the other.

This goal is formalized using model scoring from Bayesian statistics. In its raw form this
almost always creates near infinitesimal results, calculating fractions that include factorials of $\mu$
and products over $N$. However, through algebraic manipulation discussed in Appendix~\ref{chap-appendix-hboa}, we
derived a simplified form shown in Equation~\ref{eq-hboa-final}.
Here $l$ is a leaf in tree $i$, with $l'$ and $l''$ the results of splitting $l$. $m_i(l)$ is the number
of solutions that reach $l$ and $m_i(x_i, l)$ is the number of solutions that reach $l$ with the given value
for $x_i$.
If no proposed split satisfies the inequality, iteration stops.
If multiple splits do, whichever maximizes the right side is chosen.

\begin{equation}
  2^{0.5 log_2\mu} < \frac{(m_i(l) + 1)!}{m_i(0, l)!m_i(1,l)!} \cdot
  \frac{m_i(0, l')!m_i(1,l')!m_i(0, l'')!m_i(1,l'')!}{(m_i(l') + 1)!(m_i(l'') + 1)!}
  \label{eq-hboa-final}
\end{equation}

Initially there are $\Theta(N^2)$ possible ways to split existing leaves, as each of the $N$ single node
trees can be split by any of the other $N-1$ genes. Each iteration a new edge is added to the decision
forest, meaning some of the previously tested splits cannot be used. For instance, if $T_i$, which is used
to decide the value of $x_i$, is split using the value of $x_j$, $T_j$ can no longer be split using $x_i$.
As a split creates two new leaves, \BigO{N} new potential splits must also be tested. Equation~\ref{eq-hboa-final}
parses all solutions that reach a leaf to count gene frequencies, requiring $\mu$ time.
The number of total leaves created depends heavily on the problem and $\mu$.
However, assuming no splits are accepted or that the cost of testing all future splits is less than
the initial $\Theta(N^2)$, constructing the model requires $\Omega(\mu N^2)$ time. Each
model is used to generate $\mu$ solutions, leading to a cost per evaluation of $\Omega(N^2)$.

To generate a solution, the value of each gene $x_i$ is set using its corresponding decision tree $T_i$. Because
the forest is directed acyclic, there must be an ordering of $T_i$
such that,
before $T_i$ is executed,
all
$x_j$ it uses to make decisions have already been set. As such, previous decisions made by other trees
are used to follow each $T_i$ until a leaf is reached. The value of $x_i$ is then set based on the
probability that other solutions reached that leaf with each value of $x_i$.

To perform replacement, hBOA uses restricted tournament replacement. After each new solution is generated
and evaluated, a set of $w$ solutions are chosen at random from the population, where $w=\min\{N, \frac{\mu}{20}\}$.
From this set the solution which is most genetically similar to the offspring is chosen. If the offspring
is at least as fit as the chosen solution, it replaces the chosen solution in the population. Otherwise the
offspring is discarded. This method is designed to preserve genetic diversity as only genetically similar
solutions compete on fitness.

hBOA is designed to work with large population sizes, resulting in a large number of
evaluations per generation. As hBOA utilizes explicit diversity maintenance, standard methods for determining
convergence are not considered very accurate. Therefore the authors suggest that an hBOA run should be
terminated after performing generations equal to $N$.

Like other model based techniques, hBOA has few parameters. There is no mutation or crossover,
and modeling does not rely on any explicit parameters. Solution selection, generation, and replacement
are all derived from the population size, which must be set by the user.

%%%%%%%%%%%%%%% Parameter-less HBOA %%%%%%%%%%%%%%%%%%%%%%
\section{Parameter-less hBOA}
Using the methods first introduced by \cite{harik:1999:parameterlessga} for the Parameter-less GA,
\cite{pelikan:2004:parameterlesshboa} created Parameter-less hBOA which automatically scales its
population size to fit the problem. This is done by maintaining a list of concurrent populations
using exponentially scaled population sizes.

A run of Parameter-less hBOA starts with a single population of size $\mu_0$, conventionally set
to $\mu_0=10$. After two generations are performed, a new population of size $\mu_1 = 2\mu_0$ is created
and performs a generation. Evolution then continues with the $\mu_0$ population performing two generations
for each one performed by $\mu_1$. Each time population $\mu_i$ performs its second generation a new population
$\mu_{i+1}=2\mu_i$ is created, which performs generations half as often as $\mu_i$. In this way an infinite number of
parallel population can be simulated, with each population receiving the same number of total evaluations.

In all other aspects each population is identical to an hBOA population using a fixed $\mu$. No search information
is shared among populations, and each search is independently terminated. As such Parameter-less hBOA cannot
perform better than hBOA using the optimal population size for a given instance, as it must also spend evaluations
on populations of different sizes. This inefficiency is bounded by a log multiple of the total number of
evaluations~\citep{pelikan:1999:worstparameter-less}.

\section{Linkage Tree Genetic Algorithm}
\cite{thierens:2010:ltga} introduced the Linkage Tree Genetic Algorithm (LTGA) which automatically
detects and exploits problem epistasis by examining pairwise gene entropy. Due to its enhanced
ability to preserve high fitness gene subsets, LTGA was able to outperform state-of-the-art
GAs across many benchmarks. Since its introduction, many variants of LTGA have been
proposed~\citep{thierens:2011:gomea,goldman:2012:ltga} so for clarity we have chosen the version
presented by \cite{thierens:2013:ltgahiff} as our model.

LTGA's effectiveness comes from its method of performing crossover. Instead of blindly
mixing genes between parents, LTGA attempts to preserve important interrelationships
between genes. Before performing any crossovers in a generation, LTGA first builds
a set of hierarchical gene clusters that are then used to dictate how genes are mixed
during crossover.

\begin{figure}
  \begin{algorithmic}[1]
  \Procedure{Cluster-Creation}{}
    \State $unmerged \leftarrow \{\{0\}, \{1\}, \{2\}, \dots, \{N-1\}\}$
    \State $useful \leftarrow unmerged$
    \While{$|unmerged|>1$}
      \State $C_i, C_j \leftarrow \min_{C_i,C_j \in unmerged} D(C_i, C_j)$
      \State $unmerged \leftarrow unmerged - \{C_i, C_j\} + \{C_i \cup C_j\}$
      \State $useful \leftarrow useful + \{C_i \cup C_j\}$
      \If{$D(C_i, C_j) = 0$}
        \State $useful \leftarrow useful - \{C_i, C_j\}$
      \EndIf
    \EndWhile
    \State Order $useful$ based based on last merged first\label{fig-cluster-creation-ordering}
    \State Remove largest cluster from $useful$

    \Return $useful$
  \EndProcedure
\end{algorithmic}
  \caption{Algorithm describing how LTGA creates clusters using Equation~\ref{eq-distance}
           for a population. $unmerged$ and $useful$ are ordered sets of sets of gene loci.}
  \label{fig-cluster-creation}
\end{figure}

Figure~\ref{fig-cluster-creation} provides the agglomerative method LTGA uses to create gene clusters.
This algorithm creates a tree of sets using pairwise gene entropy, such that the leaves of the tree contain a single gene and
internal nodes are the union of their children's sets. These sets are then used by crossover to specify epistatic
relationships that should be preserved.
The process begins by creating the set of sets $unmerged$ that tracks all top-level clusters. Initially
$unmerged$ contains single member sets for each gene. After each iteration the two sets with the minimum average pairwise
distance (given in Equation~\ref{eq-distance}) are merged to create a single cluster. This process is repeated
until only a single set remains in $unmerged$ which contains all of the genes in the genome.

\begin{equation}
  D(C_i,C_j) = \frac{1}{\left | C_i \right |\cdot \left |C_j \right|}\sum_{c_i \in C_i}\sum_{c_j \in C_j}
  2 - \frac{H(c_i) + H(c_j)}{H(c_i \cup c_j)}
  \label{eq-distance}
\end{equation}
\begin{equation}
  H(c) = -\sum_{s\in S} p_c(s)\log(p_c(s))
  \label{eq-entropy}
\end{equation}

Throughout this process $useful$ tracks the set of all gene clusters that should be preserved for use by crossover.
This set begins with all genes in separate clusters, and each time a new cluster is created it is added to $useful$.
However, not all clusters are necessarily worth keeping. For instance, in all versions of LTGA the cluster
containing all genes is removed from $useful$ as preserving all genes during crossover can only create clones.
\cite{thierens:2013:ltgahiff} extended this removal to include any unsupported subsets. If the pairwise distance
between two clusters is 0, this means there are no individuals in the population that disrupt the relationships between the two
clusters. Therefore,
when performing crossover,
there is no reason to believe a fitness improvement can be achieved
by breaking the stored pattern. As such a cluster is only kept if its direct superset has a non-zero distance.
As a final step,
Line~\ref{fig-cluster-creation-ordering} reorders $useful$ such that clusters appear in reversed
order from which they were added to $useful$. Thus the most linked clusters and those containing single genes
appear at the end of the returned list.

\cite{thierens:2013:ltgahiff}'s version of LTGA does not use the entire population when determining pairwise entropy.
Instead, binary tournament is used to select half of the population. This is done to ensure the model is built
using only high-quality solutions, even during the first generation.

In order to efficiently perform clustering, a pairwise gene frequency table is constructed from the selected solutions.
To calculate Equation~\ref{eq-distance}, Equation~\ref{eq-entropy} is called for each
gene ($H(c_i)$) and pair of genes ($H(c_i \cup c_j)$). Extracting this information requires \BigO{\mu N^2}
time, where $\mu$ is the population size and $N$ is the genome size. The process of converting
this pairwise frequency information into clusters can be achieved in \BigO{N^2} using the bookkeeping
methods presented by \cite{gronau:2007:upgma}. This cost is performed only once per generation,
and is then used to perform approximately \BigO{\mu N} crossover evaluations. As a result, the amortized cost of
LTGA's model building is \BigO{N}.

\begin{figure}
  \begin{algorithmic}[1]
  \Procedure{Cluster-Usage}{}
    \ForAll{$C_i \in useful$}
      %\ForAll{$d \in shuffled(P_i)$}
        \State $d \leftarrow rand\_choice(P)$\label{fig-cluster-usage-donate}
        \State Copy $d$'s gene values for $C_i$ into solution
        \If{solution was changed}
          \If{solution's fitness decreased}
            \State Revert changes
          \EndIf
          %\State \textbf{break}
        \EndIf
      %\EndFor
    \EndFor
  \EndProcedure
\end{algorithmic}
  \caption{Algorithm describing how clusters are used to perform crossover.}
  \label{fig-cluster-usage}
\end{figure}

Figure~\ref{fig-cluster-usage} describes how the identified clusters are used by crossover to preserve
gene linkage while still exploring the search space. Unlike more traditional crossover methods, LTGA
crosses each individual with the entire population. Also, to produce a single offspring, multiple evaluations
of the fitness function are performed.

During each
generation,
every individual in the population undergoes crossover. In a single crossover event, each
cluster of genes $C_i$ in $useful$ is applied as a crossover mask. A random donor $d$
is chosen from the entire population (not just the model selected population), and $d$'s gene's for $C_i$ are copied
into the working solution. If a modification is made, an evaluation is then performed. If the crossover
resulted in no worse fitness then the changes are kept,
% bwg: allowing -> which allows
which allows for neutral drift across plateaus.
The resulting solution, which must be at least as fit as its parent, is then
copied into the next generation.


In total each individual can cause up to $|useful|$ evaluations. If all clusters were kept, even those deemed
unhelpful, and all donations were evaluated, even those that did not change any genes, then \Call{Cluster-Usage}{}
would perform exactly $2N-2$ evaluations for each of the $\mu$ solutions in the population. This provides the amortizing evaluations
required to make clustering only \BigO{N} operations per evaluation. However, by skipping some evaluations, it is
possible that clustering may be super-linear.

LTGA has no explicit form of diversity control and has no method for introducing new genetic information once
the population has converged. Therefore an LTGA run is considered
converged
when two consecutive populations contain the same unique solutions.

By design, LTGA only has a single parameter: population size. LTGA uses no mutation, and crossover is defined
in terms of the clustering algorithm. Selection between generations is fully elitist and embedded in the crossover,
with selection of model building solutions fixed to a binary tournament. Neither \Call{Cluster-Creation}{} nor
\Call{Cluster-Usage}{} rely on parameter values. LTGA does not provide any method for controlling or setting
the population size, relying instead on a fixed user-specified size.

%%%%%%%%%%%%%%%% Tuning %%%%%%%%%%%%%%%%%%%%%
\section{Comparison Algorithm Parameter Tuning}
\label{sec-tuning}

\begin{figure}[t]
  \begin{centering}
    \begin{subfigure}{.5\textwidth}
      \begin{centering}
        \includegraphicsfit{pop-hboa}
      \end{centering}
      \caption{Tuned Population Sizes for hBOA}
      \label{fig-pop-hboa}
    \end{subfigure}%
    \begin{subfigure}{.5\textwidth}
      \begin{centering}
        \includegraphicsfit{pop-ltga}
      \end{centering}
      \caption{Tuned Population Sizes for LTGA}
      \label{fig-pop-ltga}
    \end{subfigure}
  \end{centering}
  \caption{Optimal population sizes found using bisection on each size
           for each problem.}
  \label{fig-pop-sizes}
\end{figure}

While four of the six algorithms in our experiments do not require any user-specified parameters,
hBOA and LTGA both use a population size parameter. To ensure these techniques are not unfairly
handicapped, we extensively tuned each using the bisection method \citep{sastry:2001:bisection}
to determine the optimal population size for each problem size.
Extended by \cite{goldman:2012:ltga}, this method iteratively
doubles the population size until some success criteria is met and then performs bisection
between the 
smallest successful and the % bwg: added "the"
largest unsuccessful sizes. In this way the minimum population size
that meets the success criteria is found. \cite{goldman:2014:p3} proposed a success criteria
of performing $r$ successful runs in a row, such that the expected failure rate can be bounded
above by $\frac{3}{r+1}$~\citep{jovanovic:1997:ruleofthree}. As P3 and the other three algorithms
do not prematurely converge, we chose $r=100$ to similarly ensure hBOA and LTGA almost never do.
As bisection can make infinitely large population sizes, any run that had not found the global
optimum after 100 million evaluations or 128 computing hours was considered unsuccessful.

% bwg: reworded slightly, changed chapter reference
Figure~\ref{fig-pop-sizes} shows the results from performing bisection
on all problems to be used as comparison benchmarks in
Chapter~\ref{chap-benchmarking}.
In general hBOA
required population sizes that were at least an order of magnitude larger than LTGA. Due to
runtime and memory overhead, finding the optimal value for hBOA was
much less tractable than 
for
LTGA
on moderate to large problem sizes. LTGA's population size also grew significantly slower
than hBOA's as problem size increased, especially on the two Trap problems and Rastrigin.
Both algorithms were ineffective on MAX-SAT, with neither able to tune to problems sizes over 60 bits.
This is likely due to the fact that some randomly generated MAX-SAT landscapes are quite flat
and highly deceptive~\citep{rana:1998:gamaxsat}.

While not currently treated as a parameter, we also performed preliminary tests of
integrating hill climbing into LTGA and hBOA
as this is the procedure used in P3.
To match P3, we applied first-improvement
hill climbing to each algorithm's initial population.
We then performed bisection on the modified algorithms for the largest problem sizes
where hBOA without hill climbing was effective. We found that in general both methods
performed worse when combined with hill climbing, in some cases up to an order of
magnitude worse. There were three exceptions: both improved on MAX-SAT and
hBOA improved on Rastrigin. In all cases the inclusion of hill climbing did
not result in either algorithm outperforming P3 in terms of evaluations
required to reach the global optimum. As such, all further experiments
use the unmodified,
published versions.
