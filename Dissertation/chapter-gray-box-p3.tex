\chapter{Gray-Box P3}
P3 represents a natural method for integrating the HBHC into
a global optimization algorithm as P3 already utilizes local search.
While HBHC's inclusion adds a parameter, Section~\ref{sec-radius} and
Section~\ref{sec-over-time} provide evidence that $r$ can be fixed to 1, preserving
the parameter-less nature of P3.
Beyond using HBHC, there
are also a number of ways in which P3 can be made more efficient by leveraging the
additional information available in the gray-box domain.

First and foremost, linkage learning is no longer necessary. By definition
the direct non-linear relationships between variables are known. As a result
P3 no longer needs to store the pairwise frequency information which leads
the black-box version to require \BigO{N^2} memory. Instead,
we have developed a method for creating a linkage tree which learns clusters
from the same dependency graph defined in Section~\ref{sec-hamming}. The goal is for each
cluster to be a connected induced subgraph, with the size of clusters
mirroring those produced by the agglomerative linkage learning process normally
used with P3. To form a single cluster, a random graph search is performed from
a random starting vertex until a desired number of unique vertices have been explored.
Cluster sizes are set recursively. For each cluster of size $l>1$
a cluster of size $a$ and a cluster of size $l-a$ are also created, with $a$ chosen uniformly
from the range $[1..l-1]$. This recursive process begins by initializing $l=N$ and splitting
to form the first two clusters.

This linking algorithm has a number of useful properties. First, it creates exactly
$2N-2$ clusters, distributed in size similarly to the black-box clustering algorithm.
Performing a random graph search to find $l$ unique vertices requires \BigO{lck} time,
meaning cluster creation is optimally efficient.
The cluster splitting process has identical properties as random pivot quicksort,
meaning the sum of cluster sizes is \BigO{N\log N} in the average case.
This efficiency allows new clusters to be created before every mixing event,
unlike Black-Box P3 where they are created only when new solutions are added to the population.
For simplicity the clusters are shuffled after they are created, not sorted on size like in
Black-Box P3.

Beyond efficiency, there are good reasons to believe these clusters will be useful
for search. The closer two variables are in the dependency graph, the more likely
they are to appear in the same cluster. All variables on average are expected to appear
in at least one cluster, but variables which are central in the graph will appear in more
clusters than those on the periphery. The more paths of a given
length between two variables, the higher the probability of them being in the same cluster.
Unlike Black-Box P3, this linkage tree does not require clusters to be nested, allowing
more diversity in the types of clusters appearing in a single tree.

In effect the clusters are sampling moves which the HBHC
would make if $r \ge l$. For any solution in the search space which is not globally optimal, there
must exist some move which will improve its quality. However, this move may
be arbitrarily large and it is intractable to test all possible moves of even moderate size.
By sampling from all possible large moves, we can maintain tractability while gaining
a potentially non-zero probability of improvement.
As clusters are used to donate values between solutions, these
moves are always in the direction of previously found high quality solutions.
This assumes that the density of high quality solutions is higher than average between good solutions.

Another efficiency gain possible due to the gray-box domain is that each time a
donation is made, only the effected part of the solution's fitness needs to be
recalculated. As a result the number of subfunction evaluations required
to determine the change in fitness is only \BigO{l}. This also
allows for Gray-Box P3 to efficiently reapply hill climbing after each
donation as only effected moves need to be rechecked. As a result, a donation
and its resulting modifications from hill climbing are kept only if the
new local optima is at least as fit as the solution before the donation occurred.
All combined a single donation plus returning to a local optimum requires \BigO{l + I} time,
while just the donation in Black-Box P3 requires \BigO{N}.

