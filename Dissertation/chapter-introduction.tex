\chapter{Introduction}
\label{chap-introduction}
A primary purpose of evolutionary optimization is to efficiently find good solutions
to challenging real world problems with minimal prior knowledge about the problem itself.
This driving goal has created search algorithms that can escape user bias to create
truly novel results, sometimes publishable or patentable in their own right~\cite{kannappan:2014:humies}.
While it is not possible for any algorithm to do better than random search across all possible
problems~\cite{Wolpert:1997:nfl}, effectiveness can be achieved by assuming the search
landscape has structure and then biasing the algorithm toward exploiting that structure.

In evolutionary optimization, and genetic algorithms~(GAs) in particular, search is often
biased through parameters. This can be beneficial as it allows practitioners to inject their
knowledge about the shape of the search landscape into the algorithm.
However, the quality of solutions found, and the speed at which they are found, is strongly tied to setting these parameters
correctly~\cite{goldberg:1991:gasize}. As such, either expert
knowledge or 
expensive parameter tuning~\cite{grefenstette:1986:optimalga} are required to leverage
this feature to its fullest potential. Furthermore,  parameters such as population size, mutation rate, crossover
rate, tournament size, etc. usually have no clear relationship to the problem being solved, meaning even
domain experts may not understand how the parameters will interact with the problem or with each other.
To further complicate matters, there is mounting evidence that parameter values should change
during search~\cite{goldman:2011:dynamic-parameters,laporte:2014:adaptivepop}.

There have been periodic efforts to reduce or remove the need for parameter tuning.
\cite{rechenberg:1973:es} introduced self-adaptive parameters, in which parameter values
were included in each solution's genome and themselves underwent
evolution. This allowed 
the search process itself 
to optimize some of its own parameters, resulting in a reduced need for expert tuning.
\cite{harik:1999:parameterlessga} was able to design an entirely parameter-less GA by
leveraging schema theory and parallel populations. Unfortunately these methods were provably less efficient
than directly setting the parameters to optimal values~\cite{pelikan:1999:worstparameter-less}.

One area that has been effective at reducing the number of algorithm parameters is
model based search. \cite{pelikan:2006:hboa}'s Hierarchical Bayesian Optimization
Algorithm~(hBOA) and \cite{thierens:2010:ltga}'s Linkage Tree Genetic Algorithm~(LTGA)
both require only a single parameter: population size. \cite{posik:2011:parameterless}
leveraged model building to create a fully parameter-less algorithm, but it is restricted to
only order-k, fully decomposable, noiseless problems.

Most recently we introduced the Parameter-less Population Pyramid~(P3)~\cite{goldman:2014:p3}.
This method uses a pyramid structure of populations to combine model based search with local search
to achieve parameter-less optimization. Initial results suggest that,
unlike
previous parameter-less methods, P3 is 
more efficient than current state-of-the-art
parameterized search algorithms. In this work we shall: extend these results to cover more
comparison algorithms; compare both efficiency in reaching the global optimum and intermediate
fitnesses; analyze algorithm complexity; and provide more in depth analysis of P3 itself.

