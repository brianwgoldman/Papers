\chapter{hBOA Simplification}
\label{chap-appendix-hboa}
To measure the quality of a decision forest, hBOA applies Equation~\ref{eq-hboa}.
To favor compact
models, Equation~\ref{eq-hboa} is scaled by Equation~\ref{eq-hboa-stop}, which provides increased cost
for more total leaves.
This quality is used for two purposes: comparison of potential changes from the existing model
and comparison of that change with the existing model. The basis for our simplication
is to rearrange $p(B)BDe(B) < p(B')BDe(B')$ to be $\frac{p(B)}{p(B')}<\frac{BDe(B')}{BDe(B)}$.

\begin{equation}
  BDe(B) = \prod_{i=0}^{N}\prod_{l\in L_i} \frac{\Gamma(m'_i(l))}{\Gamma(m_i(l) + m'_i(l))}
  \prod_{x_i}\frac{\Gamma(m_i(x_i, l) + m'_i(x_i,l))}{\Gamma(m'_i(x_i,l))}
  \label{eq-hboa}
\end{equation}

\begin{equation}
  p(B) = c2^{-0.5(\sum_i|L_i|)log_2\mu}
  \label{eq-hboa-stop}
\end{equation}

The outermost product of Equation~\ref{eq-hboa} iterates over all trees in the forest. However, each split
can modify only one of the trees and therefore the contribution of all others can be canceled. The middle
product is across all leaves in the tree. Again since only one leaf can be changed, all other terms can
be canceled. By convention hBOA uses uninformed Bayesian priors of $m'_i(l)= 2$ and $m'_i(x_i, l)=1$ for
binary alphabets. As $\Gamma(a) = (a-1)!$ this means the top term in the middle product and the bottom
term in the third product reduce to 1. The only remaining terms are then $m_i(l)$ and $m_i(x_i, l)$ which
represent the number of solutions that reached leaf $l$ and the number of solutions that reached leaf $l$
with a specific value for $x_i$, respectively.

Equation~\ref{eq-hboa-stop} can also be simplified when doing comparisons. If model $B'$ has exactly one more
leaf than model $B$ then the ratio $\frac{p(B)}{p(B')}$ simplifies to $2^{0.5 log_2\mu}$ regardless of
total model size.

The resulting simplifications create Equation~\ref{eq-hboa-final},
where $B'$ is different from $B$ by exactly 1 split, such that $l$ was split to create $l'$ and $l''$.
The best split is whichever maximizes its improvement over $B$, which is equal to the right side
of the inequality. Note that these factorials can still be exceedingly large and
therefore it is imperative that implementations avoid rounding errors and overflows.
