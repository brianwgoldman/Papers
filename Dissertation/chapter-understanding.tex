\chapter{Understanding When P3 Excels}
The ruggedness and high dimensionality of most interesting landscapes makes them challenging
to visualize or otherwise understand. However, doing so can be helpful in
understanding the difficulty of a problem, and how best to design algorithms to deal
with those difficulties. Conversely, understanding the characteristics which favor
existing algorithms can help future researchers choose the algorithm which is
most likely to perform well on their problem. To further this end, we shall
explore the landscapes used in previous sections in an attempt to understand
what makes a landscape easy for P3 to solve.


\section{Big Valley}
One method for understanding the global structure of a landscape
is to examine the relationship between local optima~\cite{boese:1994:bigvalley}.
In its original form, this process involved generating thousands of random solutions and then applying
local search to each. To convert this information into something human interpretable,
the authors created two dimensional plots: distance from nearest global optima and
fitness difference from global optima.
The authors found that for a number of interesting problems a ``Big Valley'' of
local optima exist, such that the higher a solution's fitness the closer it is
in representation space to the global optimum. As a result, this suggests that
by focusing search around known high quality solutions, the likelihood of finding even
higher quality solutions is increased. This relationship is an underlying assumption
of all evolutionary computation based search methods, including P3.

We have set out to extend this method of landscape visualization by considering not
just a sample of local optima, but all local and global optima in the landscape.
When considered as a black-box, each solution must be enumerated and all of its
neighbors must be checked for improvements. This enumeration is prohibitively expensive for
even trivial landscapes, requiring $\Omega(N2^N)$ operations.
However, by leveraging the gray-box domain, some efficiency improvements can be made.

\subsection{Quickly Finding All Local Optima}

Consider that for gray-box problems, we can determine all improving moves from a given
solution in \BigO{N} time and that this list can be updated in \BigO{1} time when flipping
a single bit. As a direct result, checking if each new solution is locally optimal requires
only \BigO{1} time and the overall enumeration process is reduced to \BigO{2^N}.

Due to the limited non-linearity of the gray-box domain, its possible to exclude large
parts of the search space without potentially missing any local optima. Consider a bit flip $X_i$.
For gray-box problems there are only $ck$ other bit values which are needed to determine if flipping
$X_i$ is a fitness improvement. If a given set of values for $X_i$ and those $ck$ bits results in
$X_i$ being a fitness improvement, then any solution which contains all of those values cannot be a local optima.

This knowledge can be effectively exploited by intelligently skipping parts of the enumeration.
Before starting search we can determine the least significant bit in the solution which effects
each possible fitness improving move. This is the bit which must be flipped in order to effect
the quality of that move. As a result, when determine what solution to test next, all steps
of the enumeration which do not result in the least significant bit of a fitness improve being
flipped can be skipped. Furthermore, if there are multiple fitness improving moves, all must
have at least one dependent bit flipped, which is the same as saying the fitness improving move
with the most-significant least-significant bit must have its least significant bit flipped.


Finding which move's least significant bit needs to be flipped can also be done efficiently via binning.
Before starting search, each move can be binned based on its least significant bit. Initially
search must check from the most significant bin down to find the first fitness improving move.
This is the position from which enumeration must continue. If no move is fitness improving, a local
optima has been found, and enumeration can continue with the least significant bit in the solution.
Enumeration progresses by adding a 1 to the found position, with carry over as needed.
If the carry ever exceeds the most significant bit in the solution, the entire space has been checked and iteration ends.

When performing subsequent checks, not all moves needs to be retested for improvement. Instead, the highest
bin which must be tested is the location of the most significant bit effected by the previous iteration. This is because
the previous iteration has verified all higher moves are not fitness improving, and no action performed during
that iteration can change that. Furthermore, no 1s can exist lower than that position, so bits
below the least significant bit of the fitness improving move can be ignored.

As a final efficiency, the order in which bits of the solution are considered can be remapped. When a flip is fitness
improving, the amount of search space which is skipped depends on how significant its least significant bit is. Therefore,
by rearranging the order to make its least significant bit more significant, more search space can be skipped. We perform
this remapping in a greedy fashion, such that the flip with the least unmapped dependencies has all of its remaining dependencies
mapped to the most significant remaining positions.

All together, this can result in significant efficiency improvements for some landscapes. For instance, when applied to
the OneMax problem (each bit scores one if set to 1, zero otherwise) this method finds all local optima in \BigO{N} time.
Deceptive Trap, regardless of how the bits are arranged, requires \BigO{2^k2^{N/k}} time to find all $2^{N/k}$ local optima.
As an example, this means a 60 bit
Deceptive Trap problem with trap size of 4 requires 524,288 operations when using gray-box, but 1,152,921,504,606,846,976
with black-box.

As a final note, these methods extend to finding only $r$-bit local optima for all gray-box problems. However,
the cost trade-off of doing so is somewhat unclear. By searching only for $r$-bit local optima, it is more likely
that a move will exist which causes part of the landscape to be skipped. Yet the increase in total number of moves
requires more updating information each time the solution is changed. The exact relationship between the two
depends on the specific problem.

\subsection{Looking at Problems}
In comparison to the randomly generated problem instances of Nearest Neighbor NKQ and Ising Spin Glass,
Deceptive Trap and Deceptive Step Trap have comprehensible landscapes. Therefore, these problems represent
a good place to begin understanding what ``big valley'' plots are showing. Figure~\ref{fig-valley-trap} shows
how the local optima in each of these problems are distributed.

\begin{figure}[t]
  \begin{centering}
    \begin{subfigure}{.44\textwidth}
      \begin{centering}
        \includegraphicsfit{valley-dt}
      \end{centering}
      \caption{Deceptive Trap}
      \label{fig-valley-dt}
    \end{subfigure}%
    \begin{subfigure}{.56\textwidth}
      \begin{centering}
        \includegraphicsfit{valley-dst}
      \end{centering}
      \caption{Deceptive Step Trap}
      \label{fig-valley-dst}
    \end{subfigure}
  \end{centering}
  \caption{Location and quality of local optima in comparison to the global optima with $N=30$ and $k=5$.}
  \label{fig-valley-trap}
\end{figure}

The Deceptive Trap problem has few local optima, each of which actually
provide little deception in terms of ``big valley'' properties. Given any two local optima, the one
closer to the global optimum in fitness is also always closer to the global optimum in representation space.
This makes sense as the worst local optimum contains all 0s, and each fitness improvement beyond that involves
converting an entire trap to 1s, with the global optimum being all 1s. As a result, we should expect P3's method
of elitist mixing to produce solutions which are progressively more and more similar to the global optimum until
it is finally found.

Deceptive Step Trap's inclusion of fitness plateaus creates an enormous number of local optima. In total there are 24
million local optima on this problem, representing 2.2\% of the entire search space. In comparison, Deceptive Trap using the
same size has only 64 local optima. These additional local optima create a somewhat deceptive landscape, as selecting
on fitness between two local optima can result in an increased genetic distance to the global optimum.
Figure~\ref{fig-valley-dst} with $r=2$ shows how examining only 2-bit local optima causes Deceptive Step Trap using
a step size of 2 to effectively revert to Deceptive Trap. This is because the hill climber can overcome all of the plateaus,
leaving each trap either at the global optimum of all 1s or the local optimal value of all 0s or exactly one 1. Selection then
becomes a near perfect predictor of distance to the global optimum.

\begin{figure*}
  \centering
  \includegraphicsfit{valley-nn}
  \caption{Location and quality of local optima in comparison to the global optima for a representative Nearest Neighbor NKQ problem
           with $N=60$ and $k=2$.}
  \label{fig-valley-nn}
\end{figure*}

Figure~\ref{fig-valley-nn} shows how the local optima are distributed for Nearest Neighbor NKQ. Unlike the trap problems,
Nearest Neighbor NKQ exhibits a traditional big valley shape. Increasing the radius of the local optima significantly
reduces the total number of local optima, with those optima generally more similar to the global optima in both representation
space and quality. Together this suggests that Nearest Neighbor NKQ is a good candidate for selection based methods like P3. Also,
due to the general regularity and frequency of local optima, it may not be necessary to increase the radius in order to quickly find
the global optimum.

\begin{figure*}
  \centering
  \includegraphicsfit{valley-un}
  \caption{Location and quality of local optima in comparison to the global optima for a representative Unrestricted NKQ problem
           with $N=60$ and $k=2$.}
  \label{fig-valley-un}
\end{figure*}

While visually somewhat similar, Unrestricted NKQ shown in Figure~\ref{fig-valley-nn} suggests selection may be misleading.
Consider the shape of the bottom of $r=1$. At a distance of about 15 bits there is a slight hump, such that high quality solutions
are more frequent both closer and further from the global optimum than at the same distance.
If search finds points in the area over 15 bits away from the global optimum, small modifications and elitist selection
are more likely to lead away from the global optimum that toward it.
At higher radius values this issue becomes more apparent, with optima flattening out away
from the global optima in representation space. For many of these points, the only way to improve would require flipping over 15 bits
correctly.

\begin{figure*}
  \centering
  \includegraphicsfit{valley-is}
  \caption{Location and quality of local optima in comparison to the global optima for a representative Ising Spin Glass problem
           with $N=36$.}
  \label{fig-valley-is}
\end{figure*}

The Ising Spin Glass instance shown in Figure~\ref{fig-valley-nn} provide some insight into the stalled behavior of Gray-Box
P3 when solving that problem. In this landscape there are many local optima which have the second best fitness, but are very
different from the nearest global optima. From a search perspective this means selection can only get you within a certain distance
of a global optima, and then it becomes unhelpful. This issue is not improved by increasing the local optima radius, as even with
$r=4$ there are hundreds of local optima with the second best fitness in the landscape which are over 10 bits different (of 36) from
the nearest global optima.

\begin{figure*}
  \centering
  \includegraphicsfit{valley-sat}
  \caption{Location and quality of local optima in comparison to the global optima for a representative MAX-SAT problem
           with $N=36$.}
  \label{fig-valley-sat}
\end{figure*}

Similar to Unrestricted NKQ, the MAXSAT instance shown in Figure~\ref{fig-valley-sat} suggests selection can be misleading.
There is almost a negative correlation between fitness difference from the global optimum and representation distance from the
global optimum. As before, this problem is not solved by increasing the radius of the local optima.
This may explain why methods like LTGA and hBOA, which extensively rely on selection,
scale so poorly when tested on this problem. P3's comparative success may therefore be the result of the random restart hill climber
eventually finding a solution in the correct area of the search space, with crossover finding the global optimum mostly by chance.

\section{Pyramid Levels}
One method for understanding how P3 performs search is to examine the types of solutions being stored at each level
of the pyramid. By comparing each solution with the nearest best found solution, we can create plots similar to the ``big valley''
ones in the previous section, even for problems too large and complex to find all possible global optima. 
This also provides a look at how the different levels of the pyramid focus search on different areas of the landscape.

\begin{figure*}
  \centering
  \includegraphicsfit{level-dst}
  \caption{Distribution of local optima stored at each level of Gray-Box P3 in relation to the global optimum on
           the Deceptive Step Trap problem $N=6000$ and traps of size 5.}
  \label{fig-level-dst}
\end{figure*}

Gray-Box P3's progression on Deceptive Step Trap, shown in Figure~\ref{fig-level-dst}, follows directly from our expectations.
The local optima found using only hill climbing and stored in level 0 are quite poor and far from the global optima.
The first application of crossover dramatically improves their quality. However, this improvement is likely just overcoming
the fitness plateaus, as solutions are actually being moved away from the global optimum. Subsequent levels store solutions
which are generally closer to the global optimum in both fitness and representation space. This change means that at higher
levels the frequency of traps being set to the global optima increases, resulting in better model building and better donation quality.

\begin{figure*}
  \centering
  \includegraphicsfit{level-nn}
  \caption{Distribution of local optima stored at each level of Gray-Box P3 in relation to the best found by the run on
           a Nearest Neighbor NKQ problem $N=6000$ and $K=4$.}
  \label{fig-level-nn}
\end{figure*}

Nearest Neighbor NKQ, shown in Figure~\ref{fig-level-nn}, provides a landscape almost as free of higher order deception
as the Deceptive Step Trap problem.  Shown with a logarithmic y-axis, stored solutions exhibit a near perfect relationship
between fitness and representational distance. Again, using only hill climbing creates local optima which are significantly
worse and further away than even
after a single application of crossover. While increasing the radius of the hill climber improves the quality of those initial
solutions, even $r=3$ is only able to equal a single crossover application. As before the higher level a solution is stored,
generally the closer it is to the global optima in both fitness and representational distance.

\begin{figure*}
  \centering
  \includegraphicsfit{level-un}
  \caption{Distribution of local optima stored at each level of Gray-Box P3 in relation to the best found by the run on
           an Unrestricted NKQ problem $N=6000$ and $K=4$.}
  \label{fig-level-un}
\end{figure*}

Figure~\ref{fig-level-un} gives insight into how different Unrestricted NKQ is from Nearest Neighbor NKQ. In this landscape
Gray-Box P3, regardless of radius, appears to explore deceptive local optima which offer fitness improvements without moving
closer to the global optima in representation space. In this problem there is a gap of over 1000 bits between the deceptive
local optima and those likely to lead to the best found solution. An interesting behavior on this problem is that the branch
between the two groups always seems to begin with a single solution in level 1. This suggests that the stored solutions near
the best found may all be similar due to sharing a common ancestor.

\begin{figure*}
  \centering
  \includegraphicsfit{level-is}
  \caption{Distribution of local optima stored at each level of Gray-Box P3 in relation to the best found by the run on an Ising Spin Glass
           $N=6084$.}
  \label{fig-level-is}
\end{figure*}

Gray-Box P3's progression on Ising Spin Glass, shown in Figure~\ref{fig-level-is}, is much more well behaved than on Unrestricted
NKQ. As has been typical, each application of crossover results in improved fitness, with the largest gains between the first few levels.
When approaching higher qualities these improvements also begin to translate into increased representational similarity to the global optimum.
However, as discussed previously for this problem, we again see evidence for a large number of diverse solutions with the second best
fitness. These solutions can be over 1000 bits different from the eventual best found solution. Increasing the radius of the hill climber
does not seem to significantly overcome this issue.

\begin{figure*}
  \centering
  \includegraphicsfit{level-sat}
  \caption{Distribution of local optima stored at each level of Gray-Box P3 in relation to the best found by the run on
           a MAX-SAT problem $N=6000$.}
  \label{fig-level-sat}
\end{figure*}

Figure~\ref{fig-level-sat} shows that Gray-Box P3 acts more similarly on MAX-SAT to Unrestricted NKQ than any of the other problems. Again there
appears to be deceptive local optima between 500 and 1000 bits away from the nearest improving solution. Again a single solution appears
to be the ancestor of all of the solutions near the best found. This suggests that similar to Unrestricted NKQ, Gray-Box P3 will have a
hard time finding the global optimum on MAX-SAT.

In general it appears that the problems which have polynomial time solutions (Deceptive Step Trap, Nearest Neighbor NKQ, 2D Ising Spin Glass)
share a similar behavior of selection leading to the global optimum. However, the NP-Hard problems (Unrestricted NKQ, MAX-SAT) seem to contain
large amounts of higher order deception, making them challenging for selection based methods. This relationship warrants further investigation
into other NP-Hard problems to determine if the repeated selection found in P3 and other evolutionary computation methods are able to overcome
this deception. One potential avenue for improvement to P3 may also be a method for dealing with this issue. For instance, it may be beneficial to
explicitly partition solutions when this deceptive behavior is observed.

