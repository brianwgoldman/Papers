\chapter{Conclusions and Future Work}
The Parameter-less Population Pyramid (P3) is a recently introduced method for performing black box
optimization. P3's primary innovation is the replacement of the generational model with a pyramid of populations.
This pyramid is constructed iteratively, with both the number of levels and the number of solutions stored
at each level growing as search progresses. P3 uses a model based crossover method
which learns a linkage tree from gene epistasis. Combined with a simple hill climber, P3's design contains
many synergistic features.

Across a large number of problems and problem sizes P3 required less evaluations to reach the global optimum
than optimally tuned state-of-the-art competitors. On single instance problems P3's improvement was by a
constant factor, while for the three problem classes P3's improvement increased with problem size. This
quality extends to intermediate points during evolution, with P3 generally reaching at least as high
of fitness as the competitive techniques when using the same number of evaluations. While P3 does require
modeling overhead, the expense of this overhead is approximately linear with respect to genome size. There
is some evidence that even when compared on wall clock time, P3 performs at least as well as
the best comparison techniques. All of these achievements are made without any problem specific parameter
tuning, making P3 easier to apply to new domains than its two closets competitors in quality.

P3's quality is due to a number of desirable traits. First, mixing local search with model based
crossover lets search focus on properly mixing high quality solutions. Second, by adding diversity only
as necessary P3 tends to use the minimal amount of random initialization, unlike other techniques which must
overcompensate with larger population sizes on single instance problems and consider the worst instance
when solving problem classes. Third, by heavily exploiting existing diversity before adding more P3 is able
to reach high quality intermediate fitnesses quickly without prematurely converging. Fourth, the very
nature of the pyramid's shape allows search to preserve a desirable proportion of diversity at
each fitness level, similar to a generational model using a decreasing population size.

There are a number of meaningful avenues for future P3 experimentation. Perhaps the most
pressing for practitioner acceptance is to apply P3 to real world problems and compare
its results with other black box or even problem specific heuristics. While parameter-less,
P3 is currently limited to discrete, fixed length genomes evaluated using single objective
fitness. These limitations can be relaxed with future work to make P3 more widely applicable.
While asymptotically linear, P3's modeling techniques and local search methods are likely going
to be prohibitively expensive for genome sizes in the hundreds of thousands or millions of genes,
and the inability of the model to capture overlapping linkage may be hindering
search efficiency. Overcoming these limitations by using a new modeling technique may allow
the pyramid model even greater flexibility. Similarly, while P3 is able to overcome low
order deception via linkage learning, the iterative improvement method by which crossovers
are made may mislead search on landscapes with higher order deception.

However, even without
these improvements our results show P3 is highly efficient at finding
global optima on black box problems without any problem specific tuning.

