\chapter{Internal Workings}

%%%%%%%%%%%%% Population Sizing %%%%%%%%%%%%%
\section{Population Sizing}
A major advantage to P3 is that it does not require the user to set a population size parameter.
Beyond making P3 easier to apply, this also conveys two additional advantages: diversity scaled
to initialization and no need to sacrifice intermediate fitness for eventual optimality.

\subsection{Problem Instance versus Problem Class}
Our procedure for tuning LTGA and hBOA outlined in Section~\ref{sec-tuning} involved finding
the optimal population size for each class of problem. For real world black box optimization this is
realistically the best either algorithm could hope for as tuning to a problem
instance or population initialization involves repeatedly solving the problem being tuned.
This limitation does not exist in parameter-less methods, which scale their diversity
based on the problem instance without needing to solve that instance repeatedly.

To achieve high success rates on randomly generated problem classes, LTGA and hBOA must use a population size
large enough to solve the hardest instances in that class. Therefore these methods will have
population sizes larger than necessary to solve the easiest instances in the class. Even
on single instance problems, both methods will require population sizes large enough to
ensure the worst random initialization is diverse enough to solve the problem, which may be
much larger than the best random initialization.

\begin{figure}
  \begin{center}
  \includegraphicsfit{evals-to-success-range}
  \end{center}
  \caption{Comparison of the upper and lower quartiles of evaluations required
           to reach the global optimum for P3 and LTGA with respect to problem size.}
  \label{fig-evals-to-success-range}
\end{figure}


Figure~\ref{fig-evals-to-success-range}
highlights how this can effect the required number of evaluations to reach the global optimum, showing
the upper and lower quartiles for P3 and LTGA. On each problem
LTGA has a much smaller difference between its best and worst runs. This makes sense as LTGA uses the same
population size regardless of instance and progresses search generationally. In contrast P3 has a much
higher split, with many runs finishing very quickly. This added variability is also generally on the lower
quartile, with P3's upper quartile only worse than LTGA's on Deceptive Step Trap and Nearest
Neighbor NK. This supports the hypothesis that P3 is able to scale its diversity not just
to the problem class, but to the problem instance or even problem initialization, something
wholly infeasible for tuned population sizing to do.

This tuning distinction is also apparent when comparing Parameter-less hBOA with hBOA in Figure~\ref{fig-evals-to-success}.
While generally performing worse than hBOA, the difference between the two algorithms is smallest
on randomly generated problem classes. On MAX-SAT, Parameter-less hBOA actually outperformed both hBOA and LTGA,
likely due to its ability to scale diversity to the problem instance instead of the entire problem class.

\subsection{Fast versus Optimal}
\label{sec-fast-vs-optimal}

In Section~\ref{sec-overtime} we examined intermediate fitness qualities of LTGA and hBOA when using
population sizes tuned to reach the global optimum. As a result, both were exceptionally ineffective
at quickly reaching high quality solutions. This is because unlike P3, these methods have an explicit
trade off between optimal performance and intermediate performance caused by their population size parameter.

At 58\% of the recording intervals hBOA has the worst fitness of any solver. Most of the exceptions
occur when hBOA is still evaluating its initial population, allowing this random search to temporarily surpass the
local search methods. After $N$ evaluations, however, hBOA and LTGA both fall behind until their models begin
to improve.
Parameter-less hBOA reaches intermediate fitnesses faster than hBOA, doing so on 62\% of intervals, as its models begin
to optimize earlier than hBOA.  However, this trend is reversed after a sufficient number of evaluations, most clearly
on Deceptive Step Trap and Ising Spin Glasses, as hBOA's tuned population overtakes Parameter-less hBOA's parallel populations.


\begin{figure}[t]
  \begin{centering}
    \begin{subfigure}{.5\textwidth}
      \begin{centering}
        \includegraphicsfit{small-pop-dst}
      \end{centering}
      \caption{Deceptive Step Trap 203}
      \label{fig-small-pop-dst}
    \end{subfigure}%
    \begin{subfigure}{.5\textwidth}
      \begin{centering}
        \includegraphicsfit{small-pop-nk}
      \end{centering}
      \caption{Nearest Neighbor NK 200}
      \label{fig-small-pop-nk}
    \end{subfigure}
  \end{centering}
  \caption{Comparison of how reducing LTGA's population size effects
           the median best fitness reached during search.}
  \label{fig-small-pop}
\end{figure}

Figure~\ref{fig-small-pop} examines the effect of population size on LTGA's intermediate fitness by examining the effect
of reducing LTGA's population size to one tenth of the tuned value. The two problems shown are representative of the behavior of using
a smaller population size on the other five problems. Reducing the population size caused LTGA to improve earlier but plateau
at lower fitnesses. This caused LTGA's success rate to drop from 100 to 0 on Deceptive Step Trap and from 98 to 68 on Nearest
Neighbor NK.
Even when using a reduced population size for LTGA, P3 still achieved a fitness at
least as high as LTGA at 80\% of intervals.

%%%%%%%%%%%%%% P3 Inner Workings %%%%%%%%%%%%%%%%%%%%%%%
\section{Inner Workings Specific to P3}

\begin{figure}[t]
  \begin{centering}
    \begin{subfigure}{.5\textwidth}
      \begin{centering}
        \includegraphicsfit{cross}
      \end{centering}
      \caption{Crossover Proportion}
      \label{fig-cross}
    \end{subfigure}%
    \begin{subfigure}{.5\textwidth}
      \begin{centering}
        \includegraphicsfit{cross-success}
      \end{centering}
      \caption{Crossover Success}
      \label{fig-cross-success}
    \end{subfigure}
  \end{centering}
  \caption{For each problem Figure~\ref{fig-cross} shows the proportion of P3 evaluations spend on crossovers
           and Figure~\ref{fig-cross-success} shows the percentage of fitness improving crossover evaluations.}
\end{figure}

While analysis of optimization speed is useful from a practitioner standpoint, doing so provides very
little insight into algorithm behavior. To better understand how P3 works in detail we present here
a look at some internal features specific to P3.

\subsection{Crossover}
Figure~\ref{fig-cross} shows the proportion of evaluations P3 spends on crossover, as opposed to hill climbing,
and Figure~\ref{fig-cross-success} shows what percentage of crossover evaluations resulted in a fitness improvement.
Together these figures provide some insight into the role of crossover within P3. The behavior
for each is clearly problem dependent and generally asymptoticly stable as problem size increases.

When solving problems where epistasis can be effectively detected and represented by a linkage tree, P3 tends to spend
less evaluations performing crossover and each crossover is more likely to be successful.
Deceptive Trap and Rastrigin are the easiest problems to model epistasis, with local search quickly
reducing pairwise entropy in each. These are also the problems where P3 uses the least evaluations
on crossover and has the highest success rates for crossover. At the other extreme are Nearest Neighbor
NK and Ising Spin Glasses, which both have overlapping linkage not representable by a linkage tree.
These problems have the highest crossover usage and lowest crossover success of any problem except Deceptive
Step Trap.  While Deceptive Step Trap's epistasis can be accurately modeled by a linkage tree, the
exponential number of plateaus makes detecting gene linkage very challenging.

When compared with LTGA, P3's crossover success rates are lower but similar in problem ordering.
Counterintuitively, the use of hill climbing on the initial population reduces
P3's crossover success, not because it reduces model quality or the donation pool, but because it is much more challenging to improve
locally optimal solutions than randomly generated ones. LTGA's crossover benefits from application to unoptimized solutions, which
makes its aggregate crossover success incomparable to P3's.

Even when crossover success rates are quite low, such as Nearest Neighbor NK's 0.007\% success,
the results from Section~\ref{sec-optimum} and Section~\ref{sec-overtime} show it's still critical to optimization. Without
crossover, P3's performance would be identical to the Random Restart Hill Climber, which was unable to solve even moderately sized problems
and quickly fell behind P3 in intermediate fitness quality. Therefore even infrequently successful crossover donations are
critical to success. This does, however, suggest a potential avenue for future improvement by using more successful
modeling and donation algorithms.

\subsection{Pyramid}
\begin{figure}[t]
  \begin{centering}
    \begin{subfigure}{.5\textwidth}
      \begin{centering}
        \includegraphicsfit{level-size}
      \end{centering}
      \caption{Population Size}
      \label{fig-level-size}
    \end{subfigure}%
    \begin{subfigure}{.5\textwidth}
      \begin{centering}
        \includegraphicsfit{level-success}
      \end{centering}
      \caption{Crossover Success}
      \label{fig-level-success}
    \end{subfigure}
  \end{centering}
  \caption{For each problem Figure~\ref{fig-level-size} shows the number of solutions stored in each level of the pyramid
           and Figure~\ref{fig-level-success} shows the percentage of fitness improving crossover evaluations at each level.}
\end{figure}

Another feature unique to P3 is the shape and size of the population pyramid
constructed for each problem. Figure~\ref{fig-level-size} shows the number of solutions stored at
each level of the pyramid for the largest tested problem sizes. Each point is the
median size across 100 runs. If a run did not store any solutions at a level it is treated as 0.
No point is drawn if the median run had 0 solutions stored at that level. While pyramid size
is effected by problem size, the overall shape is not. As such the behavior shown in Figure~\ref{fig-level-size}
is representative of that for all other tested problem sizes.

With the exception of the dip in Deceptive Step Trap, all of the pyramids show a monotonic reduction
in size as the level increases. This is because a solution must be a strict fitness improvement
over its previous version to be added to a higher level, which becomes less likely
each time the solution improves. \cite{lobo:2011:dynamicpop} found theoretical evidence
and \cite{goldman:2011:dynamic-parameters} found empirical evidence that the optimal
population size decreases each generation of traditional evolutionary search.
By decreasing in size, P3 implicitly stores more diversity in low levels and focuses
search on high quality solutions at higher levels. In comparison,
LTGA and hBOA suboptimally use a fixed population size at each generation.

Figure~\ref{fig-level-success} examines how crossover success changes at different
levels of the pyramid. At low levels, success gets progressively lower
as solution quality increases faster than the model's ability to improve solutions.
At higher levels modeling becomes more accurate and donations contain higher frequencies
of high quality building blocks, resulting in increased crossover success. The
highest level of most problems has a low crossover success rate, as solutions
crossing with that level have already been improved by previous operations to
the point where the only improvement would be to create the global optimum, which can
only happen once.

\subsubsection{Deceptive Step Trap}
The number of solutions stored at the fourth level of Deceptive Step Trap is
significantly lower than that of the third or fifth levels, breaking the decreasing
trend of the other six problems. Figure~\ref{fig-level-success} has a similar
aberration, with crossover success dropping to  0.0004\% before rebounding and
following the more common trajectory. This behavior exists in all other problem
sizes tested, with the dips occurring at exactly the same level.

This behavior is rooted in the peculiar nature
of this landscape.
After local search, all traps in all solutions have a total
number of 1 bits equal to \{0, 1, 3, 5, 7\} as these correspond to the local
optima when using $k=7$ and $s=2$. Crossover is very likely to overcome the
two bit plateaus, and as a result solutions in the second level generally do not
contain the lowest fitness local optima (5 bits set) and the third
level has very few traps set to the next worst local optima (3 bits set). As a result,
solutions which reach the third level can only be improved by replacing the deceptive
local optima (0 and 1 bits set) with the global optimum (7 bits set).
The global optimum is very rare in the population, and with 8 ways to represent local optima
linkage learning is inaccurate. Therefore it is very unlikely for crossover to be successful,
meaning few solutions will be added to the fourth level. Solutions that do improve by
definition must have a higher frequency of optimal trap settings, meaning level four's model
will be more accurate and donations are more likely to contain optimal trap values.
Thus the level size and crossover success rates increase after contracting around level four.
