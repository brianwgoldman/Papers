%%% Abstract:
\thispagestyle{empty} \setcounter{page}{2}
\begin{doublespace}
\begin{centering}
ABSTRACT\\ %
\MyTitle\\ %
By \\ %
Brian W. Goldman\\ %
\end{centering}

The Parameter-less Population Pyramid (P3) is a recently introduced method for performing
evolutionary optimization without requiring any user specified parameters.
P3's primary innovation is to replace the generational model with a pyramid of
multiple populations that are iteratively created and expanded. In combination
with local search and advanced crossover, %wfp comma 
P3 scales to problem difficulty, exploiting
previously learned information before adding more diversity.

Across seven problems, each tested using on average 18 problem sizes, P3 outperformed
all five advanced comparison algorithms. This improvement includes
requiring 
%wfp change less 
fewer evaluations
and less wall clock seconds to find the global optimum and better fitness when using
the same number of evaluations. Using both algorithm analysis and
comparison we 
% wfp change find 
show P3's
effectiveness is due to its ability to properly maintain, add, and exploit diversity.

Unlike the best comparison algorithms, P3 was able to achieve this
quality 
% wfp emphasis
\textit{without any
problem specific tuning}. Thus, unlike previous parameter-less methods, P3 does not
sacrifice quality for applicability. Therefore we conclude that
P3 is an efficient, general, parameter-less approach to black-box
optimization that is more effective than existing state-of-the-art techniques.

% wfp
Furthermore, P3 can be specialized for gray-box problems, problems with known
linkage such as those with known, limited, non-linear relationships
between variables.

% wfp
% P3 can be further improved by specializing to the gray-box domain, which includes
% partial solution evaluation and known, limited, non-linear
% relationships between variables.

% wfp This 
Gray-Box P3 leverages the Hamming-Ball Hill Climber, an exceptionally efficient
form of local search, as well as a novel method for performing crossover using the
known variable interactions. In doing so Gray-Box P3 is able to find the global
optimum of large problems in seconds, improving over Black-Box P3 by up to
two orders of magnitude.
\end{doublespace}
\newpage
