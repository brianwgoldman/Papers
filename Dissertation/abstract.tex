%%% Abstract:
\thispagestyle{empty} \setcounter{page}{2}
\begin{doublespace}
\begin{centering}
ABSTRACT\\ %
\MyTitle\\ %
By \\ %
Brian W. Goldman\\ %
\ \\
\end{centering}

The Parameter-less Population Pyramid (P3) is a recently introduced method for performing
evolutionary optimization without requiring any user specified parameters.
P3's primary innovation is to replace the generational model with a pyramid of
multiple populations that are iteratively created and expanded. In combination
with local search and advanced crossover P3 scales to problem difficulty, exploiting
previously learned information before adding more diversity.

Across seven problems, each tested using on average 18 problem sizes, P3 outperformed
all five advanced comparison algorithms. This improvement includes requiring less evaluations
and less wall clock seconds to find the global optimum and better fitness when using
the same number of evaluations. Using both algorithm analysis and comparison we find P3's
effectiveness is due to its ability to properly maintain, add, and exploit diversity.

Unlike the best comparison algorithms, P3 was able to achieve this quality without any
problem specific tuning. Thus, unlike previous parameter-less methods, P3 does not
sacrifice quality for applicability. Therefore we conclude that
P3 is an efficient, general, parameter-less approach to black box
optimization which is more effective than existing state-of-the-art techniques.

We propose to extend this research to include the ``gray box'' problem domain, which includes many
interesting real world problems. We plan to develop a variant of P3 which exploits
features of this domain, such as efficient local search and known variable linkage.
Furthermore, we want to examine how ``hamming ball search'' can be integrated into
global optimization, both through simple crossover and as part of P3.
Finally, we plan compare this P3 variant with state of the art black box, gray box, and
white box methods for solving challenging real world problems.
\end{doublespace}
\newpage
