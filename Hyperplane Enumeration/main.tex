
%%%%%%%%%%%%%%%%%%%%%%% file typeinst.tex %%%%%%%%%%%%%%%%%%%%%%%%%
%
% This is the LaTeX source for the instructions to authors using
% the LaTeX document class 'llncs.cls' for contributions to
% the Lecture Notes in Computer Sciences series.
% http://www.springer.com/lncs       Springer Heidelberg 2006/05/04
%
% It may be used as a template for your own input - copy it
% to a new file with a new name and use it as the basis
% for your article.
%
% NB: the document class 'llncs' has its own and detailed documentation, see
% ftp://ftp.springer.de/data/pubftp/pub/tex/latex/llncs/latex2e/llncsdoc.pdf
%
%%%%%%%%%%%%%%%%%%%%%%%%%%%%%%%%%%%%%%%%%%%%%%%%%%%%%%%%%%%%%%%%%%%


\documentclass[runningheads,a4paper]{llncs}
%\linespread{2.5}
\usepackage{amssymb}
\setcounter{tocdepth}{3}
\usepackage{graphicx}
\usepackage{algorithm}
\usepackage[noend]{algpseudocode}
\renewcommand{\algorithmicrequire}{\textbf{Input:}}
\renewcommand{\algorithmicensure}{\textbf{Output:}}

\usepackage{url}
\newcommand{\keywords}[1]{\par\addvspace\baselineskip
\noindent\keywordname\enspace\ignorespaces#1}

\newcommand{\includegraphicsfit}[1]
{\includegraphics[width=\columnwidth,height=\textheight,keepaspectratio]{#1}}

\newcommand{\BigO}[1]{$\mathcal{O}{(#1)}$}

\begin{document}

\mainmatter  % start of an individual contribution

% first the title is needed
\title{Hyperplane Elimination for Quickly Enumerating Local Optima}

% a short form should be given in case it is too long for the running head
%\titlerunning{Lecture Notes in Computer Science: Authors' Instructions}

% the name(s) of the author(s) follow(s) next
%
% NB: Chinese authors should write their first names(s) in front of
% their surnames. This ensures that the names appear correctly in
% the running heads and the author index.
%
%\author{Brian W.~Goldman\and William F. Punch}
\author{Anonymous Author\and Anonymous Author}

%
%\authorrunning{Lecture Notes in Computer Science: Authors' Instructions}
% (feature abused for this document to repeat the title also on left hand pages)

% the affiliations are given next; don't give your e-mail address
% unless you accept that it will be published
%\institute{BEACON Center for the Study of Evolution in Action,\\
%Michigan State University, U.S.A.\\
%brianwgoldman@acm.org, punch@msu.edu}
\institute{Department\\ Organization\\ Email}

%
% NB: a more complex sample for affiliations and the mapping to the
% corresponding authors can be found in the file "llncs.dem"
% (search for the string "\mainmatter" where a contribution starts).
% "llncs.dem" accompanies the document class "llncs.cls".
%

%\toctitle{Lecture Notes in Computer Science}
%\tocauthor{Authors' Instructions}
\maketitle


\begin{abstract}
TODO

%The abstract should summarize the contents of the paper and should
%contain at least 70 and at most 150 words. It should be written using the
%\emph{abstract} environment.
\keywords{Landscape Understanding, Gray-Box, Mk Landscapes}
\end{abstract}


\section{Introduction}
The ruggedness and high dimensionality of most problem landscapes makes them challenging
analyze and understand. However, doing so can be helpful in
quantifying the difficulty of a problem, and how to design algorithms to deal
with those difficulties. 
Similarly, knowing which characteristics favor
a particular
algorithm can help researchers choose the algorithm
most likely to perform well on their problem.

A common way of analyzing landscapes is to
examine the frequency and distribution of local optima~\cite{boese:1994:bigvalley},
and how operators allow search to transition between these
optima~\cite{tomassini:2008:nknetworks,verel:2011:nknetworks,ochoa:2015:crossovernetworks}.
However, finding all of the local optima is prohibitively time consuming
even for small problems, with many studies limited to 18-bit
problems~\cite{tomassini:2008:nknetworks,verel:2011:nknetworks}.
By leveraging recent advancements in Gray-Box optimization that allow for
constant time local search~\cite{chicano:2014:ball}, this limit was raised
to 30-bit problems~\cite{ochoa:2015:crossovernetworks}. While sampling
methods can approximate some of these statistics for much larger
landscapes~\cite{iclanzan:2014:somnetworks}, they do not provide enough
detail necessary for some metrics~\cite{ochoa:2015:crossovernetworks}.

Here we introduce a method for finding all local optima of problems
with up to 80 bits. The cornerstone of this method is the identification
of hyperplanes that cannot contain any local optima, allowing large
portions of the search space to be skipped during enumeration.
The techniques used apply to the generalized problem class of Mk Landscapes,
which contains many real world and benchmark combinatorial problems.



\section{Mk Landscapes as a tool for Problem Generalization}
An Mk Landscape~\cite{whitley:2015:mk} is any function
who's value is equal to the sum of a set of subfunctions.
Each subfunction uses a small subset of variables from the original function's input.
Combined with limitations on the number of subfunctions and the size of each subfunction's
subset, Mk Landscapes allow for highly efficient local~\cite{whitley:2012:constant,chicano:2014:ball}
and global~\cite{goldman:2015:GBO,tintos:2015:partitioncross} search.

Formally, an Mk Landscape is any function $f : \mathbb{B}^{N}\rightarrow \mathbb{R}$
that can be expressed in the following form:
\begin{equation}
  f(x) = \sum_{i=1}^{M} f_i(mask(x, s_i))
  \label{eq-mk}
\end{equation}
In this equation $f_i$ is a subfunction such that $f_i : \mathbb{B}^{|s_i|}\rightarrow \mathbb{R}$.
Each $s_i$ is a set of variables in $x$, such that $|s_i| \leq k$.
The $mask$ function
returns the values in $x$ associated with each variable in $s_i$.
The total number of subfunctions $M$ is constrained to grow at \BigO{N}, and $k$
is constant with repect to $N$.

This formulation can represent many problems of real-world interest, as well as
the most commonly used combinatorial benchmark problems. In this work we
examine 5 problems in particular: Concatenated Traps, Adjacent and Random NKq Landscapes,
Ising Spin Glasses, and MAX-kSAT.

The Concatenated Trap problem~\cite{deb:1992:trap} is a composition of $k$-order deceptive
separable subfunctions. In Mk Landscape terms, $M=N/k$ such that $\forall_i |s_i| = k$ and
$\forall_{i \neq j} s_i \cap s_j = \emptyset$. Each $f_i$ applies an identical function based
on the number of variables set to 1:
\begin{equation}
   trap(t) = \left\{
     \begin{array}{rl}
       k-1-t, &  t<k\\
       k,   &  t = k
     \end{array}
   \right.
  \label{eq-trap}
\end{equation}
While this problem is still in common use~\cite{hsu:2015:dsmgaII,inoue:2015:adaptivep3},
most advanced methods can solve it trivially~\cite{goldman:2012:ltga} and when expressed
as an Mk Landscape it can be solved exactly in \BigO{N} time~\cite{whitley:2015:mk}.
We therefore include it only because its structure allows for straight forward algorithm analysis.
Here we set $k=5$.

NKq landscapes specify a class of randomly generated problem instances using 3 paramters:
(1) the number of problem variables $N$ (2) the amount of variable epistasis $K$ where $K=k-1$
and (3) the number of unique fitness values $q$. From these parameters a landscape is generated
by creating $M=N$ subfunctions $f_i$, where $f_i$ uses variable $x_i$ and $K$ others to look up
a fitness value in the range $[0..q-1]$ from a randomly generated table. This structure
makes NKq landscapes a natural fit for Mk landscapes (sum of bounded subfunctions)
and are the most studied problem class for Gray-Box
optimization~\cite{whitley:2012:constant,chicano:2014:ball,goldman:2015:GBO,tintos:2015:partitioncross,ochoa:2015:crossovernetworks,whitley:2015:mk}.
Here we choose $k=3$ with $q=2^{K+1}=2^{k}=8$.

We use two common variants of NKq in our experiments which specify how the $K$ additional
variables in each subfunction are chosen: Adjacent and Random. In Adjacent NKq, $f_i$ depends
on variable indices $[i..(i+k) \bmod N]$. In Random NKq, $f_i$ depends on $x_i$ and a random
set of $K$ unique variables that does not include $x_i$. While Random NKq is NP-Hard,
the structure of Adjacent NKq allows for a polynomial time dynamic programming solution~\cite{wright:2000:solvingnk}.
Adjacent NKq is therefore often used to understand a search algorithm's ability to reach the known global
optimum.

Ising Spin Glasses are a type of MAX-CUT problem derived from statistical physics.
Each atom in the glass (vertex) can be assigned a spin, with the goal being to
find the set of assignments that minimize the energy between nearby atoms (edges).
Similar to Adjacent NKq, the $2D\pm J$ subset of Ising Spin Glasses can be polynomially
solved~\cite{saul:1994:spinglass}\footnote{\url{http://www.informatik.uni-koeln.de/spinglass/}}.
In this subset, the graph is defined as a square two-dimensional torus with each edge
weight chosen from $\{-1, 1\}$. Each vertex is assigned a spin from $\{-1, 1\}$ with
the energy in the glass equal to
\begin{equation}
\sum_{e_{ij} \in E} x_ie_{ij}x_j
  \label{eq-ising}
\end{equation}
where $e_{ij}$ is the weight of the edge vertex $i$ to vertex $j$. In Mk landscape terms
this type of spin glass has $M=2N$ and $k=2$.

Our final examined problem is randomly generated maximum satisfiability or MAX-kSAT.
This version of the canonical NP-Complete boolean satisfiability problem is formulated
as the maximization of $M=4.27N$ clauses, each containing exactly $k=3$ unique literals.
A clause is satisfied if any of its literals match how a solution's variables are set.

\section{Gray-Box Enumeration of Mk Landscapes}
When considered as a black box, the process of finding all local optima
in a landscape requires \BigO{N2^N} time. This is because evaluating
a solution requires $N$ time, and all $2^N$ solutions in the landscape
must be evaluated once. Finding the local
optima then requires each solution to compare its fitness with each
of its neighbors, which when properly cached takes \BigO{N} per solution.
Extending this method to look for solutions that cannot be improved by
flipping $r$ or fewer bits requires \BigO{N^r2^N} time. However,
by exploiting Gray-Box optimization methods, previous work~\cite{ochoa:2015:crossovernetworks}
was able to find all $r$-bit local optima of Mk Landscapes in \BigO{2^N} for any small,
fixed value of $r$.

The first major result in Gray-Box optimization was the proof that the list
of fitness-improving moves can be updated after each flip in local search in \BigO{1}
time per bit flip~\cite{whitley:2012:constant}. In an Mk Landscape variables
can only have a non-linear relationship if they appear together in at least one $s_i$.
The fitness effect of flipping a bit can therefore only change when it has a non-linear
relationship with the bit modified by local search. By definition the number of non-linear
relationships in an Mk Landscape is linear with $N$, meaning on average the amortized
number of updates is \BigO{1}.

This result was extended by~\cite{chicano:2014:ball} to include local search for local
optima that cannot be improved using $r$ or fewer bits with no increase in asymptotic costs.
Consider two variables $x_i$ and $x_j$ that do not appear in the same $s_i$.
By definition the fitness effect of flipping both $x_i$ and $x_j$ is equal to the sum
of flipping each independently. As a result if neither individual flip is fitness improving,
flipping both together cannot be fitness improving. By examining the graph of non-linear
interactions between variables this principle can be extended to any collection of $r$
variables, with the number of useful collections growing at \BigO{N} when $r$ is a small constant.
As a result the time to update the list of improving moves after a modification is still \BigO{1}.

These advances can be applied directly to the task of finding all local optima in a
landscape~\cite{ochoa:2015:crossovernetworks}.
Instead of requiring \BigO{N} time to evaluate each solution and to check if it
is a local optima, only \BigO{1} time is needed. Consider an enumeration ordering of the landscape
that uses gray-codes, meaning that each transition between solutions requires exactly 1 bit flip.
In Gray-Box optimization updating fitness and the list of improving moves after a single bit flip
takes \BigO{1} time. That property holds even when looking for $r$-bit fitness-improving moves.
Therefore Gray-Box enumeration is able to find all local optima in \BigO{2^N} time.

\section{Hyperplane Elimination}
\begin{figure*}
  \centering
  \includegraphicsfit{Enumerate}
  \caption{Example change of enumeration ordering. The gray variables represent all dependencies
           for some move $m_i$. By reordering, $m_i$'s lowest $index$ dependency improves from 2 to 4.}
  \label{fig-enumerate}
\end{figure*}

Due to the limited non-linearity of the Gray-Box domain, it is possible to exclude large
parts of the search space without missing any local optima.
Consider the representation presented in the top of Figure~\ref{fig-enumerate}. In a Black-Box
domain, enumeration would progress as a binary counter, treating $index$ zero (symbol $A$ in the solution) as
the least significant bit. This ensures that before changing $index$ $i$, all possible settings of $index$
0 through $i-1$ have been tested. This corresponds to examining the hyperplane with the lowest $i-1$ positions
variable and all other positions fixed. The Gray-Box domain makes it possible to skip hyperplanes
that cannot be local optima. In Figure~\ref{fig-enumerate} there exists a move $m_i$ that is a fitness improvement
when enumeration starts (all variables set to 0). Due to the known relationships between variables,
we know that the quality of $m_i$ only depends on variables $C$, $E$, $F$, and $H$.
Therefore, until one of those four variables are modified, the solution cannot be a local optimum.
As a result, the hyperplane **0*00*0 cannot contain a local optima and can
therefore be eliminated from consideration during enumeration.
More generally, if at any point during enumeration
there exists a fitness-improving move, no local optima can exist until at least one
dependency of that move is modified. Any hyperplane that has all of a fitness-improving
move's dependencies fixed cannot contain any local optima.

\begin{algorithm}
  \caption{Find all local optima using Hyperplane Elimination.}
  \label{alg-enumerate}
  \begin{algorithmic}[1]
    \Require $solution \leftarrow \{0\}^N$
    \Require $move\_bin$ (List of sets of improving moves in $solution$)
    \Require \textproc{MakeMove} (Function that flips $index$ in $solution$ and updates $move\_bin$)
    \Ensure $found$ (list of all local optima)
    \State $found \leftarrow [~]$
    \State $index \leftarrow N-1$
    \While{$index < N$}
      \While{$index > 0$ \textbf{and} $move\_bin[index] = \emptyset$}\label{alg-enumerate-bincheck}
        \State $index \leftarrow index-1$
      \EndWhile
      \If{$index = -1$}
        \State $found \leftarrow found + [solution]$
        \State $index \leftarrow 0$
      \EndIf
      \While{$index < N$ \textbf{and} $solution[index] = 1$}\label{alg-enumerate-counter}
        \State \Call{MakeMove}{$index$}
        \State $index \leftarrow index + 1$
      \EndWhile
      \If{$index < N$}
        \State \Call{MakeMove}{$index$}\label{alg-enumerate-insert}
      \EndIf
    \EndWhile
  \end{algorithmic}
\end{algorithm}

This knowledge can be exploited to skip parts of the enumeration,
as shown in Algorithm~\ref{alg-enumerate}.
Before starting, each fitness-improving move is put into a table $move\_bin$
based on that move's lowest $index$ dependency. This is the first $index$ that
can be modified by enumeration to change the fitness effect of making that move.
In order to determine how much of the enumeration can be skipped, we must find
the highest $index$ in $move\_bin$ that contains a fitness-improving move,
as done by Line~\ref{alg-enumerate-bincheck}. If no move is fitness-improving,
then a local optimum has been found.

Algorithm~\ref{alg-enumerate} works by finding the highest $index$ in $move\_bin$
that is not empty and then adding a 1 to that $index$ in $solution$. Initially
all bins could contain a fitness-improving move, so $index$ starts at $N-1$.
If at any point all bins are empty then the solution is added to the list of
local optima $found$. Algorithm~\ref{alg-enumerate} then
adds a 1 to $index$ using the loop on Line~\ref{alg-enumerate-counter}
to perform carry operations and Line~\ref{alg-enumerate-insert} to create the new 1 value.
Iteration stops when the carry exceeds the solution length.

When performing subsequent iterations, not all bins need to be checked. Instead, the highest $index$
bin that must be tested is the highest $index$ flipped by the previous iteration. This
simplification is possible because
the previous iteration has verified that all moves in higher $index$ bins are not fitness improving, and no action performed during
that iteration can make them fitness improving.
Furthermore, $index$ is always the location of the least significant 1 in $solution$, meaning
iteration can continue immediately from the found $index$. This is true by construction
as initially $solution$ contains all 0s. 1s are always inserted at the value of $index$
with $index$ only increasing due to carry operations which clear all 1s lower than $index$.

This hyperplane skipping extends to finding only $r$-bit local optima for all Mk landscapes.
By increasing the number of potential fitness-improving moves, the number of hyperplanes that
can be eliminated increases. However, the cost trade-off of increasing $r$ is unclear.
While more solutions can be skipped, the time required to update $move\_bin$ each iteration
also increases. Therefore, the runtime effect of increasing $r$
depends on the specific problem.

\section{Improved Enumeration Ordering}
As a final efficiency, the order in which variables in $solution$ are $index$ed can be reordered.
When a move is fitness-improving,
the amount of search space that is skipped depends on how high its lowest $index$ dependency is. Therefore,
by rearranging the order to make its lowest $index$ dependency higher, more search space can be skipped. 
Figure~\ref{fig-enumerate} shows how changing the $index$ order of variables
improves $m_i$'s lowest $index$ dependency from 2 to 4. Consider that before reordering $m_i$ allows the hyperplane
**000000 to be skipped by Algorithm~\ref{alg-enumerate}, while after reordering that becomes ****0000.
Now whenever $m_i$ is a fitness improvement, Line~\ref{alg-enumerate-bincheck} in Algorithm~\ref{alg-enumerate}
skips 4 times as many solutions.

We perform this reordering in a greedy fashion, such that the move with the
least-unmapped dependencies has all of its remaining dependencies mapped to the
most significant remaining positions. With proper bookkeeping this algorithm requires
\BigO{N} time. Assuming that all moves are equally likely to be fitness improving
during enumeration, we argue below that the optimal bit ordering for skipping solutions is a greedy
ordering.

First we show that all optimal orderings of bits are optimal orderings of packed moves.
Consider that lowering a bit in enumeration order can only decrease the number of skipped
solutions if doing so decreases the minimum dependency of a move. However, doing so may
allow the minimum dependency of some unrelated move to increase. Therefore when ordering
bits, all that matters is the order in which the last bit of each move is assigned.

% Proof
Moves can have their dependencies fixed in some permutation $O$ of all possible moves,
such that $O_0$ is the first move to have its dependencies fixed.
$D_O(i, m)$ is a function that returns how many unfixed dependencies move $m$ has after all
moves in $O$ before $i$ have been fixed. A greedy solution to this problem is one such
that $O_i$ is set to be the move that minimizes $D_O(i, m)$.

All non-greedy solutions must have some $i$ such that $D_O(i, O_i) > D_O(i, O_{i+1})$.
This is because $D_O(i, m)$ can only decrease as $i$ increases, meaning that if some $m$
has a lower value at $i$ than $O_i$, this property must hold for
some index between $i$ and when $m$ appears in $O$.
Finally, all non-greedy solutions must have some $\hat{i}$
that is the maximum $i$ value for which $D_O(i, O_i) > D_O(i, O_{i+1})$ is true. 

Consider an optimal ordering $O^\star$ that is not greedy.
These criteria mean that $D_{O^\star}(\hat{i}, O_{\hat{i}}) > D_{O^\star}(\hat{i}, O_{\hat{i}+1})$.
Swapping $O_{\hat{i}}$ and $O_{\hat{i}+1}$ cannot change the minimum dependency of any other moves.
Performing the swap improves the minimum dependency of $O_{\hat{i}+1}$ more than it hurts
$O_{\hat{i}}$. Therefore, performing this swap makes $O^\star$ skip even more solutions,
contradicting the assertion that it was optimal. Therefore there cannot exist an optimal ordering that is not greedy.

\section{Complexity Classes for Simple Landscapes}
Understanding the runtime complexity of Algorithm~\ref{alg-enumerate} requires
knowledge of how often each move is fitness improving. For most landscapes
this is intractable to do theoretically. However, for some restricted problem
types its possible to rigorously determine Algorithm~\ref{alg-enumerate}'s complexity class.

\subsection{Linear Functions}
A linear function is any $f : \mathbb{B}^{N}\rightarrow \mathbb{R}$ which contains
no non-linear terms between variables. These functions are of the form:
\begin{equation}
  f(x) = \sum_{i=0}^{N-1} w_ix_i
  \label{eq-linear}
\end{equation}
In MK Landscape terms, $M=N$ and $k=1$. The most well known linear function is OneMax,
in which $w_i=1$ for all $i$.

When Algorithm~\ref{alg-enumerate} is applied to a linear
function, each index of $move\_bin$ can contain at most 1
move, and a move is fitness improving whenever it disagrees with the global optimum.
Initially $index$ is set to $N-1$ and is decreased by at least 1 in any step after the first.
Before finding the global optimum no carry operations can occur, meaning the
cost of each iteration is equal to the amount by which $index$ is decreased.
Each step will flip a bit which disagreed with the global optimum.
Therefore, Algorithm~\ref{alg-enumerate} requires \BigO{N} time to reach
the global optimum.

After finding the global optimum, $index$ is set to 0. One iteration
is then spent adding a 1 to $index$ 0, which in the worst case requires \BigO{N}
carry operations. For all future iterations $index$ is a fitness-improving move
and is currently set to 1. In these iterations at least 1 carry operation must
occur, and $index$ cannot be decreased. Iteration ends when $index$ exceeds $N$
which requires at most $N$ carry operations. Therefore, Algorithm~\ref{alg-enumerate}
requires \BigO{N} time to reach termination after finding the global optimum. Combined
with initialization and the time to find the global optimum, the total complexity is \BigO{N},
which is optimal.

\subsection{$k$-Bound Separable Problem}
Consider any problem that is composed of non-overlapping
subfunctions, each using using $k$ or fewer bits. Regardless
of how they are initially ordered, reordering ensures that
variables of a function are always consecutively placed in enumeration.
Consider a numbering of $f_i$ such that if $i < j$ then all of $f_i$'s
variables appear before $f_j$'s in enumeration ordering.

While any $f_i$ contains a fitness-improving move, Algorithm~\ref{alg-enumerate}
finds the one with the highest $i$ and enumerates its variables until it no longer
contains an improving move. In total this enumeration requires less than $2^k$
steps. Each $f_i$ is considered sequentially, meaning the first local optimum
is found in \BigO{2^kN/k} time.

Once the first local optimum is found, Algorithm~\ref{alg-enumerate} proceeds
by finding all local optimum in larger and larger hyperplanes. Initially $f_0$
is enumerated to find all local optima in the hyperplane with all $f_i, i>0$
fixed. This process ends when a carry operation modifies a bit of $f_1$. At
this point $f_1$ is enumerated, such that each time $f_1$ contains no improving moves
$f_0$ is enumerated again. This finds all local optima in the hyperplane with $f_i, i>1$
fixed. We represent the time required to find all local optima in the hyperplane
with $f_j, j>i$ fixed as $T(i)$. $T(0)=2^k$ as all values of $f_0$ must be enumerated.
$T(i) = |l_i|*T(i-1)+2^k$ where $|l_i|$ is the number of ways $f_i$ can be set such
that it contains no fitness-improving moves. Each $l_i$ exposes a new hyperplane
that can contain local optima, which causes the recursive call to $T(i-1)$.
Assuming for simplicity that all subfunctions use exactly $k$ bits,
the time required to find all local optima is $T(N/k-1)$.
Assuming that $\forall_i |l_i|=c$, $T(N/k-1)=\sum_0^{N/k-1}2^kc^i<2^kc^{N/k}$.
The number of local optima in the landscape is $c^{N/k}$ meaning that Algorithm~\ref{alg-enumerate}
is within a constant factor ($2^k$) of optimal.

When looking for only $r$-bit local optima of $k$-bound separable problems,
Algorithm~\ref{alg-enumerate} continues to be within a constant of optimal.
For $r < k$ the additional moves reduce $c$ without increasing the cost
by more than a constant.
When $r \geq k$ there can only be one local optimum: the global optimum. Furthermore, as
only non-linearly related subsets of variables must be checked
for fitness improvements, no subsets can be added that are larger than $k$.
Therefore even if $r=N-1$, the number of subsets only grows at \BigO{N}.
As $c=1$ in this case, the cost of finding the single global optimum
only requires the \BigO{2^kN/k} time to find the first local optimum, which is
optimal.

Concatenated Traps is a commonly used $k$-bound separable problem
with $c=2$.  Algorithm~\ref{alg-enumerate} requires $2^k2^{N/k}$ time
to find all $2^{N/k}$ local optima in this problem. Note also that
in the degenerate case of $k=1$, a $k$-bound separable problem
is a linear function. As such, when $k=1$ and $c=1$,
\BigO{2^kN/k} = \BigO{N}.

\section{Timing Comparisons}
\subsection{Increasing problem size}
Consider including boxplots of largest size all are successful.

\subsection{The Effect of Radius}

\section{Examining Local Optima}

TODO Consider discussing neutral networks

\section{Conclusions}

\subsubsection*{Acknowledgments.} TODO Acknowledge BEACON.

\bibliographystyle{splncs03}
\bibliography{../main}

\end{document}
