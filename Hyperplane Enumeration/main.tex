
%%%%%%%%%%%%%%%%%%%%%%% file typeinst.tex %%%%%%%%%%%%%%%%%%%%%%%%%
%
% This is the LaTeX source for the instructions to authors using
% the LaTeX document class 'llncs.cls' for contributions to
% the Lecture Notes in Computer Sciences series.
% http://www.springer.com/lncs       Springer Heidelberg 2006/05/04
%
% It may be used as a template for your own input - copy it
% to a new file with a new name and use it as the basis
% for your article.
%
% NB: the document class 'llncs' has its own and detailed documentation, see
% ftp://ftp.springer.de/data/pubftp/pub/tex/latex/llncs/latex2e/llncsdoc.pdf
%
%%%%%%%%%%%%%%%%%%%%%%%%%%%%%%%%%%%%%%%%%%%%%%%%%%%%%%%%%%%%%%%%%%%


\documentclass[runningheads,a4paper]{llncs}
%\linespread{2.5}
\usepackage{amssymb}
\setcounter{tocdepth}{3}
\usepackage{graphicx}

\usepackage{url}
\newcommand{\keywords}[1]{\par\addvspace\baselineskip
\noindent\keywordname\enspace\ignorespaces#1}

\newcommand{\BigO}[1]{$\mathcal{O}{(#1)}$}

\begin{document}

\mainmatter  % start of an individual contribution

% first the title is needed
\title{Hyperplane Elimination for Quickly Enumerating MK Landscapes}

% a short form should be given in case it is too long for the running head
%\titlerunning{Lecture Notes in Computer Science: Authors' Instructions}

% the name(s) of the author(s) follow(s) next
%
% NB: Chinese authors should write their first names(s) in front of
% their surnames. This ensures that the names appear correctly in
% the running heads and the author index.
%
%\author{Brian W.~Goldman\and William F. Punch}
\author{Anonymous Author\and Anonymous Author}

%
%\authorrunning{Lecture Notes in Computer Science: Authors' Instructions}
% (feature abused for this document to repeat the title also on left hand pages)

% the affiliations are given next; don't give your e-mail address
% unless you accept that it will be published
%\institute{BEACON Center for the Study of Evolution in Action,\\
%Michigan State University, U.S.A.\\
%brianwgoldman@acm.org, punch@msu.edu}
\institute{Department\\ Organization\\ Email}

%
% NB: a more complex sample for affiliations and the mapping to the
% corresponding authors can be found in the file "llncs.dem"
% (search for the string "\mainmatter" where a contribution starts).
% "llncs.dem" accompanies the document class "llncs.cls".
%

%\toctitle{Lecture Notes in Computer Science}
%\tocauthor{Authors' Instructions}
\maketitle


\begin{abstract}
TODO

%The abstract should summarize the contents of the paper and should
%contain at least 70 and at most 150 words. It should be written using the
%\emph{abstract} environment.
\keywords{Landscape Understanding, Gray-Box}
\end{abstract}


\section{Introduction}
TODO Motivate using network stuff.

\section{MK Landscapes and Gray-Box Optimization}
TODO Explain all problems here

TODO Explain increased radius size~\cite{chicano:2014:ball}

\section{Gray-Box Enumeration}
TODO Explain how gray-box properties and gray coding can be used to
enumerate in \BigO{2^N} time.

\section{Hyperplane Elimination}

\section{Improved Enumeration Ordering}

\section{Timing Comparisons}

\section{Examining Local Optima}

\section{Conclusions}

\subsubsection*{Acknowledgments.} TODO Acknowledge BEACON.

\bibliographystyle{splncs03}
\bibliography{../main}

\end{document}
